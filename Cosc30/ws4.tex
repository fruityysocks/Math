\documentclass[12pt]{article}

\usepackage{fullpage}
\usepackage{mdframed}
\usepackage{colonequals}
\usepackage{algpseudocode}
\usepackage{algorithm}
\usepackage[most, breakable]{tcolorbox}
\usepackage[all]{xy}
\usepackage{proof}
\usepackage{mathtools}
\usepackage{bbm}
\usepackage{amssymb}
\usepackage{amsthm}
\usepackage{amsmath}
\usepackage{amsxtra}
\usepackage{enumitem}
\newcommand{\bb}{\mathbb}


\newtheorem{theorem}{Theorem}[section]
\newtheorem{theorem*}{Theorem}
\newtheorem{definition}[theorem]{Definition}
\newtheorem{corollary}{Corollary}[theorem]
\newtheorem{lemma}[theorem]{Lemma}
\newtheorem{prop}[theorem]{Proposition}
\newtheorem{remark}[theorem]{Remark}


\newtheorem*{exercisehelper}{Exercise.}
\newenvironment{exercise}[1]{%
  \IfBlankTF{#1}
    {\renewcommand{\exercisehelper}{\textbf{Exercise} \unskip}}
    {\renewcommand\exercisehelper{\textbf{Exercise #1}}}%
  \exercisehelper
}{\endexercisehelper}

\theoremstyle{remark}
\newtheorem*{solution}{Solution}
\newcommand{\mathcat}[1]{\textup{\textbf{\textsf{#1}}}} % for defined terms

\newenvironment{problem}[1]
{ \begin{tcolorbox}[breakable]\noindent\textbf{Problem #1}.}
{\vskip 6pt \end{tcolorbox}}

\newenvironment{enumalph}
{\begin{enumerate}\renewcommand{\labelenumi}{\textnormal{(\alph{enumi})}}}
{\end{enumerate}}

\newenvironment{enumroman}
{\begin{enumerate}\renewcommand{\labelenumi}{\textnormal{(\roman{enumi})}}}
{\end{enumerate}}

\newcommand{\defi}[1]{\textsf{#1}} % for defined terms



\setlength{\hfuzz}{4pt}

\let\H\relax
\let\P\relax
\newcommand{\H}{\mathbb H}
\newcommand{\P}{\mathbb P}
\newcommand{\C}{\mathbb C}
\newcommand{\N}{\mathbb N}
\newcommand{\Q}{\mathbb Q}
\newcommand{\R}{\mathbb R}
\newcommand{\Z}{\mathbb Z}
\newcommand{\F}{\mathbb F}
\newcommand{\br}{\mathbf{r}}
\newcommand{\RP}{\mathbb{RP}}
\newcommand{\CP}{\mathbb{CP}}
\newcommand{\nbit}[1]{\{0, 1\}^{#1}}
\newcommand{\bits}{\{0, 1\}^{n}}
\newcommand{\bbni}{\bigbreak \noindent}
\newcommand{\norm}[1]{\left\vert\left\vert#1\right\vert\right\vert}
\newcommand{\dbar}{\overline{\partial}}
\let\d\relax
\newcommand{\d}{\partial}
\newcommand{\calO}{\mathcal{O}}
\newcommand{\calF}{\mathcal{F}}
\newcommand{\calG}{\mathcal{G}}
\newcommand{\calH}{\mathcal{H}}
\newcommand{\calE}{\mathcal{E}}
\newcommand{\calC}{\mathcal{C}}
\newcommand{\calD}{\mathcal{D}}

\let\1\relax
\newcommand{\1}{\mathbf{1}}
\newcommand{\fr}[2]{\left(\frac{#1}{#2}\right)}
\newcommand{\todo}[1]{\textcolor{red}{\textbf{TODO:} #1}}
\newcommand{\vecz}{\mathbf{z}}
\newcommand{\vecr}{\mathbf{r}}
\DeclareMathOperator{\Cinf}{C^{\infty}}
\DeclareMathOperator{\Id}{Id}
\DeclareMathOperator{\Ell}{Ell}
\DeclareMathOperator{\CL}{\mathcal{CL}}

\DeclareMathOperator{\Alt}{Alt}
\DeclareMathOperator{\Aut}{Aut}
\DeclareMathOperator{\ann}{ann}
\DeclareMathOperator{\codim}{codim}
\DeclareMathOperator{\End}{End}
\DeclareMathOperator{\Hom}{Hom}
\DeclareMathOperator{\id}{id}
\DeclareMathOperator{\M}{M}
\DeclareMathOperator{\Mat}{Mat}
\DeclareMathOperator{\Ob}{Ob}
\DeclareMathOperator{\opchar}{char}
\DeclareMathOperator{\opspan}{span}
\DeclareMathOperator{\rk}{rk}
\DeclareMathOperator{\sgn}{sgn}
\DeclareMathOperator{\Sym}{Sym}
\DeclareMathOperator{\tr}{tr}
\DeclareMathOperator{\img}{img}
\DeclareMathOperator{\coker}{coker}
\DeclareMathOperator{\Spec}{Spec}
\DeclareMathOperator{\CandE}{CandE}
\DeclareMathOperator{\CandO}{CandO}
\DeclareMathOperator{\argmax}{argmax}
\DeclareMathOperator{\first}{first}
\DeclareMathOperator{\last}{last}
\DeclareMathOperator{\cost}{cost}
\DeclareMathOperator{\dist}{dist}
\DeclareMathOperator{\path}{path}
\DeclareMathOperator{\parent}{parent}
\DeclareMathOperator{\argmin}{argmin}
\DeclareMathOperator{\excess}{excess}
\let\Pr\relax
\DeclareMathOperator{\Pr}{\mathbf{Pr}}
\DeclareMathOperator{\Exp}{\mathbb{E}}
\DeclareMathOperator{\Var}{\mathbf{Var}}
\let\limsup\relax
\DeclareMathOperator{\limsup}{limsup}
%Paired Delims
\DeclarePairedDelimiter\ceil{\lceil}{\rceil}
\let\oldceil\ceil
\renewcommand{\ceil}[1]{\oldceil*{#1}}

\DeclarePairedDelimiter{\floor}{\lfloor}{\rfloor}
\let\oldfloor\floor
\renewcommand{\floor}[1]{\oldfloor*{#1}}





\newcommand{\dagstar}{*}

\newcommand{\tbigwedge}{{\textstyle{\bigwedge}}}
\setlength{\parindent}{0pt}
\setlength{\parskip}{5pt}


\usepackage{listings}
\usepackage{courier}
\usepackage{microtype}


\lstset{
  basicstyle=\footnotesize\ttfamily,
  breaklines=true,
  breakatwhitespace=true
  columns=fullflexible,
  keepspaces=true,
  frame=single,
  escapeinside={(*@}{@*)}
}

\begin{document}

\title{Math 71: Abstract Algebra}

\author{Prishita Dharampal}
\date{}
\maketitle


\begin{problem}{1}
Prove that any positive integer $n$ greater than $1$ can be written as a product of primes.
\end{problem}

\begin{solution}
    \bbni 

    Let $P(n) = \forall n \in \Z_{>1}: \text{n is a product of primes}$. 

    Assume that $P(k)$ holds where $ 1 < k < n$. I.e. the inductive hypothesis is that any integer greater than $1$ and less than $n$ can be written as a product of primes. 

    Then we have two cases: 
    \begin{enumerate}
        \item $n$ is prime: 

        Then $n$ is a product of itself, and we are done. 

        \item $n$ is not prime: 
        
        Then we can write $n$ as a product of two integers $a, b$. Then, 
        \begin{enumerate}
            \item $a,b$ are prime. 
            
            By definition, $n = a * b$. Hence, $n$ can be written as a product of primes. 

            \item Either one or both of $a,b$ is not prime. 
            
            Since $ b = \frac{n}{a}, a = \frac{n}{b}$ we can see that  $a, b < n$. Then by the induction hypothesis, both $a$ and $b$ can be written as products of primes. Therefore, their product $n = ab$ can also be written as a product of primes.

        \end{enumerate}   
        
    \end{enumerate}
\end{solution}

\newpage 

\begin{problem}{2}
Prove that for every real number $r \neq 1$ and every nonnegative integer $n$,
\[
\sum_{i=0}^{n} r^i = \frac{r^{n+1}-1}{r-1}.
\]
\end{problem}

\begin{solution}
    \bbni 

    Let $P(x)$ be the statement:
\[
\sum_{i=0}^{x} r^i = \frac{r^{x+1}-1}{r-1},
\]
for a fixed real number $r \neq 1$.

    Assume $P(x)$ holds upto $x = n -1$. Then we have two cases: $x =0$ or $x > 0$ 
    
    \begin{enumerate}
        \item Base Case: $x = 0$
    
        $\sum_{i=0}^{0} r^i = r^0 = 1$ 
        and  $\frac{r^{1} - 1}{r-1} = \frac{r-1}{r-1} = 1$. Hence, $P(0)$ holds. 

        \item $x = n$, then $P(n)$: 
        \begin{align*}
        \sum_{i=0}^{n} r^i &= \sum_{i=0}^{n-1} r^i + r^n \\ 
        &= \frac{r^{n}-1}{r-1} + r^n \\ 
        &= \frac{r^{n}-1}{r-1} + \frac{r^n(r -1)}{r -1} \\
        &= \frac{r^{n}-1 +r^{n+1} - r^n}{r -1} \\
        &= \frac{r^{n+1} -1 }{r -1} \\
    \end{align*} 

    \end{enumerate}
    Hence the proposition holds. 
\end{solution}

\newpage

\begin{problem}{3}
    Prove that, given an unlimited supply of $6$-cent coins, $10$-cent coins, and $15$-cent coins, one can make change for any amount that is at least $30$ cents. 
 
    \emph{[Hint: You need many cases.]}
\end{problem}

\begin{solution}
    \bbni

    Let $P(n)= \forall n \geq 30: n$ cents can be made using $6$, $10$, and $15$-cent coins. 

    Assume that for any integer $x$ such that $30 \leq x < n$, we can make $x$ cents using $6, 10, 15$-cent coins. 
    Then there are seven cases to consider: $x = 30, x = 31, x = 32, x = 33, x = 34, x = 35, x > 35$
    \begin{enumerate}
        \item $30$ = $10  + 10 + 10$   
        \item $31$ = $10  + 15 + 6$  
        \item $32$ = $10  + 10 + 6 + 6$  
        \item $33$ = $15  + 6 + 6 + 6$  
        \item $34$ = $10  + 6 + 6 + 6 + 6$  
        \item $35$ = $10  + 10 + 15$  
        \item $x > 35$
        
        If $n > 35$, then $n-6 \geq 30$. By the induction hypothesis, $n-6$ cents can be made. Adding one $6$-cent coin gives us a representation for $n$ cents.
    \end{enumerate} 
    Therefore for any $n \geq 30$, we can make $n$ cents using $6, 10, 15$-cent coins. 
\end{solution}
\end{document}