\documentclass[12pt]{article}

\usepackage{fullpage}
\usepackage{mdframed}
\usepackage{colonequals}
\usepackage{algpseudocode}
\usepackage{algorithm}
\usepackage[most, breakable]{tcolorbox}
\usepackage[all]{xy}
\usepackage{proof}
\usepackage{mathtools}
\usepackage{bbm}
\usepackage{amssymb}
\usepackage{amsthm}
\usepackage{amsmath}
\usepackage{amsxtra}
\usepackage{enumitem}
\newcommand{\bb}{\mathbb}


\newtheorem{theorem}{Theorem}[section]
\newtheorem{theorem*}{Theorem}
\newtheorem{definition}[theorem]{Definition}
\newtheorem{corollary}{Corollary}[theorem]
\newtheorem{lemma}[theorem]{Lemma}
\newtheorem{prop}[theorem]{Proposition}
\newtheorem{remark}[theorem]{Remark}


\newtheorem*{exercisehelper}{Exercise.}
\newenvironment{exercise}[1]{%
  \IfBlankTF{#1}
    {\renewcommand{\exercisehelper}{\textbf{Exercise} \unskip}}
    {\renewcommand\exercisehelper{\textbf{Exercise #1}}}%
  \exercisehelper
}{\endexercisehelper}

\theoremstyle{remark}
\newtheorem*{solution}{Solution}
\newcommand{\mathcat}[1]{\textup{\textbf{\textsf{#1}}}} % for defined terms

\newenvironment{problem}[1]
{ \begin{tcolorbox}[breakable]\noindent\textbf{Problem #1}.}
{\vskip 6pt \end{tcolorbox}}

\newenvironment{enumalph}
{\begin{enumerate}\renewcommand{\labelenumi}{\textnormal{(\alph{enumi})}}}
{\end{enumerate}}

\newenvironment{enumroman}
{\begin{enumerate}\renewcommand{\labelenumi}{\textnormal{(\roman{enumi})}}}
{\end{enumerate}}

\newcommand{\defi}[1]{\textsf{#1}} % for defined terms



\setlength{\hfuzz}{4pt}

\let\H\relax
\let\P\relax
\newcommand{\H}{\mathbb H}
\newcommand{\P}{\mathbb P}
\newcommand{\C}{\mathbb C}
\newcommand{\N}{\mathbb N}
\newcommand{\Q}{\mathbb Q}
\newcommand{\R}{\mathbb R}
\newcommand{\Z}{\mathbb Z}
\newcommand{\F}{\mathbb F}
\newcommand{\br}{\mathbf{r}}
\newcommand{\RP}{\mathbb{RP}}
\newcommand{\CP}{\mathbb{CP}}
\newcommand{\nbit}[1]{\{0, 1\}^{#1}}
\newcommand{\bits}{\{0, 1\}^{n}}
\newcommand{\bbni}{\bigbreak \noindent}
\newcommand{\norm}[1]{\left\vert\left\vert#1\right\vert\right\vert}
\newcommand{\dbar}{\overline{\partial}}
\let\d\relax
\newcommand{\d}{\partial}
\newcommand{\calO}{\mathcal{O}}
\newcommand{\calF}{\mathcal{F}}
\newcommand{\calG}{\mathcal{G}}
\newcommand{\calH}{\mathcal{H}}
\newcommand{\calE}{\mathcal{E}}
\newcommand{\calC}{\mathcal{C}}
\newcommand{\calD}{\mathcal{D}}

\let\1\relax
\newcommand{\1}{\mathbf{1}}
\newcommand{\fr}[2]{\left(\frac{#1}{#2}\right)}
\newcommand{\todo}[1]{\textcolor{red}{\textbf{TODO:} #1}}
\newcommand{\vecz}{\mathbf{z}}
\newcommand{\vecr}{\mathbf{r}}
\DeclareMathOperator{\Cinf}{C^{\infty}}
\DeclareMathOperator{\Id}{Id}
\DeclareMathOperator{\Ell}{Ell}
\DeclareMathOperator{\CL}{\mathcal{CL}}

\DeclareMathOperator{\Alt}{Alt}
\DeclareMathOperator{\Aut}{Aut}
\DeclareMathOperator{\ann}{ann}
\DeclareMathOperator{\codim}{codim}
\DeclareMathOperator{\End}{End}
\DeclareMathOperator{\Hom}{Hom}
\DeclareMathOperator{\id}{id}
\DeclareMathOperator{\M}{M}
\DeclareMathOperator{\Mat}{Mat}
\DeclareMathOperator{\Ob}{Ob}
\DeclareMathOperator{\opchar}{char}
\DeclareMathOperator{\opspan}{span}
\DeclareMathOperator{\rk}{rk}
\DeclareMathOperator{\sgn}{sgn}
\DeclareMathOperator{\Sym}{Sym}
\DeclareMathOperator{\tr}{tr}
\DeclareMathOperator{\img}{img}
\DeclareMathOperator{\coker}{coker}
\DeclareMathOperator{\Spec}{Spec}
\DeclareMathOperator{\CandE}{CandE}
\DeclareMathOperator{\CandO}{CandO}
\DeclareMathOperator{\argmax}{argmax}
\DeclareMathOperator{\first}{first}
\DeclareMathOperator{\last}{last}
\DeclareMathOperator{\cost}{cost}
\DeclareMathOperator{\dist}{dist}
\DeclareMathOperator{\path}{path}
\DeclareMathOperator{\parent}{parent}
\DeclareMathOperator{\argmin}{argmin}
\DeclareMathOperator{\excess}{excess}
\let\Pr\relax
\DeclareMathOperator{\Pr}{\mathbf{Pr}}
\DeclareMathOperator{\Exp}{\mathbb{E}}
\DeclareMathOperator{\Var}{\mathbf{Var}}
\let\limsup\relax
\DeclareMathOperator{\limsup}{limsup}
%Paired Delims
\DeclarePairedDelimiter\ceil{\lceil}{\rceil}
\let\oldceil\ceil
\renewcommand{\ceil}[1]{\oldceil*{#1}}

\DeclarePairedDelimiter{\floor}{\lfloor}{\rfloor}
\let\oldfloor\floor
\renewcommand{\floor}[1]{\oldfloor*{#1}}





\newcommand{\dagstar}{*}

\newcommand{\tbigwedge}{{\textstyle{\bigwedge}}}
\setlength{\parindent}{0pt}
\setlength{\parskip}{5pt}


\usepackage{listings}
\usepackage{courier}
\usepackage{microtype}


\lstset{
  basicstyle=\footnotesize\ttfamily,
  breaklines=true,
  breakatwhitespace=true
  columns=fullflexible,
  keepspaces=true,
  frame=single,
  escapeinside={(*@}{@*)}
}

\begin{document}

\title{Math 71: Abstract Algebra}

\author{Prishita Dharampal}
\date{}
\maketitle


\begin{problem}{1}
For any integer $n$, prove that $n^3$ is even if and only if $n$ is even.
\end{problem}

\begin{solution}
    \bbni 

    ($\implies$) 

    If $n^3$ is even, then by definition $n*n*n$ is even. Since,
    \begin{align*}
        even &* even = even \\ 
        even &* odd = even \\ 
        odd &* even = even \\ 
        odd &* odd = odd
    \end{align*} 
    for a product to be even at least one of the terms must be even. Hence, $n$ must be even. 

    ($\impliedby$)

    We know that for any even integer $a$, $a * a$ is even. I.e. if $n$ is even, then $n*n$ is even. Again, since $n*n$ is even, and $n$ is even, $(n*n)*n$ must be even. Hence, if $n$ is even $n^3$ is even. 
\end{solution}

\newpage 

\begin{problem}{2}
    $\sqrt[3]{4}$ is not a fraction; in other words, $\sqrt[3]{4}$ cannot be written as a ratio of two integers. 

    \textbf{Hint: Does your proof remain unchanged if we replace $\sqrt[3]{4}$ with $\sqrt[3]{8}$? We are in trouble}
\end{problem}

\begin{solution}
    \bbni 

    We will prove that $\sqrt[3]{4}$ isn't rational by contradiction. 
    
    Let $\sqrt[3]{4} = \frac{a}{b}$, such that $a, b \in \Z, \, b\neq 0, \,gcd (a,b) = 1$. Then, 
    \begin{align*}
        \sqrt[3]{4} &= \frac{a}{b} \\ 
        4 &= \left (\frac{a}{b} \right)^3 \\
        2^2b^3 &= a^3 \\
    \end{align*}
    Hence, $a^3$ is divisible by $2$. From (Problem 1) we know that if $a^3$ is even, then $a$ must be even. I.e. $a$ must be divisible by 2. Substitute $a = 2k$. 
    \begin{align*}
        2^2b^3 &= (2k)^3 \\
        2^2 b^3 &= 2^3 k^3 \\
        b^3 &= 2 k^3 \\
    \end{align*}
    Similarly, $b^3$ is divisible by $2$. From (Problem 1) we know that if $b^3$ is even, then $b$ must be even. I.e. $b$ must be divisible by 2. But we defined $gcd(a,b) = 1$. Hence this is a contradiction! $\sqrt[3]{4}$ cannot be written as a ratio of two integers. 

    \bbni 

    Checking the proof with $\sqrt[3]{8}$. Let $\sqrt[3]{8} = \frac{a}{b}$, such that $a, b \in \Z, \, b\neq 0, \,gcd (a,b) = 1$. Then, 
    \begin{align*}
        \sqrt[3]{8} &= \frac{a}{b} \\ 
        8 &= \left (\frac{a}{b} \right)^3 \\
        2^3b^3 &= a^3 \\
    \end{align*}
    Hence, $a^3$ is divisible by $2 \implies a$ must be divisible by 2. Substitute $a = 2k$. 
    \begin{align*}
            2^3b^3 &= (2k)^3 \\
            2^3 b^3 &= k^3 \\
            b^3 &= k^3 \\
    \end{align*}
    We can't move ahead from here, and the proof breaks. 
\end{solution}

\newpage 

\begin{problem}{3}
    Given an array $A[1 \ldots n]$, a value $x$ is called \emph{abundant} if there are more than one-hundredth of the elements in array $A$ with value equal to $x$.

    Prove that every array has at most $100$ distinct abundant values.
\end{problem}

\begin{solution}
    \bbni 

    Let $a \in \Z$ be the number of distinct abundant values in the array. By definition, an abundant value occurs in the array more than $n/100$ times. 

    Define an array $B[c_1, c_2, \cdots, c_k]$ to denote the number of times each abundant value appears in the array. Then we can say that, 
    \[\sum_{i=1}^{k}c_i > k \cdot \frac{n}{100} \]
    
    But also since there are $n$ elements in the array, 
    \[\sum_{i=1}^{k}c_i \leq n \]
    \begin{align*}
        &\implies k \cdot \frac{n}{100} \leq n \\
        &\implies k  \leq 100
    \end{align*}
   
    Therefore, there are at most $100$ distinct abundant values.
\end{solution}

\bbni 

\begin{problem}{4*}
    We call numbers like $\sqrt{2}$ which cannot be written as a fraction \emph{irrational}. Are there two irrational numbers $x$ and $y$ such that $x^y$ is rational? Why?
\end{problem}

\begin{solution}
    \bbni   

    Yes. 

    We know that square roots of prime numbers are irrational. Let $x = \sqrt{5}$ and $y = \sqrt{2}$. Then, $x^y$ is either rational, in which case we are done, or $x^y$ is irrational. If $x^y$ is irrational, redefine $x = \sqrt{5}^{\sqrt{2}}, y = \sqrt{2}$. Then \[x^y =  (\sqrt{5}^{\sqrt{2}})^{\sqrt{2}} = \sqrt{5}^{\sqrt{2} * \sqrt{2}} = \sqrt{5}^2 = 5 = \frac{5}{1}\] which is rational.
    However, I couldn't come up with a general proof for all irrational numbers.
\end{solution}


\end{document}