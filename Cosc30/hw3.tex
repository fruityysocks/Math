\documentclass[12pt]{article}

\usepackage{fullpage}
\usepackage{mdframed}
\usepackage{colonequals}
\usepackage{algpseudocode}
\usepackage{algorithm}
\usepackage[most, breakable]{tcolorbox}
\usepackage[all]{xy}
\usepackage{proof}
\usepackage{mathtools}
\usepackage{bbm}
\usepackage{amssymb}
\usepackage{amsthm}
\usepackage{amsmath}
\usepackage{amsxtra}
\usepackage{enumitem}
\newcommand{\bb}{\mathbb}


\newtheorem{theorem}{Theorem}[section]
\newtheorem{theorem*}{Theorem}
\newtheorem{definition}[theorem]{Definition}
\newtheorem{corollary}{Corollary}[theorem]
\newtheorem{lemma}[theorem]{Lemma}
\newtheorem{prop}[theorem]{Proposition}
\newtheorem{remark}[theorem]{Remark}


\newtheorem*{exercisehelper}{Exercise.}
\newenvironment{exercise}[1]{%
  \IfBlankTF{#1}
    {\renewcommand{\exercisehelper}{\textbf{Exercise} \unskip}}
    {\renewcommand\exercisehelper{\textbf{Exercise #1}}}%
  \exercisehelper
}{\endexercisehelper}

\theoremstyle{remark}
\newtheorem*{solution}{Solution}
\newcommand{\mathcat}[1]{\textup{\textbf{\textsf{#1}}}} % for defined terms

\newenvironment{problem}[1]
{ \begin{tcolorbox}[breakable]\noindent\textbf{Problem #1}.}
{\vskip 6pt \end{tcolorbox}}

\newenvironment{enumalph}
{\begin{enumerate}\renewcommand{\labelenumi}{\textnormal{(\alph{enumi})}}}
{\end{enumerate}}

\newenvironment{enumroman}
{\begin{enumerate}\renewcommand{\labelenumi}{\textnormal{(\roman{enumi})}}}
{\end{enumerate}}

\newcommand{\defi}[1]{\textsf{#1}} % for defined terms



\setlength{\hfuzz}{4pt}

\let\H\relax
\let\P\relax
\newcommand{\H}{\mathbb H}
\newcommand{\P}{\mathbb P}
\newcommand{\C}{\mathbb C}
\newcommand{\N}{\mathbb N}
\newcommand{\Q}{\mathbb Q}
\newcommand{\R}{\mathbb R}
\newcommand{\Z}{\mathbb Z}
\newcommand{\F}{\mathbb F}
\newcommand{\br}{\mathbf{r}}
\newcommand{\RP}{\mathbb{RP}}
\newcommand{\CP}{\mathbb{CP}}
\newcommand{\nbit}[1]{\{0, 1\}^{#1}}
\newcommand{\bits}{\{0, 1\}^{n}}
\newcommand{\bbni}{\bigbreak \noindent}
\newcommand{\norm}[1]{\left\vert\left\vert#1\right\vert\right\vert}
\newcommand{\dbar}{\overline{\partial}}
\let\d\relax
\newcommand{\d}{\partial}
\newcommand{\calO}{\mathcal{O}}
\newcommand{\calF}{\mathcal{F}}
\newcommand{\calG}{\mathcal{G}}
\newcommand{\calH}{\mathcal{H}}
\newcommand{\calE}{\mathcal{E}}
\newcommand{\calC}{\mathcal{C}}
\newcommand{\calD}{\mathcal{D}}

\let\1\relax
\newcommand{\1}{\mathbf{1}}
\newcommand{\fr}[2]{\left(\frac{#1}{#2}\right)}
\newcommand{\todo}[1]{\textcolor{red}{\textbf{TODO:} #1}}
\newcommand{\vecz}{\mathbf{z}}
\newcommand{\vecr}{\mathbf{r}}
\DeclareMathOperator{\Cinf}{C^{\infty}}
\DeclareMathOperator{\Id}{Id}
\DeclareMathOperator{\Ell}{Ell}
\DeclareMathOperator{\CL}{\mathcal{CL}}

\DeclareMathOperator{\Alt}{Alt}
\DeclareMathOperator{\Aut}{Aut}
\DeclareMathOperator{\ann}{ann}
\DeclareMathOperator{\codim}{codim}
\DeclareMathOperator{\End}{End}
\DeclareMathOperator{\Hom}{Hom}
\DeclareMathOperator{\id}{id}
\DeclareMathOperator{\M}{M}
\DeclareMathOperator{\Mat}{Mat}
\DeclareMathOperator{\Ob}{Ob}
\DeclareMathOperator{\opchar}{char}
\DeclareMathOperator{\opspan}{span}
\DeclareMathOperator{\rk}{rk}
\DeclareMathOperator{\sgn}{sgn}
\DeclareMathOperator{\Sym}{Sym}
\DeclareMathOperator{\tr}{tr}
\DeclareMathOperator{\img}{img}
\DeclareMathOperator{\coker}{coker}
\DeclareMathOperator{\Spec}{Spec}
\DeclareMathOperator{\CandE}{CandE}
\DeclareMathOperator{\CandO}{CandO}
\DeclareMathOperator{\argmax}{argmax}
\DeclareMathOperator{\first}{first}
\DeclareMathOperator{\last}{last}
\DeclareMathOperator{\cost}{cost}
\DeclareMathOperator{\dist}{dist}
\DeclareMathOperator{\path}{path}
\DeclareMathOperator{\parent}{parent}
\DeclareMathOperator{\argmin}{argmin}
\DeclareMathOperator{\excess}{excess}
\let\Pr\relax
\DeclareMathOperator{\Pr}{\mathbf{Pr}}
\DeclareMathOperator{\Exp}{\mathbb{E}}
\DeclareMathOperator{\Var}{\mathbf{Var}}
\let\limsup\relax
\DeclareMathOperator{\limsup}{limsup}
%Paired Delims
\DeclarePairedDelimiter\ceil{\lceil}{\rceil}
\let\oldceil\ceil
\renewcommand{\ceil}[1]{\oldceil*{#1}}

\DeclarePairedDelimiter{\floor}{\lfloor}{\rfloor}
\let\oldfloor\floor
\renewcommand{\floor}[1]{\oldfloor*{#1}}





\newcommand{\dagstar}{*}

\newcommand{\tbigwedge}{{\textstyle{\bigwedge}}}
\setlength{\parindent}{0pt}
\setlength{\parskip}{5pt}


\usepackage{listings}
\usepackage{courier}
\usepackage{microtype}


\lstset{
  basicstyle=\footnotesize\ttfamily,
  breaklines=true,
  breakatwhitespace=true
  columns=fullflexible,
  keepspaces=true,
  frame=single,
  escapeinside={(*@}{@*)}
}

\begin{document}

\title{Math 71: Abstract Algebra}

\author{Prishita Dharampal}
\date{}
\maketitle


\begin{problem}{1(a)}
    \textbf{Faulty induction.}  
    
    Induction proofs can break for many reasons. Careless summoning of the genie or misuse
    of the induction boilerplate may result in fake ``proofs'' for statements that are just plain false.
    Point out the most significant error from the following arguments and justify.

    Let $B$ be a set of blocks, arranged into stack(s). The blocks in each stack are in a linear
    order from bottom to top, with every block touching exactly one block from below and one
    from above (except for the top- and bottom-most blocks); two blocks from two separate stacks
    do not touch one another.

    Consider the ``proof'' of the following statement:  
    If every block in $B$ is touching at least one other block in $B$, then all blocks in $B$ are in
    a single stack.

    \begin{tcolorbox}
        We prove the theorem by induction on the number of blocks in $B$. Let $P(B)$ be the statement ``if every block in $B$ is touching at least one other block in $B$, then all blocks in $B$ are in a single stack.''

        Consider an arbitrary set of blocks $B$ arranged in stacks with exactly $n$ blocks, where
        every block in $B$ touches at least one other block in $B$. Assume as the induction hypothesis
        that $P(B')$ holds for any set of blocks $B'$ with fewer than $n$ blocks.
        \begin{itemize}
            \item When $B$ is empty, the statement ``every block in $B$ is touching at least one other block in $B$'' is vacuously true, and ``all blocks in $B$ are in a single stack'' is also vacuously true.
            \item Otherwise, $B$ is non-empty. Remove one block $b$ from $B$ to obtain a set $B'$ containing $n-1$ blocks. By the induction hypothesis on $B'$, all the blocks in $B'$ are in a single stack. Because in $P(B)$ every block in $B$ is assumed to touch at least one other block in $B$, block $b$ must touch another block in $B$ (which must be in $B'$). This means $b$ is in the same stack as all blocks in $B'$. Therefore, all blocks in $B$ are in a single stack.
        \end{itemize}
        In both cases $P(B)$ holds, so the theorem holds.
    \end{tcolorbox}
\end{problem}

\begin{problem}{1(b)}

    Recall the definition of big-O notation: the proposition $f(n) = O(g(n))$ means
    \[
    \exists C > 0 \; \exists N \; \forall n \ge N,\quad f(n) \le C \cdot g(n).
    \]

    Let $\texttt{MS}(n)$ be defined by the following pseudocode:
    \begin{lstlisting}
    MS(n):
        if n = 1: return 1
        else: return 2 * MS(floor(n/2)) + n
    \end{lstlisting}

    Consider the following ``proof'' claiming that $\texttt{MS}(n) = O(n)$.

    We prove the statement by induction on $n$. Let $n$ be an arbitrary positive integer, and
    assume that for every positive integer $m < n$, $\texttt{MS}(m) = O(m)$.

    \begin{itemize}
    \item If $n = 1$, then $\texttt{MS}(1) = 1 \le C \cdot 1$ for any $C \ge 1$.

    \item If $n > 1$, then
    \[
    \texttt{MS}(n) = 2 \cdot \texttt{MS}(\lfloor n/2 \rfloor) + n.
    \]
    By the induction hypothesis, there exist constants $C' > 0$ and $N'$ such that
    \[
    \texttt{MS}(\lfloor n/2 \rfloor) \le C' \cdot \lfloor n/2 \rfloor \quad \text{for all } n \ge N'.
    \]
    Thus,
    \[
    \texttt{MS}(n) \le 2 C' \cdot \lfloor n/2 \rfloor + n \le C' n + n = (C' + 1)n.
    \]
    Choosing $C \ge C' + 1$ gives $\texttt{MS}(n) \le C n$.
    \end{itemize}

    Therefore, $\texttt{MS}(n) = O(n)$.    
\end{problem}

\begin{problem}{2}
    \textbf{Balanced parentheses.}  

    A balanced parentheses string is a string over the symbols \texttt{[} and \texttt{]},
    defined recursively as one of the following:
    \begin{itemize}
        \item the empty string $\epsilon$;
        \item a string \texttt{[}$w$\texttt{]} for some balanced parentheses string $w$;
        \item a string $xy$ for some nonempty balanced parentheses strings $x$ and $y$.
    \end{itemize}

    For example, \texttt{[[[][]]][][[[]]][]} is a balanced parentheses string of length $18$.

    \begin{enumerate}
        \item Prove by induction that removing any pair of consecutive symbols \texttt{[]} (if one
        exists) from a balanced parentheses string results in another balanced parentheses string.

        \item[$\star$ (b)]
        Prove by induction that removing any pair of consecutive symbols \texttt{][} (if one exists)
        from a balanced parentheses string results in another balanced parentheses string.
        \end{enumerate}
\end{problem}

\begin{problem}{3}
\textbf{Fibonacci representations.}  
    The Fibonacci numbers are defined by
    \[
    F_n =
    \begin{cases}
    0 & \text{if } n = 0, \\
    1 & \text{if } n = 1, \\
    F_{n-1} + F_{n-2} & \text{otherwise}.
    \end{cases}
    \]

    Prove that every non-negative integer can be written as a sum of distinct,
    non-consecutive Fibonacci numbers using the following algorithm:

    \begin{lstlisting}
TOFIBO(n):
    if n = 0: return <>
    else:
        F_l <- largest Fibonacci number <= n
        <i_k, ..., i_1> <- TOFIBO(n - F_l)
        return <l, i_k, ..., i_1>
    \end{lstlisting}

    In other words, no two consecutive Fibonacci numbers $F_i$ and $F_{i+1}$ appear in the sum.
    (Assume $F_0 = 0$ never appears.)

    For example,
    \[
    30 = F_8 + F_6 + F_2.
    \]
\end{problem}


\begin{problem}{4$^\star$}
\textbf{Ludicrous tiling.}  

    Let $Q$ be the first quadrant of the real plane $\mathbb{R}^2$. You have an unlimited supply
    of $L$-shaped tiles, each consisting of a $2 \times 2$ square with one unit square removed.

    Prove that you can tile the entire quadrant $Q$ using only $L$-shapes and backward-$L$-shapes,
    with no rotations allowed.

%     \centering
%     \includegraphics[width=0.5\linewidth]{assests/2B26F7FC-37B3-4A0B-BD01-1A360959BC0D_4_5005_c.jpeg}
% \end
%     \caption{A partial tiling of the first quadrant using only $L$-shapes (pink) and backward-$L$-shapes (blue).}

\end{problem}

\begin{problem}{5$^\star$}
\textbf{Rooted binary tree.}  

    A rooted binary tree is defined recursively as either:
    \begin{itemize}
    \item the empty tree, denoted \texttt{null}; or
    \item a root node $T.x$, a left subtree $T_\ell$, and a right subtree $T_r$, where
    $T_\ell$ and $T_r$ are rooted binary trees.
    \end{itemize}

    In C, such a structure can be defined as:
    \begin{lstlisting}
struct binaryTree {
    int root;
    struct binaryTree *leftSubtree;
    struct binaryTree *rightSubtree;
};
    \end{lstlisting}

    Let $T$ be a rooted binary tree with $n$ vertices. Deleting any node $v$ splits $T$ into
    at most three subtrees: one containing the left child of $v$, one containing the right child
    of $v$, and one containing the parent of $v$ (if they exist).

    We call $v$ a \emph{central node} if each resulting subtree has at most $n/2$ nodes.
    Prove that every rooted binary tree has at least one central node.
\end{problem}



\end{document}