\documentclass[12pt]{article}

\usepackage{fullpage}
\usepackage{mdframed}
\usepackage{colonequals}
\usepackage{algpseudocode}
\usepackage{algorithm}
\usepackage[most, breakable]{tcolorbox}
\usepackage[all]{xy}
\usepackage{proof}
\usepackage{mathtools}
\usepackage{bbm}
\usepackage{amssymb}
\usepackage{amsthm}
\usepackage{amsmath}
\usepackage{amsxtra}
\usepackage{enumitem}
\newcommand{\bb}{\mathbb}


\newtheorem{theorem}{Theorem}[section]
\newtheorem{theorem*}{Theorem}
\newtheorem{definition}[theorem]{Definition}
\newtheorem{corollary}{Corollary}[theorem]
\newtheorem{lemma}[theorem]{Lemma}
\newtheorem{prop}[theorem]{Proposition}
\newtheorem{remark}[theorem]{Remark}


\newtheorem*{exercisehelper}{Exercise.}
\newenvironment{exercise}[1]{%
  \IfBlankTF{#1}
    {\renewcommand{\exercisehelper}{\textbf{Exercise} \unskip}}
    {\renewcommand\exercisehelper{\textbf{Exercise #1}}}%
  \exercisehelper
}{\endexercisehelper}

\theoremstyle{remark}
\newtheorem*{solution}{Solution}
\newcommand{\mathcat}[1]{\textup{\textbf{\textsf{#1}}}} % for defined terms

\newenvironment{problem}[1]
{ \begin{tcolorbox}[breakable]\noindent\textbf{Problem #1}.}
{\vskip 6pt \end{tcolorbox}}

\newenvironment{enumalph}
{\begin{enumerate}\renewcommand{\labelenumi}{\textnormal{(\alph{enumi})}}}
{\end{enumerate}}

\newenvironment{enumroman}
{\begin{enumerate}\renewcommand{\labelenumi}{\textnormal{(\roman{enumi})}}}
{\end{enumerate}}

\newcommand{\defi}[1]{\textsf{#1}} % for defined terms



\setlength{\hfuzz}{4pt}

\let\H\relax
\let\P\relax
\newcommand{\H}{\mathbb H}
\newcommand{\P}{\mathbb P}
\newcommand{\C}{\mathbb C}
\newcommand{\N}{\mathbb N}
\newcommand{\Q}{\mathbb Q}
\newcommand{\R}{\mathbb R}
\newcommand{\Z}{\mathbb Z}
\newcommand{\F}{\mathbb F}
\newcommand{\br}{\mathbf{r}}
\newcommand{\RP}{\mathbb{RP}}
\newcommand{\CP}{\mathbb{CP}}
\newcommand{\nbit}[1]{\{0, 1\}^{#1}}
\newcommand{\bits}{\{0, 1\}^{n}}
\newcommand{\bbni}{\bigbreak \noindent}
\newcommand{\norm}[1]{\left\vert\left\vert#1\right\vert\right\vert}
\newcommand{\dbar}{\overline{\partial}}
\let\d\relax
\newcommand{\d}{\partial}
\newcommand{\calO}{\mathcal{O}}
\newcommand{\calF}{\mathcal{F}}
\newcommand{\calG}{\mathcal{G}}
\newcommand{\calH}{\mathcal{H}}
\newcommand{\calE}{\mathcal{E}}
\newcommand{\calC}{\mathcal{C}}
\newcommand{\calD}{\mathcal{D}}

\let\1\relax
\newcommand{\1}{\mathbf{1}}
\newcommand{\fr}[2]{\left(\frac{#1}{#2}\right)}
\newcommand{\todo}[1]{\textcolor{red}{\textbf{TODO:} #1}}
\newcommand{\vecz}{\mathbf{z}}
\newcommand{\vecr}{\mathbf{r}}
\DeclareMathOperator{\Cinf}{C^{\infty}}
\DeclareMathOperator{\Id}{Id}
\DeclareMathOperator{\Ell}{Ell}
\DeclareMathOperator{\CL}{\mathcal{CL}}

\DeclareMathOperator{\Alt}{Alt}
\DeclareMathOperator{\Aut}{Aut}
\DeclareMathOperator{\ann}{ann}
\DeclareMathOperator{\codim}{codim}
\DeclareMathOperator{\End}{End}
\DeclareMathOperator{\Hom}{Hom}
\DeclareMathOperator{\id}{id}
\DeclareMathOperator{\M}{M}
\DeclareMathOperator{\Mat}{Mat}
\DeclareMathOperator{\Ob}{Ob}
\DeclareMathOperator{\opchar}{char}
\DeclareMathOperator{\opspan}{span}
\DeclareMathOperator{\rk}{rk}
\DeclareMathOperator{\sgn}{sgn}
\DeclareMathOperator{\Sym}{Sym}
\DeclareMathOperator{\tr}{tr}
\DeclareMathOperator{\img}{img}
\DeclareMathOperator{\coker}{coker}
\DeclareMathOperator{\Spec}{Spec}
\DeclareMathOperator{\CandE}{CandE}
\DeclareMathOperator{\CandO}{CandO}
\DeclareMathOperator{\argmax}{argmax}
\DeclareMathOperator{\first}{first}
\DeclareMathOperator{\last}{last}
\DeclareMathOperator{\cost}{cost}
\DeclareMathOperator{\dist}{dist}
\DeclareMathOperator{\path}{path}
\DeclareMathOperator{\parent}{parent}
\DeclareMathOperator{\argmin}{argmin}
\DeclareMathOperator{\excess}{excess}
\let\Pr\relax
\DeclareMathOperator{\Pr}{\mathbf{Pr}}
\DeclareMathOperator{\Exp}{\mathbb{E}}
\DeclareMathOperator{\Var}{\mathbf{Var}}
\let\limsup\relax
\DeclareMathOperator{\limsup}{limsup}
%Paired Delims
\DeclarePairedDelimiter\ceil{\lceil}{\rceil}
\let\oldceil\ceil
\renewcommand{\ceil}[1]{\oldceil*{#1}}

\DeclarePairedDelimiter{\floor}{\lfloor}{\rfloor}
\let\oldfloor\floor
\renewcommand{\floor}[1]{\oldfloor*{#1}}





\newcommand{\dagstar}{*}

\newcommand{\tbigwedge}{{\textstyle{\bigwedge}}}
\setlength{\parindent}{0pt}
\setlength{\parskip}{5pt}


\usepackage{listings}
\usepackage{courier}
\usepackage{microtype}


\lstset{
  basicstyle=\footnotesize\ttfamily,
  breaklines=true,
  breakatwhitespace=true
  columns=fullflexible,
  keepspaces=true,
  frame=single,
  escapeinside={(*@}{@*)}
}

\begin{document}

\title{Math 71: Abstract Algebra}

\author{Prishita Dharampal}
\date{}
\maketitle


\begin{problem}{1}
Let $A[1..n]$ be an array of $n$ distinct integers (which can be negative, positive, or zero),
sorted in increasing order; in other words,
\[
A[1] < A[2] < \cdots < A[n].
\]

The following algorithms, on the input $A[1..n]$, should either correctly compute an index
$i$ such that $A[i] = i$, or report that there is no such index.
\begin{minipage}[t]{0.48\textwidth}
\begin{lstlisting}
FINDMATCH1(A[lo .. hi]):
    if lo > hi return no such index exists
    m <- {(*@ $\ceil{\frac{lo + hi}{ 2 }}$ @*)}
    if A[m] = m:
        return m
    if A[m] < m:
        FINDMATCH1(A[lo .. m-1])
    if A[m] > m:
        FINDMATCH1(A[m+1 .. hi])
\end{lstlisting}
\end{minipage}
\hfill
\begin{minipage}[t]{0.48\textwidth}
\begin{lstlisting}
FINDMATCH2(A[lo .. hi]):
    if lo > hi return no such index exists
    m <- {(*@ $\ceil{\frac{lo + hi}{ 2 }}$ @*)}
    if A[m] = m:
        return m
    if A[m] > m:
        FINDMATCH2(A[lo .. m-1])
    if A[m] < m:
        FINDMATCH2(A[m+1 .. hi])
\end{lstlisting}

\end{minipage}

For each algorithm, decide whether it correctly performs the task.
For incorrect algorithms, construct an input where the algorithm fails.
For correct algorithms, prove correctness using induction.
\end{problem}

\begin{solution}
    \bbni
    \begin{enumerate}
        \item \texttt{FINDMATCH1} fails with the input \texttt{[-2,0,1,4,20]}. At $m=3$, we have $A[3]=1<3$, so the algorithm recurses on $A[1..2]$, but the correct solution $A[4]=4$ lies in the discarded right half.

        \item We prove the correctness of the proposition $P(A[1\ldots n])$ for every integer array of length at least $1$ satisfying the input conditions, by induction: 
        
        \textbf{Proposition.}


        \textbf{$P(A[lo \ldots hi]):$}  
        If $A[lo \ldots hi]$ is a strictly increasing array of integers, then  
        \texttt{FINDMATCH2}$(A[lo \ldots hi])$ correctly returns an index $m$ such that
        $A[m]=m$ if such an index exists, and otherwise correctly reports that no such
        index exists.


        \bbni
        
        Let $A$ be an arbitrary integer array of length $n \geq 1$ sorted in increasing order. Assume statement $P$ is true for every array $P(A[lo .. hi])$ shorter than $A$.
        \begin{enumerate}
            \item If $A[m] > m$, then for all $n > m$ we have
            \[
            A[n] \ge A[m] + (n - m) > n,
            \]
            so no index $n > m$ can satisfy $A[n] = n$. Thus any solution must lie in $A[lo \ldots m-1]$. \texttt{FINDMATCH2} recurses on $A[lo \ldots m-1]$, and by induction hypothesis returns a correctly output since $A[lo \ldots m-1]$ is smaller than $A[lo \ldots hi]$. 

            \item If $A[m] < m$, then for all $n < m$ we have 
            \[ A[n] \le A[m] - (m - n) < n,\]

            so no index $n < m$ can satisfy $A[n] = n$.  Thus any solution must lie in \\ 
            $A[m+1 \ldots hi]$.
            \texttt{FINDMATCH2} recurses on $A[m+1 \ldots hi]$, and by induction hypothesis  correctly returns an output since $A[m+1 \ldots hi]$ is smaller than $A[lo \ldots hi]$. 
            \item Otherwise if $A[m] = m$ \texttt{FINDMATCH2} correctly returns index $m$ (by definition). 
            
        \end{enumerate}

    \textbf{Base case:} If $lo > hi$, the array is empty and no index can satisfy $A[i] = i$. 
    
    \texttt{FINDMATCH2} correctly tells us that no such index exists.

    Hence, the base case also holds. And \texttt{FINDMATCH2} correctly solves the problem for all valid arrays, i.e. $P(A[lo \ldots hi])$ is true. 

    \end{enumerate}

\end{solution}

\newpage

\begin{problem}{2}
Let $A[1..n]$ be an array of $n$ distinct integers (which can be negative, positive, or zero).

The following algorithms, on the input $A[1..n]$, should sort the elements in array $A$
in increasing order; that is,
\[
A[1] < A[2] < \cdots < A[n].
\]
\begin{minipage}[t]{0.48\textwidth}
\begin{lstlisting}
AWKWARDSORT1(A[lo .. hi]):
    if hi <= lo + 1:(*@ \textcolor{red}{$\langle \langle$if n <= 2$\rangle \rangle$} @*)
        if A[lo] > A[hi]: 
            swap A[lo] and A[hi]
        return A
    else: (*@ \textcolor{red}{$\langle \langle$if n > 2$\rangle \rangle$} @*)
        AWKWARDSORT1(A[lo .. hi-1])
        AWKWARDSORT1(A[lo+1 .. hi])
        return A
\end{lstlisting}
\end{minipage}
\hfill
\begin{minipage}[t]{0.48\textwidth}
\begin{lstlisting}
AWKWARDSORT2(A[lo .. hi]):
    if A[lo] > A[hi]: 
        swap A[lo] and A[hi]
    if hi <= lo + 1:(*@ \textcolor{red}{$\langle \langle$if n <= 2$\rangle \rangle$} @*)
        return A
    else: (*@ \textcolor{red}{$\langle \langle$if n > 2$\rangle \rangle$} @*)
        AWKWARDSORT2(A[lo .. hi-1])
        AWKWARDSORT2(A[lo+1 .. hi])
        return A
\end{lstlisting}
\end{minipage}
For each algorithm, decide whether it correctly performs the task.
For incorrect algorithms, construct an input where the algorithm fails.
For correct algorithms, prove correctness using induction.
\end{problem}

\begin{solution}
    \bbni

    \begin{enumerate}
        \item \texttt{AWKWARDSORT1} breaks with the input [3, 2, 1].  
        \[\texttt{AWKWARDSORT1([3, 2, 1]) =  [2, 1, 3]},\]
        after the recursive calls, the element 1 never moves past 2, so the array remains unsorted. 

        \item We prove the correctness of the proposition $P(A[1\ldots n])$ for every integer array of length greater than $2$ satisfying the input conditions, by induction: 
        
        \textbf{Proposition.}

        \textbf{$P(A[lo \ldots hi]):$} If $A[lo \ldots hi]$ contains distinct integers, then \texttt{AWKWARDSORT2}($A [lo \ldots hi$]) returns an array sorted in increasing order. 

        \bbni        
        Assume the statement holds for all arrays $A[1\ldots k]$ of length $k$ where $ 3 \leq k < n$. Then, for $A[1\ldots n]$: 
        \begin{enumerate}
            \item If $A[1] > A[n]$: the first \texttt{if} block swaps  them. 
            This ensures that the smaller of $A[1]$ and $A[n]$ is placed at position $1$ and the larger at position $n$, which is required so that the ordering of the endpoints is compatible with the results of the recursive calls.


            \item The second if block fails (by assumption $n\geq 3$), and the else block recurses with \texttt{AWKWARDSORT2}$(A[1 \ldots n-1])$, and \texttt{AWKWARDSORT2}$(A[2 \ldots n])$
            
            Since both of these arrays have length less than $n$, by the induction hypothesis \texttt{AWKWARDSORT2} correctly sorts them. 

            So from the first recursive call we get, 
            \[ A[1] < A[2] < \cdots < A[n-1]\]

            and from the second recursive call, we get 
            \[ A[2] < A[3] < \cdots < A[n]\]
            
            Combining these inequalities with $A[1] < A[n]$, we conclude that 
            \[ A[1] < A[2] <  \cdots < A[n],\] so the entire array $A[1 \ldots n]$ is sorted. Therefore, \texttt{AWKWARDSORT2} correctly sorts arrays of length $n$.
        \end{enumerate}
    
    
    \textbf{Base case:} If $n = 2$, there are only two cases: 
    \begin{enumerate}
        \item $A[lo] < A[hi]$, the first \texttt{if} block is skipped and since $A$ has length $2$, it is already sorted. The array is returned in the second \texttt{if} block. 
        \item $A[lo] > A[hi]$, the first \texttt{if} block swaps the two integers, and returns the swapped array (in the second \texttt{if} block).
    \end{enumerate}
    Hence, the base case also holds. And \texttt{AWKWARDSORT2} correctly sorts all arrays, i.e. $P(A[lo \ldots hi])$ is true. 
    \end{enumerate}
\end{solution}





\end{document}