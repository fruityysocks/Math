\documentclass[12pt]{article}

\usepackage{fullpage}
\usepackage{mdframed}
\usepackage{colonequals}
\usepackage{algpseudocode}
\usepackage{algorithm}
\usepackage[most, breakable]{tcolorbox}
\usepackage[all]{xy}
\usepackage{proof}
\usepackage{mathtools}
\usepackage{bbm}
\usepackage{amssymb}
\usepackage{amsthm}
\usepackage{amsmath}
\usepackage{amsxtra}
\usepackage{enumitem}
\newcommand{\bb}{\mathbb}


\newtheorem{theorem}{Theorem}[section]
\newtheorem{theorem*}{Theorem}
\newtheorem{definition}[theorem]{Definition}
\newtheorem{corollary}{Corollary}[theorem]
\newtheorem{lemma}[theorem]{Lemma}
\newtheorem{prop}[theorem]{Proposition}
\newtheorem{remark}[theorem]{Remark}


\newtheorem*{exercisehelper}{Exercise.}
\newenvironment{exercise}[1]{%
  \IfBlankTF{#1}
    {\renewcommand{\exercisehelper}{\textbf{Exercise} \unskip}}
    {\renewcommand\exercisehelper{\textbf{Exercise #1}}}%
  \exercisehelper
}{\endexercisehelper}

\theoremstyle{remark}
\newtheorem*{solution}{Solution}
\newcommand{\mathcat}[1]{\textup{\textbf{\textsf{#1}}}} % for defined terms

\newenvironment{problem}[1]
{ \begin{tcolorbox}[breakable]\noindent\textbf{Problem #1}.}
{\vskip 6pt \end{tcolorbox}}

\newenvironment{enumalph}
{\begin{enumerate}\renewcommand{\labelenumi}{\textnormal{(\alph{enumi})}}}
{\end{enumerate}}

\newenvironment{enumroman}
{\begin{enumerate}\renewcommand{\labelenumi}{\textnormal{(\roman{enumi})}}}
{\end{enumerate}}

\newcommand{\defi}[1]{\textsf{#1}} % for defined terms



\setlength{\hfuzz}{4pt}

\let\H\relax
\let\P\relax
\newcommand{\H}{\mathbb H}
\newcommand{\P}{\mathbb P}
\newcommand{\C}{\mathbb C}
\newcommand{\N}{\mathbb N}
\newcommand{\Q}{\mathbb Q}
\newcommand{\R}{\mathbb R}
\newcommand{\Z}{\mathbb Z}
\newcommand{\F}{\mathbb F}
\newcommand{\br}{\mathbf{r}}
\newcommand{\RP}{\mathbb{RP}}
\newcommand{\CP}{\mathbb{CP}}
\newcommand{\nbit}[1]{\{0, 1\}^{#1}}
\newcommand{\bits}{\{0, 1\}^{n}}
\newcommand{\bbni}{\bigbreak \noindent}
\newcommand{\norm}[1]{\left\vert\left\vert#1\right\vert\right\vert}
\newcommand{\dbar}{\overline{\partial}}
\let\d\relax
\newcommand{\d}{\partial}
\newcommand{\calO}{\mathcal{O}}
\newcommand{\calF}{\mathcal{F}}
\newcommand{\calG}{\mathcal{G}}
\newcommand{\calH}{\mathcal{H}}
\newcommand{\calE}{\mathcal{E}}
\newcommand{\calC}{\mathcal{C}}
\newcommand{\calD}{\mathcal{D}}

\let\1\relax
\newcommand{\1}{\mathbf{1}}
\newcommand{\fr}[2]{\left(\frac{#1}{#2}\right)}
\newcommand{\todo}[1]{\textcolor{red}{\textbf{TODO:} #1}}
\newcommand{\vecz}{\mathbf{z}}
\newcommand{\vecr}{\mathbf{r}}
\DeclareMathOperator{\Cinf}{C^{\infty}}
\DeclareMathOperator{\Id}{Id}
\DeclareMathOperator{\Ell}{Ell}
\DeclareMathOperator{\CL}{\mathcal{CL}}

\DeclareMathOperator{\Alt}{Alt}
\DeclareMathOperator{\Aut}{Aut}
\DeclareMathOperator{\ann}{ann}
\DeclareMathOperator{\codim}{codim}
\DeclareMathOperator{\End}{End}
\DeclareMathOperator{\Hom}{Hom}
\DeclareMathOperator{\id}{id}
\DeclareMathOperator{\M}{M}
\DeclareMathOperator{\Mat}{Mat}
\DeclareMathOperator{\Ob}{Ob}
\DeclareMathOperator{\opchar}{char}
\DeclareMathOperator{\opspan}{span}
\DeclareMathOperator{\rk}{rk}
\DeclareMathOperator{\sgn}{sgn}
\DeclareMathOperator{\Sym}{Sym}
\DeclareMathOperator{\tr}{tr}
\DeclareMathOperator{\img}{img}
\DeclareMathOperator{\coker}{coker}
\DeclareMathOperator{\Spec}{Spec}
\DeclareMathOperator{\CandE}{CandE}
\DeclareMathOperator{\CandO}{CandO}
\DeclareMathOperator{\argmax}{argmax}
\DeclareMathOperator{\first}{first}
\DeclareMathOperator{\last}{last}
\DeclareMathOperator{\cost}{cost}
\DeclareMathOperator{\dist}{dist}
\DeclareMathOperator{\path}{path}
\DeclareMathOperator{\parent}{parent}
\DeclareMathOperator{\argmin}{argmin}
\DeclareMathOperator{\excess}{excess}
\let\Pr\relax
\DeclareMathOperator{\Pr}{\mathbf{Pr}}
\DeclareMathOperator{\Exp}{\mathbb{E}}
\DeclareMathOperator{\Var}{\mathbf{Var}}
\let\limsup\relax
\DeclareMathOperator{\limsup}{limsup}
%Paired Delims
\DeclarePairedDelimiter\ceil{\lceil}{\rceil}
\let\oldceil\ceil
\renewcommand{\ceil}[1]{\oldceil*{#1}}

\DeclarePairedDelimiter{\floor}{\lfloor}{\rfloor}
\let\oldfloor\floor
\renewcommand{\floor}[1]{\oldfloor*{#1}}





\newcommand{\dagstar}{*}

\newcommand{\tbigwedge}{{\textstyle{\bigwedge}}}
\setlength{\parindent}{0pt}
\setlength{\parskip}{5pt}


\usepackage{listings}
\usepackage{courier}
\usepackage{microtype}


\lstset{
  basicstyle=\footnotesize\ttfamily,
  breaklines=true,
  breakatwhitespace=true
  columns=fullflexible,
  keepspaces=true,
  frame=single,
  escapeinside={(*@}{@*)}
}

\begin{document}

\title{Math 71: Abstract Algebra}

\author{Prishita Dharampal}
\date{}
\maketitle


\begin{problem}{1}
    \textbf{Size and power sets.}

    A function $f$ from domain $A$ to codomain $B$ is called a bijection if there exists another function $f^{-1}$ from $B$ to $A$, such that $f^{-1}(f(a)) = a$ for every element $a \in A$, and $f(f^{-1}(b)) = b$ for every element $b \in B$. The size of a finite set is the number of elements inside (which is always a nonnegative integer).

    \begin{enumerate}
        \item State the Pigeonhole Principle as a quantified proposition, and state its contrapositive.

        \item Argue that if there is a bijection from a finite set $A$ to another finite set $B$, then $A$ and $B$ have the same size, using the contrapositive statement of the Pigeonhole Principle.

        \item The power set $\mathrm{Pow}(S)$ of a set $S$ is the set containing all subsets of $S$; in notation,
        \[
        A \in \mathrm{Pow}(S) \text{ if and only if } A \subseteq S.
        \]
        If a finite set $S$ has size $n$, what is the size of $\mathrm{Pow}(S)$? Justify your answer using (2).
    \end{enumerate}
\end{problem}

\begin{solution}
    \bbni 

    \begin{enumerate}
        \item \textbf{Pigeonhole Principle:} If there are more pigeons than the pigeonholes, then at least two pigeons stay in the same hole. 
        
        \textbf{Propositition}: For $x \, \text{pigeons}, y\, \text{pigeonholes}, z \, \text{pigeons in hole } y_0$
        \[x > y \implies \exists y_0, z \geq 2\]

        \textbf{Contrapositive:} For $x \, \text{pigeons}, y\, \text{pigeonholes}, z \, \text{pigeons in hole } y_0$
        \[\forall y_0, z < 2 \implies x \leq y\]

        \item Let $a \in A, b \in B$. If there exists a bijection from a finite set $A$ to a finite set $B$, we know that $\forall a \text{ (pigeons)}, f(a)$ corresponds to a unique (z = 1) $b \in B$ (pigeonholes), where $f$ is the bijective function. 
        
        \[\forall b, z < 1 \implies \mid A \mid \leq \mid B \mid\]

        Similarly, we also know that $\forall b \text{ (pigeons) }\in B, \exists a$ (pigeonholes) such that $f^{-1}(b) = a$, where $f^{-1}$ is the inverse of the function $f$. I.e. each $b$ corresponds to a unique ($z = 1$) $a$. 

        \[\forall a, z < 1 \implies \mid B \mid \leq \mid A \mid\]

        Since, $| B | \leq | A |$ and $| A | \leq | B |$, $| A |$ must be equal to $| B| $.  Hence, if there is a bijection from a finite set $A$ to another finite set $B$, then $A$ and $B$ have the same size. 

        \item Let $|S| = n$. Every subset $A \in Pow(S)$ can be determined by deciding for each element $s \in S$, whether
        \begin{enumerate}
            \item it is included (1), i.e. $s \in A$,
            \item it is excluded (0), i.e. $s \notin A$,
        \end{enumerate}
        I.e. every subset corresponds exactly to a unique string of $1$s and $0$s of size $n$. I.e. there exists a bijection between $Pow(S)$ and $\underbrace{\{0, 1\} \times \{0, 1\} \times, \cdots \times \{0, 1\}}_{n-times}$ 
        denoted by $\{0, 1\}^n$. 
        
        Let $f: Pow(S) \leftrightarrow \{0, 1\}^n$. 
        \[f(A) = (x_1, x_2, \cdots x_n), \quad x_i = \begin{cases}
            1 & s \in A \\ 
            0 & s \notin A 
        \end{cases} \]

        Since there are exactly $2^n$ elements in $\{0, 1\}$ and there exists a bijection between the two sets there must also be $2^n$ elements in $Pow(S)$. 
    \end{enumerate} 
\end{solution}

\newpage

\begin{problem} {2}
    \textbf{Algebra for quantifiers.}

    \begin{enumerate}
        \item Write down the English meaning of the following propositions. Do they have the same meaning? Why?
        \begin{itemize}
            \item $\forall$ even $n \in \mathbb{Z}\; \exists$ prime $p \in \mathbb{Z}\; \exists$ prime $q \in \mathbb{Z}$ such that $n = p + q$.
            \item $\exists$ prime $p \in \mathbb{Z}\; \forall$ even $n \in \mathbb{Z}\; \exists$ prime $q \in \mathbb{Z}$ such that $n = p + q$.
            \item $\forall$ even $n \in \mathbb{Z}\; \exists$ prime $q \in \mathbb{Z}\; \exists$ prime $p \in \mathbb{Z}$ such that $n = p + q$.
        \end{itemize}

        \item Does the following implication rule hold for arbitrary predicates $P(x)$ and $Q(x)$? Why or why not?
        \begin{itemize}
            \item $\forall x \, (P(x) \lor Q(x))$ implies $(\forall x \, P(x)) \lor (\forall x \, Q(x))$
            \item $\forall x \, (P(x) \land Q(x))$ implies $(\forall x \, P(x)) \land (\forall x \, Q(x))$
        \end{itemize}

        \item Is the following statement true or false? Why?
        \[
        \forall x \in \{0,1\}^* \; \exists y \in \{0,1\}^* \;
        \bigl( (y \text{ has fewer 0s than } x) \Rightarrow (x \text{ has at least two 0s}) \bigr)
        \]
        (Here $\{0,1\}^*$ is the set of all binary strings of finite length.)
    \end{enumerate}
\end{problem}

\begin{solution}
    \bbni 
    \begin{enumerate}
        \item \begin{itemize}
            \item For every even integer $n$, there exist two prime numbers $p, q$ such that $n = p + q$.
            \item There exists a (single) prime number $p$, such that for every even integer $n$ there exists another prime number $q$ such that the $n = p + q$. 
            \item For every even integer $n$, there exists two prime numbers $q, p$ such that $n = p+q$.
            
            \bbni 

            The first and the third proposition have the same meaning, because the order in which we define the primes does not matter. However the second proposition is different because it implies that there exists a specific prime number for all even integers such that that difference of every even integer and that specific prime number results in another prime number. The second statement is much stronger than the other two.  
        \end{itemize}

        \item \begin{itemize}
            \item Does not hold. 
            
            In $\forall x \, (P(x) \lor Q(x))$ means that for all values $x$ either $P(x)$ is true or $Q(x)$ is true. Whereas $(\forall x \, P(x)) \lor (\forall x \, Q(x))$ means that for all values $x$ either $P(x)$ is true, or for all values $x$ $Q(x)$ is true. For example, let \[P(a) = \text{true}, P(b) = \text{false}, Q(a) = \text{false}, Q(b) = \text{true}\]
            Then, $\forall x \, (P(x) \lor Q(x))$ is true. However, $(\forall x \, P(x)) \lor (\forall x \, Q(x))$ is false because both $(\forall x \, P(x)) $ and $(\forall x \, Q(x))$ are false. 
            
            The implication is that if everyone has one of two properties, then everyone must either have the first property or everyone must have the second property. 
            
            \bbni 

            \item Holds. 
            
            In $\forall x \, (P(x) \wedge Q(x))$ means that for all values $x$ both $P(x)$ is true and $Q(x)$ is true. And $(\forall x \, P(x)) \lor (\forall x \, Q(x))$ means that for all values $x$, $P(x)$ is true, and for all values $x$ $Q(x)$ is true. 
            The implication is that if everyone has both properties, then everyone having the first property must have the second property. 
        \end{itemize}

        \bbni

        \item The statement is true because for all values of $x$ we can pick a $y$ such that $y = x$ (i.e. the premise is false) making the implication vacously true. 
    \end{enumerate}
\end{solution}


\newpage

\begin{problem}{3}
    \textbf{Rachmaninoff had big hands.}

    Sergei Vasilyevich Rachmaninoff, the famous pianist, is known to have big hands. As a practice for himself, he chose 56 keys from a 108-key grand piano arbitrarily. Can he always find a pair of chosen keys among the 56 so that he can play an 11th (that is, having 10 keys in between)?

    Let $S$ be a set of 56 arbitrarily chosen numbers from $1$ to $108$. Prove the following statement: There are two numbers in $S$ that differ by exactly $11$.

    One student attempted the following proof:

    \emph{We will apply the pigeonhole principle. As we are choosing 56 numbers arbitrarily from 1 to 108, we treat the 56 chosen numbers as pigeons. We then divide the 108 numbers into eleven disjoint sets, each containing all numbers between 1 and 108 that share the same remainder when divided by 11}:
    \begin{align*}
            S_1 &= \{1,12,\ldots,100\}, \\
            S_2 &= \{2,13,\ldots,101\}, \\ 
            \ldots \\
            S_{11} &= \{11,22,\ldots,99\}
    \end{align*}


    The sets $S_i$ are all either of size 9 or size 10 and therefore we can choose 5 elements from each without picking 2 numbers that differ by 11. Therefore we have 55 available numbers to choose from because $5 \times 11 = 55$. These 55 numbers represent the pigeonholes. By the pigeonhole principle, there must be at least two pigeons in one hole, and in this case that is not possible as we cannot choose the same number twice, so we must choose a different number, and doing so would create a pair that differs by exactly 11.

    \begin{enumerate}
        \item Point out the most significant issue in the proof, and justify your criticism.

        \item Prove the statement in a way that is clear, correct, convincing, and concise.
    \end{enumerate}
\end{problem}

\begin{solution}
    \bbni
    \begin{enumerate}
        \item The most significant issue in the proof is the the argument doesn't work for an arbitrary case. The author can pick at most 5 elements from each $S_i$, so the number of pigeonholes would change. Also, the assignment of pigeons into pigeonholes isn't specified, so we don't know if there even exists a mapping between the arbitrary numbers Rachmaninoff picks and the numbers the author picked. 
        \item Let $A = \{1, 2, \cdots, 108\}$. We can then divide the $108$ numbers into eleven disjoint sets, each containing all numbers between 1 and 108 that share the same remainder when divided by 11:
        \begin{align*}
            S_1 &= \{1,12,\ldots,100\}, \\
            S_2 &= \{2,13,\ldots,101\}, \\ 
            \vdots \\
            S_{11} &= \{11,22,\ldots,99\}
        \end{align*}
        Let $S_{PH} = \{S_1, S_2, \cdots, S_{11}\}$ be the set containing the sets that share the same remainder when divided by $11$. 
        These are the pigeonholes ($n=11$). The 56 numbers Rachmaninoff picks in $S$ are the pigeons. For every $s \in S$ assign it to it's corresponding set in $S_{PH}$, i.e.,
        \[f: S \to S_{PH} \]

        By the c-fold ($c =5$) pigeonhole principle we know that if there are $n$ pigeonholes but more than $c \cdot n$ ($n=11$) pigeons, then at least one pigeonhole contains $c + 1$ pigeons. Hence, that there would at least be $6$ $(c + 1)$ integers in $S$ that share the same remainder modulo $11$. Let $S_i$ be the disjoint set that contains these $6$ elements. This set has at most $10$ elements. Let's break this into $5$ pairs of adjacent elements. Since each key is in exactly one of these pairs, by the pigeonhole principle there is a pair that has with two keys. That is, there are two numbers in $S$ that differ by $11$.
    \end{enumerate}
\end{solution}

\newpage

\begin{problem}{*4}
    \textbf{Moving robots.}

    You are leading a team to write a program that lets the mini-robots perform synchronized walking: There are $k^2$ mini-robots arranged in a $k$-by-$k$ array, surrounded by a wall. At each step, every mini-robot moves to an adjacent spot all at once, while keeping the $k$-by-$k$ formation; that is, each robot moves either up, down, left, or right by one step, and no robots adjacent to the wall are allowed to move towards the wall. Also, no two robots switch places between themselves as they will bump into each other.

    The team tested the program on 16 robots in a $4$-by-$4$ array first, and the robots walked smoothly into a new $4$-by-$4$ formation, exactly as expected. But when they ran the program on a larger scale, with 49 robots and $k = 7$, some robots kept bumping into each other. Your teammates come to you for diagnosis. What should you tell them?
\end{problem}

\end{document}