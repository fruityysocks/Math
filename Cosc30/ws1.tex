\documentclass[12pt]{article}

\usepackage{fullpage}
\usepackage{mdframed}
\usepackage{colonequals}
\usepackage{algpseudocode}
\usepackage{algorithm}
\usepackage[most, breakable]{tcolorbox}
\usepackage[all]{xy}
\usepackage{proof}
\usepackage{mathtools}
\usepackage{bbm}
\usepackage{amssymb}
\usepackage{amsthm}
\usepackage{amsmath}
\usepackage{amsxtra}
\usepackage{enumitem}
\newcommand{\bb}{\mathbb}


\newtheorem{theorem}{Theorem}[section]
\newtheorem{theorem*}{Theorem}
\newtheorem{definition}[theorem]{Definition}
\newtheorem{corollary}{Corollary}[theorem]
\newtheorem{lemma}[theorem]{Lemma}
\newtheorem{prop}[theorem]{Proposition}
\newtheorem{remark}[theorem]{Remark}


\newtheorem*{exercisehelper}{Exercise.}
\newenvironment{exercise}[1]{%
  \IfBlankTF{#1}
    {\renewcommand{\exercisehelper}{\textbf{Exercise} \unskip}}
    {\renewcommand\exercisehelper{\textbf{Exercise #1}}}%
  \exercisehelper
}{\endexercisehelper}

\theoremstyle{remark}
\newtheorem*{solution}{Solution}
\newcommand{\mathcat}[1]{\textup{\textbf{\textsf{#1}}}} % for defined terms

\newenvironment{problem}[1]
{ \begin{tcolorbox}[breakable]\noindent\textbf{Problem #1}.}
{\vskip 6pt \end{tcolorbox}}

\newenvironment{enumalph}
{\begin{enumerate}\renewcommand{\labelenumi}{\textnormal{(\alph{enumi})}}}
{\end{enumerate}}

\newenvironment{enumroman}
{\begin{enumerate}\renewcommand{\labelenumi}{\textnormal{(\roman{enumi})}}}
{\end{enumerate}}

\newcommand{\defi}[1]{\textsf{#1}} % for defined terms



\setlength{\hfuzz}{4pt}

\let\H\relax
\let\P\relax
\newcommand{\H}{\mathbb H}
\newcommand{\P}{\mathbb P}
\newcommand{\C}{\mathbb C}
\newcommand{\N}{\mathbb N}
\newcommand{\Q}{\mathbb Q}
\newcommand{\R}{\mathbb R}
\newcommand{\Z}{\mathbb Z}
\newcommand{\F}{\mathbb F}
\newcommand{\br}{\mathbf{r}}
\newcommand{\RP}{\mathbb{RP}}
\newcommand{\CP}{\mathbb{CP}}
\newcommand{\nbit}[1]{\{0, 1\}^{#1}}
\newcommand{\bits}{\{0, 1\}^{n}}
\newcommand{\bbni}{\bigbreak \noindent}
\newcommand{\norm}[1]{\left\vert\left\vert#1\right\vert\right\vert}
\newcommand{\dbar}{\overline{\partial}}
\let\d\relax
\newcommand{\d}{\partial}
\newcommand{\calO}{\mathcal{O}}
\newcommand{\calF}{\mathcal{F}}
\newcommand{\calG}{\mathcal{G}}
\newcommand{\calH}{\mathcal{H}}
\newcommand{\calE}{\mathcal{E}}
\newcommand{\calC}{\mathcal{C}}
\newcommand{\calD}{\mathcal{D}}

\let\1\relax
\newcommand{\1}{\mathbf{1}}
\newcommand{\fr}[2]{\left(\frac{#1}{#2}\right)}
\newcommand{\todo}[1]{\textcolor{red}{\textbf{TODO:} #1}}
\newcommand{\vecz}{\mathbf{z}}
\newcommand{\vecr}{\mathbf{r}}
\DeclareMathOperator{\Cinf}{C^{\infty}}
\DeclareMathOperator{\Id}{Id}
\DeclareMathOperator{\Ell}{Ell}
\DeclareMathOperator{\CL}{\mathcal{CL}}

\DeclareMathOperator{\Alt}{Alt}
\DeclareMathOperator{\Aut}{Aut}
\DeclareMathOperator{\ann}{ann}
\DeclareMathOperator{\codim}{codim}
\DeclareMathOperator{\End}{End}
\DeclareMathOperator{\Hom}{Hom}
\DeclareMathOperator{\id}{id}
\DeclareMathOperator{\M}{M}
\DeclareMathOperator{\Mat}{Mat}
\DeclareMathOperator{\Ob}{Ob}
\DeclareMathOperator{\opchar}{char}
\DeclareMathOperator{\opspan}{span}
\DeclareMathOperator{\rk}{rk}
\DeclareMathOperator{\sgn}{sgn}
\DeclareMathOperator{\Sym}{Sym}
\DeclareMathOperator{\tr}{tr}
\DeclareMathOperator{\img}{img}
\DeclareMathOperator{\coker}{coker}
\DeclareMathOperator{\Spec}{Spec}
\DeclareMathOperator{\CandE}{CandE}
\DeclareMathOperator{\CandO}{CandO}
\DeclareMathOperator{\argmax}{argmax}
\DeclareMathOperator{\first}{first}
\DeclareMathOperator{\last}{last}
\DeclareMathOperator{\cost}{cost}
\DeclareMathOperator{\dist}{dist}
\DeclareMathOperator{\path}{path}
\DeclareMathOperator{\parent}{parent}
\DeclareMathOperator{\argmin}{argmin}
\DeclareMathOperator{\excess}{excess}
\let\Pr\relax
\DeclareMathOperator{\Pr}{\mathbf{Pr}}
\DeclareMathOperator{\Exp}{\mathbb{E}}
\DeclareMathOperator{\Var}{\mathbf{Var}}
\let\limsup\relax
\DeclareMathOperator{\limsup}{limsup}
%Paired Delims
\DeclarePairedDelimiter\ceil{\lceil}{\rceil}
\let\oldceil\ceil
\renewcommand{\ceil}[1]{\oldceil*{#1}}

\DeclarePairedDelimiter{\floor}{\lfloor}{\rfloor}
\let\oldfloor\floor
\renewcommand{\floor}[1]{\oldfloor*{#1}}





\newcommand{\dagstar}{*}

\newcommand{\tbigwedge}{{\textstyle{\bigwedge}}}
\setlength{\parindent}{0pt}
\setlength{\parskip}{5pt}


\usepackage{listings}
\usepackage{courier}
\usepackage{microtype}


\lstset{
  basicstyle=\footnotesize\ttfamily,
  breaklines=true,
  breakatwhitespace=true
  columns=fullflexible,
  keepspaces=true,
  frame=single,
  escapeinside={(*@}{@*)}
}

\begin{document}

\title{Math 71: Abstract Algebra}

\author{Prishita Dharampal}
\date{}
\maketitle


\begin{problem}{1}
     Prove that at least seven people in this room were born on the same day of the week.
    (How many people do we need for this statement to be true?)
\end{problem}

\begin{solution}
    \bbni
    
    There are $7$ days in a week. Let $H$ be the set of days in a week (pigeonholes), and $P$ the set of all people in the room (pigeons) with $\mid P \mid = x$, and $f$ be the function that maps every person to the day they were born.  

    Now, since we have $x$ pigeons and $7$ pigeonholes and we need to have at least $7$ pigeons in $1$ pigeonhole. We can find the number of pigeons using the c-fold pigeonhole equation: 
    \begin{align*}
        \text{number of pigeons} & = (\text{number of pigeons in 1 hole} -1) \cdot (\text{ number of pigeon holes})+ 1 \\ 
        x &= (7-1) \cdot 7 + 1 \\ 
        x & = 6 \cdot 7  + 1 \\ 
        x &= 43
    \end{align*}
    I.e  we need $6 \times 7 + 1 = 43$ people in the room for at least seven of them to have been born on the same day. 
\end{solution}

\newpage 

\begin{problem}{2}
    Pick $11$ distinct integers of your liking into a set $S$. Prove that you can always find three integers from $S$, each differ from the other two by some multiple of $5$s; in other words, the three integers share the same remainder modulo $5$. For example, if we choose
    \[S := \{31, 41, 59, 26, 53, 58, 97, 93, 23, 84, 62\},\] 
    then the three numbers $31, 41, 26$ share the same remainder modulo $5$. 
\end{problem}

\begin{solution}
    \bbni

    For any $x \in S$, consider it's remainder mod $5$. There are exactly $5$ such possibilities. Let this be the set $H$. 
    \[H = \{0, 1, 2, 3, 4\}\]
    Let the elements of set $S$ be the pigeons, the elements of set $H$ be the pigeonholes. Now, since we have $11$ pigeons and $5$ pigeonholes, we can find the c-fold number using the equation: 
    \begin{align*}
        \text{number of pigeons} & = c \cdot \text{ number of pigeon holes} + 1 \\ 
        11 &= c \cdot 5 + 1 \\ 
        c & = 2
    \end{align*}
    Define a function $f:S \to H$ that maps every integer to it's remainder modulo $5$. By the c-fold pigeonhole principle we know that if there are $n$ pigeonholes but more than $c \cdot n$ pigeons, then at least one pigeonhole contains $c + 1$ pigeons. Hence, that there would at least be $3$ $(c + 1)$ integers in $S$ that share the same remainder modulo $5$. 
\end{solution}

\begin{problem}{3}
    Pick $16$ distinct integers in $[1 ..30]$. Prove that there is at least a pair that adds up to $31$. (Notation $[1 \ldots 30]$ is a shorthand for the set $\{n \in N : 1 \leq n \leq 30\}$.)
\end{problem}

\begin{solution}
    \bbni
    
    Integers in the interval $[1\ldots 30]$ can be partitioned into exactly $15$ disjoint pairs that add up to $31$. Define the set of all such pairs 
    \[S = \{(1, 30), (2, 29), \cdots, (15, 16)\}\]
    
    Let $P$ be the set of the 16 distinct integers. The elements of $S$ are the pigeonholes, and the elements of $P$ are the pigeons. Since integer in $P$ belongs to exactly one of these pairs (they're disjoint). Thus by assigning each integer to the pair it belongs in puts $16$ pigeons in $15$ holes. And by the (1-fold) pigeon hole principle, at least one pigeonhole (pair) must have two pigeons (integers in the set $P$). I.e. there is at least one pair in $P$ that adds up to $31$. 
\end{solution}

\end{document}
