\documentclass[12pt]{article}

\usepackage{fullpage}
\usepackage{mdframed}
\usepackage{colonequals}
\usepackage{algpseudocode}
\usepackage{algorithm}
\usepackage[most, breakable]{tcolorbox}
\usepackage[all]{xy}
\usepackage{proof}
\usepackage{mathtools}
\usepackage{bbm}
\usepackage{amssymb}
\usepackage{amsthm}
\usepackage{amsmath}
\usepackage{amsxtra}
\usepackage{enumitem}
\newcommand{\bb}{\mathbb}


\newtheorem{theorem}{Theorem}[section]
\newtheorem{theorem*}{Theorem}
\newtheorem{definition}[theorem]{Definition}
\newtheorem{corollary}{Corollary}[theorem]
\newtheorem{lemma}[theorem]{Lemma}
\newtheorem{prop}[theorem]{Proposition}
\newtheorem{remark}[theorem]{Remark}


\newtheorem*{exercisehelper}{Exercise.}
\newenvironment{exercise}[1]{%
  \IfBlankTF{#1}
    {\renewcommand{\exercisehelper}{\textbf{Exercise} \unskip}}
    {\renewcommand\exercisehelper{\textbf{Exercise #1}}}%
  \exercisehelper
}{\endexercisehelper}

\theoremstyle{remark}
\newtheorem*{solution}{Solution}
\newcommand{\mathcat}[1]{\textup{\textbf{\textsf{#1}}}} % for defined terms

\newenvironment{problem}[1]
{ \begin{tcolorbox}[breakable]\noindent\textbf{Problem #1}.}
{\vskip 6pt \end{tcolorbox}}

\newenvironment{enumalph}
{\begin{enumerate}\renewcommand{\labelenumi}{\textnormal{(\alph{enumi})}}}
{\end{enumerate}}

\newenvironment{enumroman}
{\begin{enumerate}\renewcommand{\labelenumi}{\textnormal{(\roman{enumi})}}}
{\end{enumerate}}

\newcommand{\defi}[1]{\textsf{#1}} % for defined terms



\setlength{\hfuzz}{4pt}

\let\H\relax
\let\P\relax
\newcommand{\H}{\mathbb H}
\newcommand{\P}{\mathbb P}
\newcommand{\C}{\mathbb C}
\newcommand{\N}{\mathbb N}
\newcommand{\Q}{\mathbb Q}
\newcommand{\R}{\mathbb R}
\newcommand{\Z}{\mathbb Z}
\newcommand{\F}{\mathbb F}
\newcommand{\br}{\mathbf{r}}
\newcommand{\RP}{\mathbb{RP}}
\newcommand{\CP}{\mathbb{CP}}
\newcommand{\nbit}[1]{\{0, 1\}^{#1}}
\newcommand{\bits}{\{0, 1\}^{n}}
\newcommand{\bbni}{\bigbreak \noindent}
\newcommand{\norm}[1]{\left\vert\left\vert#1\right\vert\right\vert}
\newcommand{\dbar}{\overline{\partial}}
\let\d\relax
\newcommand{\d}{\partial}
\newcommand{\calO}{\mathcal{O}}
\newcommand{\calF}{\mathcal{F}}
\newcommand{\calG}{\mathcal{G}}
\newcommand{\calH}{\mathcal{H}}
\newcommand{\calE}{\mathcal{E}}
\newcommand{\calC}{\mathcal{C}}
\newcommand{\calD}{\mathcal{D}}

\let\1\relax
\newcommand{\1}{\mathbf{1}}
\newcommand{\fr}[2]{\left(\frac{#1}{#2}\right)}
\newcommand{\todo}[1]{\textcolor{red}{\textbf{TODO:} #1}}
\newcommand{\vecz}{\mathbf{z}}
\newcommand{\vecr}{\mathbf{r}}
\DeclareMathOperator{\Cinf}{C^{\infty}}
\DeclareMathOperator{\Id}{Id}
\DeclareMathOperator{\Ell}{Ell}
\DeclareMathOperator{\CL}{\mathcal{CL}}

\DeclareMathOperator{\Alt}{Alt}
\DeclareMathOperator{\Aut}{Aut}
\DeclareMathOperator{\ann}{ann}
\DeclareMathOperator{\codim}{codim}
\DeclareMathOperator{\End}{End}
\DeclareMathOperator{\Hom}{Hom}
\DeclareMathOperator{\id}{id}
\DeclareMathOperator{\M}{M}
\DeclareMathOperator{\Mat}{Mat}
\DeclareMathOperator{\Ob}{Ob}
\DeclareMathOperator{\opchar}{char}
\DeclareMathOperator{\opspan}{span}
\DeclareMathOperator{\rk}{rk}
\DeclareMathOperator{\sgn}{sgn}
\DeclareMathOperator{\Sym}{Sym}
\DeclareMathOperator{\tr}{tr}
\DeclareMathOperator{\img}{img}
\DeclareMathOperator{\coker}{coker}
\DeclareMathOperator{\Spec}{Spec}
\DeclareMathOperator{\CandE}{CandE}
\DeclareMathOperator{\CandO}{CandO}
\DeclareMathOperator{\argmax}{argmax}
\DeclareMathOperator{\first}{first}
\DeclareMathOperator{\last}{last}
\DeclareMathOperator{\cost}{cost}
\DeclareMathOperator{\dist}{dist}
\DeclareMathOperator{\path}{path}
\DeclareMathOperator{\parent}{parent}
\DeclareMathOperator{\argmin}{argmin}
\DeclareMathOperator{\excess}{excess}
\let\Pr\relax
\DeclareMathOperator{\Pr}{\mathbf{Pr}}
\DeclareMathOperator{\Exp}{\mathbb{E}}
\DeclareMathOperator{\Var}{\mathbf{Var}}
\let\limsup\relax
\DeclareMathOperator{\limsup}{limsup}
%Paired Delims
\DeclarePairedDelimiter\ceil{\lceil}{\rceil}
\let\oldceil\ceil
\renewcommand{\ceil}[1]{\oldceil*{#1}}

\DeclarePairedDelimiter{\floor}{\lfloor}{\rfloor}
\let\oldfloor\floor
\renewcommand{\floor}[1]{\oldfloor*{#1}}





\newcommand{\dagstar}{*}

\newcommand{\tbigwedge}{{\textstyle{\bigwedge}}}
\setlength{\parindent}{0pt}
\setlength{\parskip}{5pt}


\usepackage{listings}
\usepackage{courier}
\usepackage{microtype}


\lstset{
  basicstyle=\footnotesize\ttfamily,
  breaklines=true,
  breakatwhitespace=true
  columns=fullflexible,
  keepspaces=true,
  frame=single,
  escapeinside={(*@}{@*)}
}
\usepackage{euler}

\begin{document}

\title{Math 71: Abstract Algebra}

\author{Prishita Dharampal}
\date{}
\maketitle


\begin{problem}{1}
    Verify that the following are metric spaces:
    \begin{enumerate}
        \item all n-tuples of real numbers, with 
        \[d((x_1, \cdots, x_n)(y_1, \cdots, y_n)) = \sum_{i=1}^{n}|x_i - y_i|\]
        \item all bounded infinite sequences $x = (x_1, x_2, x_3, \cdots)$ of elements of $\R$ with \[d(x,y) = l.u.b. \, \{|x_1 - y_1|, |x_2 - y_2|, |x_3 - y_3|, \cdots \}\]
    \end{enumerate}
\end{problem}

\begin{solution}
    \bbni
    \begin{enumerate}
        \item Let $E = \{\text{all n-tuples of real numbers}\}$, with the distance between two points $x$ and $y$ $d(x, y) = d((x_1, \cdots, x_n)(y_1, \cdots, y_n)) = \sum_{i=1}^{n}|x_i - y_i|$. To see if $(E,d)$ defines a metric space, we check the following: 
        \begin{enumerate}
            \item $d((x_1, \cdots, x_n)(y_1, \cdots, y_n)) \geq 0$
            
            Since, $d((x_1, \cdots, x_n)(y_1, \cdots, y_n)) = \sum_{i=1}^{n}|x_i - y_i|$, and each term $|x_i - y_i| \geq 0$, the sum of these terms is also non-negative. Hence, we can say that $\forall x, y \in E$,
             $d((x_1, \cdots, x_n)(y_1, \cdots, y_n)) \geq 0$. 

            \item $d((x_1, \cdots, x_n)(y_1, \cdots, y_n)) = 0 \iff  (x_1, \cdots, x_n) = (y_1, \cdots, y_n)$ 
            
            $(\implies)$

            Assume $d((x_1, \cdots, x_n)(y_1, \cdots, y_n)) = 0$. Since every term $|x_i - y_i|$ is non-negative, the sum being zero implies that every term is also zero. I.e. $|x_i - y_i| = 0 \implies x_i = y_i$, for all  $x_i, y_i$. Hence, $(x_1, \cdots, x_n) = (y_1, \cdots, y_n)$. 

            \bbni
            
            $(\impliedby)$ 

            Assume $(x_1, \cdots, x_n) = (y_1, \cdots, y_n)$. Then,
             
            $d((x_1, \cdots, x_n), (y_1, \cdots, y_n)) = \sum_{i=1}^n |x_i - y_i| = \sum_{i=1}^n |x_i - x_i| = \sum_{i=1}^n 0 = 0$

            \item  $d(x,y) = d(y,x), \, \forall x, y \in E$ 
            
            For every term $|x_i - y_i|$ in the summation, we know that by definition, $|x_i - y_i| = |y_i - x_i|$. Hence, $d((x_1, \cdots, x_n)(y_1, \cdots, y_n))  = d((y_1, \cdots, y_n), (x_1, \cdots, x_n))$ or $d(x,y) = d(y,x)$. 

            \item Triangle Inequality: $d(x,z) \leq d(x,y) + d(y,z)$
            
            \begin{align*}
                d(x,y) + d(y,z) &= d((x_1, \cdots, x_n), (y_1, \cdots, y_n)) + d((y_1, \cdots, y_n), (z_1, \cdots, z_n)) \\ 
                &= \sum_{i=1}^n |x_i - y_i| + \sum_{i=1}^n |y_i - z_i| \\ 
                &= \sum_{i=1}^{n} |x_i - y_i| + |y_i - z_i| \\ 
        \text{ We know that, } \\ 
                & |x_i - y_i| + |y_i - z_i| \geq  | x_i - y_i + y_i - z_i | \\ 
                \implies &|x_i - y_i| + |y_i - z_i| \geq |x_i - z_i|\\ 
            \end{align*}
            I.e., 
            \begin{align*}
                    \sum_{i=1}^{n} |x_i - y_i| + |y_i - z_i| &\geq \sum_{i=1}^{n} |x_i - z_i| \\ 
                    d((x_1, \cdots, x_n), (y_1, \cdots, y_n)) + d((y_1, \cdots, y_n), (z_1, \cdots, z_n)) &\geq d((x_1, \cdots, x_n), (z_1, \cdots, z_n)) \\ 
                    d(x,y) + d(y,z) &\geq d(x,z)
            \end{align*}
        \end{enumerate}
        Hence, $(E,d)$ defines a metric space. 

        \item Let $E = \{\text{all bounded infinite sequences of real numbers}\}$ with the distance function $d$ defined as $d(x,y) = l.u.b. \, \{|x_1 - y_1|, |x_2 - y_2|, \cdots\} = \{|x_i - y_i\}_{i \in \N}$. To see if $(E,d)$ defines a metric space, we check the following: 
        \begin{enumerate}
            \item $d(x,y) \geq 0$ 
            
            \[d(x,y) = l.u.b. \, \{|x_i - y_i|\}_{i \in \N}\]
            
            with each element in the set $|x_i - y_i| \geq 0, \, \forall i \in \N$ and  $l.u.b. \{|x_i - y_i|\}_{i \in \N} \geq |x_j - y_j|$ for any $j \in \N$. On combining the inequalities we get, 
            \begin{align*}
                l.u.b. \{|x_i - y_i|\}_{i \in \N} &\geq |x_j - y_j| 
                \geq 0 \\ 
                l.u.b. \{|x_i - y_i|\}_{i \in \N} &\geq 0 
            \end{align*}

            \item $d(x,y) = 0 \iff x = y$
            
            $(\implies)$

            Assume $d(x,y) = 0$. Then $l.u.b. \{|x_i - y_i|\}_{i \in \N} = 0 \implies |x_i - y_i| \leq 0, \, \forall {i \in \N}$. But by definition, the absolute value is positive, i.e. $|x_i - y_i| \geq 0, \, \forall {i \in \N}$. On combining the inequalities, we get: 
            $|x_i - y_i| = 0, \, \forall {i \in \N}$.
            \\ 
            I.e. $\forall i \in \N, x_i = y_i \implies x = y$.
            
            \bbni 

            $(\impliedby)$

            Assume $x =y$. 
            
            Then $\forall i \in \N, x_i = y_i \implies x_i - y_i = 0 \implies l.u.b. \{|x_i - y_i|\}_{i \in \N} = 0 = d(x,y)$. 

            \item $d(x,y) = d(y,x)$ 
            
            By definition,  $|x_i - y_i| = |y_i - x_i|$, so $l.u.b. \{|x_i - y_i|\}_{i \in \N} = l.u.b. \{|y_i - x_i|\}_{i \in \N} $.  Hence, $d(x,y) = d(y,x)$. 

            \item $d(x, z) \leq d(x,y) + d(y,z)$ 
            
            Since 
            \[|x_i - z_i| \le |x_i - y_i |+ |y_i - z_i| \, \forall i \in \N,\]
             
            we can see that because 
            \[l.u.b. \{ |x_i - y_i| + | y_i - z_i|\}_{i\in \N}\] 
            is an upper bound on $\{ |x_i - y_i| + | y_i - z_i|\}_{i\in \N}$ it is also an upper bound on $\{ |x_i - z_i|\}_{i\in \N}$.  I.e. 
            \[\{ |x_i - z_i|\}_{i\in \N} \leq \{ |x_i - y_i| + | y_i - z_i|\}_{i\in \N} \leq l.u.b. \{ |x_i - y_i| + | y_i - z_i|\}_{i\in \N}\] 
  
            And (from pset $1$, problem $9$) we know for non-empty, bounded subsets of $\R$ that,  $l.u.b. \, \{x+y: x\in S_1, y \in S_2 \} = l.u.b. \, S_1 + l.u.b. \, S_2$. I.e., 

            \[\implies l.u.b. \{|x_i - y_i| + |y_i - z_i|\}_{i \in \N}  = l.u.b. \{|x_i - y_i| \} + l.u.b. \{ |y_i - z_i|\}_{i \in \N}\]
            
            Combining and substituting gives us, 

            \[\implies l.u.b. \{|x_i - z_i|\}_{i \in \N}  \leq l.u.b. \{|x_i - y_i| \} + l.u.b. \{ |y_i - z_i|\}_{i \in \N}\]

        \end{enumerate}
        
        Hence the triangle inequality holds, and $(E,d)$ defines a metric space.
    \end{enumerate}
\end{solution}

\newpage 

\begin{problem}{2}
    What are the open and closed balls in the metric space of example (4), § 1? Show that two balls of different centers and radii may be equal. What are the open sets in this metric space?
    \\ 

    \textbf{Example (4),§ 1: Let $E$ be an arbitrary set and, for $p, q  \in E$, define $d(p, q ) = 0$ if $p= q, d(p, q) = 1$ if $p \neq q$.}
\end{problem}

\begin{solution}
    \bbni
    \begin{enumerate}
        \item  \begin{enumerate}
        \item $0 < r < 1$
        \begin{enumerate}
            \item Open Balls:
            \[B_r(p) = \{d(p, q) < r; q \in E\} = \{p\}\]
            Since $d(p,q) < 1$ implies $p = q$. 

            \item Closed Balls: 
            \[B_r(p) = \{d(p, q) \leq r; q \in E\} = \{p\}\]
            Since by definition $d(p,q) < 1$ implies $p = q$. 
        \end{enumerate}
        \item $r = 1$
        \begin{enumerate}
            \item Open Balls:
            \[B_r(p) = \{d(p, q) < r; q \in E\} = \{p\}\]
            Since $d(p,q) < 1$ implies $p = q$. 

            \item Closed Balls: 
            \[B_r(p) = \{d(p, q) \leq r; q \in E\} = E\]

            Since $d(p,q)\le 1$ for all $q\in E$.
        \end{enumerate}
        \item $r > 1$
        \begin{enumerate}
            \item Open Balls:
            \[B_r(p) = \{d(p, q) \leq r; q \in E\} = E\]

            Since $d(p,q)\le 1$ for all $q\in E$.

            \item Closed Balls: 
            \[B_r(p) = \{d(p, q) \leq r; q \in E\} = E\]

            Since $d(p,q)\le 1$ for all $q\in E$.
        \end{enumerate}    
        \end{enumerate}

            \bbni 

            \item To see that two balls with different centers and radii may be equal, consider two open balls $B_{r_{1}}(p), B_{r_2}(q)$ with $p \neq q$, $r_1 \neq r_2$, and $r_1, r_2 > 0$. Then, as we saw above, 
            \[B_{r_1}(p) = E ,\qquad B_{r_2}(q) = E\]  
            I.e. two balls with different centers and radii may be equal. 

            \bbni 

            \item Every singleton set $\{p\} \in E$ is open, as we can construct an open ball $B_r(p)$ with $0 < r < 1$ centered at $p$. Since unions of any collection of open sets are open, any subset of $E$ is open. 
    \end{enumerate}
\end{solution}

\newpage

\begin{problem}{3}
    Show that the subset of $E^2$ given by $\{(x_1, x_2) \in E^2 : x_1 > x_2\}$ is open.
\end{problem}

\begin{solution}
    \bbni

    Let $S = \{(x_1, x_2) \in E^2 : x_1 > x_2\}$, with distance function defined as  \\ $d((x_1,x_2), (y_1, y_2)) = \sqrt{(x_1-y_1)^2 + (x_2 - y_2)^2}$ for any two points $(x_1, x_2), (y_1, y_2) \in E^2$. To show that $S$ is an open subset we pick an arbitrary point in $S$ and show that there exists an open ball centered at that point contained entirely inside $S$.
    
    Let $B_{r/4}((x_1, x_2))$ be an open ball centered around an arbitrary point $(x_1, x_2) \in S$, with $r = x_1 - x_2$ (obviously, $r>0$), and let $(y_1, y_2) \in B_{r/4}((x_1,x_2))$.  Then, 

    \begin{align*}
        d((x_1,x_2), (y_1, y_2)) &< r/4 \\ 
        \sqrt{(x_1-y_1)^2 + (x_2 - y_2)^2} &< r/4 \\ 
        (x_1-y_1)^2 + (x_2 - y_2)^2 &< r^2/16 \\
        \implies (x_1 - y_1) ^2  \leq (x_1-y_1)^2 + (x_2 - y_2)^2 &< r^2/16 \\ 
        |x_1 - y_1| &< r/4 \\ 
    \intertext{ Similarly, we get} \\ 
        |x_2 - y_2| &< r/4 \\ 
    \end{align*}
    We can then write: 
        $y_1 - y_2 = x_1 - x_2 - (x_1 - y_1) + (x_2 - y_2)$. 

    We know that, $x_1 - x_2  = r,  |x_1 - y_1| < r/4, |x_2 - y_2| < r /4$. Then, \[y_1 - y_2 > r/2 \implies y_1 - y_2 > 0 \implies y_1 > y_2 \implies (y_1, y_2) \in S\]
    I.e. $B_{r/4}((x_1,x_2)) \in S$. Since, $(x_1, x_2)$ was an arbitrary point, this works for any element of $S$. $S$ is thus an open subset because of every point of $S$ we can construct such a ball. 
\end{solution}

\newpage 

\begin{problem}{4}
    Prove that any bounded open subset of $\R$ is the union of disjoint open intervals.
\end{problem}

\begin{solution}
    \bbni 

    Let $S$ be a bounded open subset of $\R$. $\forall p \in S$ let $S_p$ be the set of all open intervals in $S$ containing $p$. We know that $S_p$ is not empty because by definition of open sets there exists at least one ball centred at $p$ in $S$. (All open balls $B_\epsilon(a)$ are open intervals in $\R$ because they contain all elements between $a - \epsilon$ and $a + \epsilon$). 

    Consider the set $R_p$ of the right end point of each of the intervals in $S_p$. Again, we know that this set is not empty because there exists at least one interval in $S_p$. We also know that this set is bounded from above because $S$ has an upper bound. Since, $R_p$ is nonempty and bounded from above, there must exist a $l.u.b.$ for $R_p$. 
    Similarly, construct the set $L_p$ of the left end points of each of the intervals in $S_p$. By the same logic, there exist a $g.l.b.$ for $L_p$. 
    
    Now, consider the open interval $X_p = (g.l.b. \, L_p, l.u.b. \, R_p)$. To show that $X_p \subset S$, consider $x \in X_p$. Assume $x \notin S$ then since $S$ is open one can find intervals in $S_p$ arbitrarily close to $x$, contradicting the definitions of $g.l.b. \, L_p \, , \, l.u.b. \, R_p$. Hence, $x$ must be in $S$. Additionally, note that $X_p$ is the maximal open interval contained in $S$ that contains $p$, because any open interval in $S$ containing $p$ has  endpoints in $L_p, R_p$, and hence is contained in $X_p$.

    \bbni 

    \textbf{Claim: }If $X_q \cap X_p  \neq \emptyset $ then $ X_q = X_p$. 

    Suppose there exists $r \in X_p \cap X_q$. Then $r \in S$. Because $X_p$ is an open interval containing both $p$ and $r$ and $X_q$ is an open interval containing both $q$ and $r$, the union $X_p \cup X_q$ itself an open interval in $S$ containing $p$. But by construction, $X_p$ is the maximal open interval contained in $S$ that contains $p$.
    Hence, 
    \[X_p \cup X_q \subset X_p \implies X_q \subset X_p \]

    Similarly, we also get, 
        \[X_p \cup X_q \subset X_q \implies X_p \subset X_q \]

    Therefore, $X_p = X_q$. 

    So, either $X_q \cap X_p  = \emptyset $ or $X_q = X_p$.     
        
    Then we can write $S = \underset{p}{\bigcup} \, X_{p}$. By the claim, these intervals are either equal or disjoint, so selecting all unique intervals gives us a union of disjoint open intervals.
\end{solution}
\end{document}
