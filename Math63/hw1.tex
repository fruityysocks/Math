\documentclass[12pt]{article}

\usepackage{fullpage}
\usepackage{mdframed}
\usepackage{colonequals}
\usepackage{algpseudocode}
\usepackage{algorithm}
\usepackage[most, breakable]{tcolorbox}
\usepackage[all]{xy}
\usepackage{proof}
\usepackage{mathtools}
\usepackage{bbm}
\usepackage{amssymb}
\usepackage{amsthm}
\usepackage{amsmath}
\usepackage{amsxtra}
\usepackage{enumitem}
\newcommand{\bb}{\mathbb}


\newtheorem{theorem}{Theorem}[section]
\newtheorem{theorem*}{Theorem}
\newtheorem{definition}[theorem]{Definition}
\newtheorem{corollary}{Corollary}[theorem]
\newtheorem{lemma}[theorem]{Lemma}
\newtheorem{prop}[theorem]{Proposition}
\newtheorem{remark}[theorem]{Remark}


\newtheorem*{exercisehelper}{Exercise.}
\newenvironment{exercise}[1]{%
  \IfBlankTF{#1}
    {\renewcommand{\exercisehelper}{\textbf{Exercise} \unskip}}
    {\renewcommand\exercisehelper{\textbf{Exercise #1}}}%
  \exercisehelper
}{\endexercisehelper}

\theoremstyle{remark}
\newtheorem*{solution}{Solution}
\newcommand{\mathcat}[1]{\textup{\textbf{\textsf{#1}}}} % for defined terms

\newenvironment{problem}[1]
{ \begin{tcolorbox}[breakable]\noindent\textbf{Problem #1}.}
{\vskip 6pt \end{tcolorbox}}

\newenvironment{enumalph}
{\begin{enumerate}\renewcommand{\labelenumi}{\textnormal{(\alph{enumi})}}}
{\end{enumerate}}

\newenvironment{enumroman}
{\begin{enumerate}\renewcommand{\labelenumi}{\textnormal{(\roman{enumi})}}}
{\end{enumerate}}

\newcommand{\defi}[1]{\textsf{#1}} % for defined terms



\setlength{\hfuzz}{4pt}

\let\H\relax
\let\P\relax
\newcommand{\H}{\mathbb H}
\newcommand{\P}{\mathbb P}
\newcommand{\C}{\mathbb C}
\newcommand{\N}{\mathbb N}
\newcommand{\Q}{\mathbb Q}
\newcommand{\R}{\mathbb R}
\newcommand{\Z}{\mathbb Z}
\newcommand{\F}{\mathbb F}
\newcommand{\br}{\mathbf{r}}
\newcommand{\RP}{\mathbb{RP}}
\newcommand{\CP}{\mathbb{CP}}
\newcommand{\nbit}[1]{\{0, 1\}^{#1}}
\newcommand{\bits}{\{0, 1\}^{n}}
\newcommand{\bbni}{\bigbreak \noindent}
\newcommand{\norm}[1]{\left\vert\left\vert#1\right\vert\right\vert}
\newcommand{\dbar}{\overline{\partial}}
\let\d\relax
\newcommand{\d}{\partial}
\newcommand{\calO}{\mathcal{O}}
\newcommand{\calF}{\mathcal{F}}
\newcommand{\calG}{\mathcal{G}}
\newcommand{\calH}{\mathcal{H}}
\newcommand{\calE}{\mathcal{E}}
\newcommand{\calC}{\mathcal{C}}
\newcommand{\calD}{\mathcal{D}}

\let\1\relax
\newcommand{\1}{\mathbf{1}}
\newcommand{\fr}[2]{\left(\frac{#1}{#2}\right)}
\newcommand{\todo}[1]{\textcolor{red}{\textbf{TODO:} #1}}
\newcommand{\vecz}{\mathbf{z}}
\newcommand{\vecr}{\mathbf{r}}
\DeclareMathOperator{\Cinf}{C^{\infty}}
\DeclareMathOperator{\Id}{Id}
\DeclareMathOperator{\Ell}{Ell}
\DeclareMathOperator{\CL}{\mathcal{CL}}

\DeclareMathOperator{\Alt}{Alt}
\DeclareMathOperator{\Aut}{Aut}
\DeclareMathOperator{\ann}{ann}
\DeclareMathOperator{\codim}{codim}
\DeclareMathOperator{\End}{End}
\DeclareMathOperator{\Hom}{Hom}
\DeclareMathOperator{\id}{id}
\DeclareMathOperator{\M}{M}
\DeclareMathOperator{\Mat}{Mat}
\DeclareMathOperator{\Ob}{Ob}
\DeclareMathOperator{\opchar}{char}
\DeclareMathOperator{\opspan}{span}
\DeclareMathOperator{\rk}{rk}
\DeclareMathOperator{\sgn}{sgn}
\DeclareMathOperator{\Sym}{Sym}
\DeclareMathOperator{\tr}{tr}
\DeclareMathOperator{\img}{img}
\DeclareMathOperator{\coker}{coker}
\DeclareMathOperator{\Spec}{Spec}
\DeclareMathOperator{\CandE}{CandE}
\DeclareMathOperator{\CandO}{CandO}
\DeclareMathOperator{\argmax}{argmax}
\DeclareMathOperator{\first}{first}
\DeclareMathOperator{\last}{last}
\DeclareMathOperator{\cost}{cost}
\DeclareMathOperator{\dist}{dist}
\DeclareMathOperator{\path}{path}
\DeclareMathOperator{\parent}{parent}
\DeclareMathOperator{\argmin}{argmin}
\DeclareMathOperator{\excess}{excess}
\let\Pr\relax
\DeclareMathOperator{\Pr}{\mathbf{Pr}}
\DeclareMathOperator{\Exp}{\mathbb{E}}
\DeclareMathOperator{\Var}{\mathbf{Var}}
\let\limsup\relax
\DeclareMathOperator{\limsup}{limsup}
%Paired Delims
\DeclarePairedDelimiter\ceil{\lceil}{\rceil}
\let\oldceil\ceil
\renewcommand{\ceil}[1]{\oldceil*{#1}}

\DeclarePairedDelimiter{\floor}{\lfloor}{\rfloor}
\let\oldfloor\floor
\renewcommand{\floor}[1]{\oldfloor*{#1}}





\newcommand{\dagstar}{*}

\newcommand{\tbigwedge}{{\textstyle{\bigwedge}}}
\setlength{\parindent}{0pt}
\setlength{\parskip}{5pt}


\usepackage{listings}
\usepackage{courier}
\usepackage{microtype}


\lstset{
  basicstyle=\footnotesize\ttfamily,
  breaklines=true,
  breakatwhitespace=true
  columns=fullflexible,
  keepspaces=true,
  frame=single,
  escapeinside={(*@}{@*)}
}

\begin{document}

\title{Math 71: Abstract Algebra}

\author{Prishita Dharampal}
\date{}
\maketitle


\textbf{Credit Statement:} Talked to Sair Shaikh'26, and Math Stack Exchange. \todo{}

\begin{problem}{1}
    If $A, B, C$ are sets show that: \[A - (B- C)=(A-B)\cup(A \cap B \cap C)\]
\end{problem}

\begin{solution}
    \bbni 

    To show that $A - (B- C)=(A-B)\cup(A \cap B \cap C)$ we first show that \[A - (B- C) \subset (A-B)\cup(A \cap B \cap C)\] and then \[A - (B- C) \supset (A-B)\cup(A \cap B \cap C).\]

    \begin{enumerate}
        \item $A - (B- C) \subset (A-B)\cup(A \cap B \cap C)$ 

        $\forall x \in A - (B - C)$, 
        \begin{align*}
            &\implies x \in A \text{ and } x \notin (B-C) \\
            &\implies x \in A \text { and } (x \notin B \text{ or } x \in B \cap C) \\ 
            &\implies (x \in A \text{ and } x \notin B) \text{ or } (x \in A \text{ and } x \in B \cap C)  \\ 
            &\implies (x \in A \cap \bar{B}) \cup  (x \in A \cap B \cap C) \\ 
            &\implies (x \in A  -  B) \cup  (x \in A \cap B \cap C) \\ 
            &\implies x \in (A  -  B) \cup  (A \cap B \cap C) \\
            & \implies A - (B - C) \subset (A - B) \cup (A \cap B \cap C) 
        \end{align*}

        \newpage

        \item $A - (B- C) \supset (A-B)\cup(A \cap B \cap C)$ 

                $\forall x \in (A-B) \cup ( A \cap B \cap C)$, 
                \begin{align*}
                    &\implies (x \in A \text{ and } x \notin B) \text{ or } (x \in A \cap B \cap C) \\ 
                    &\implies (x \in A) \text{ and } (x \notin B \text{ or } x \in B \cap C) \\ 
                    &\implies x \in A \cap (\bar{B} \cup (B \cap C)) \\ 
                    &\implies x \in A \cap ((\bar{B} \cup B )\cap (\bar{B} \cup C)) \\ 
                    &\implies x \in A \cap (1 \cap (\bar{B} \cup C)) \\ 
                    &\implies x \in A \cap (\bar{B} \cup C) \\ 
                    &\implies x \in A - (\overline{\bar{B} \cup C}) \\
                    &\implies x \in A - (B \cap \bar{C}) \\
                    &\implies x \in A - (B - C) 
                \end{align*}    
    \end{enumerate}

    Since, $A - (B- C) \subset (A-B)\cup(A \cap B \cap C)$ and $A - (B- C) \supset (A-B)\cup(A \cap B \cap C)$ we can say that $A - (B- C) =  (A-B)\cup(A \cap B \cap C)$. 
\end{solution}

\newpage

\begin{problem}{2}
    Let $I$ be a set and for each $i \in I$ let $X_i$ , be a set. Prove that for any set $B$ we have:  
    \[B \cap \underset{i \in I}{\cup}X_i = \underset{i \in I}{\cup} (B\cap X_i)\]
\end{problem}

\begin{solution}
    \bbni 

    To show that $B \cap \underset{i \in I}{\cup}X_i = \underset{i \in I}{\cup} (B\cap X_i)$ we first show that  $B \cap \underset{i \in I}{\cup}X_i \subset \underset{i \in I}{\cup} (B\cap X_i)$ and then  $B \cap \underset{i \in I}{\cup}X_i \supset \underset{i \in I}{\cup} (B\cap X_i)$. 
    \begin{enumerate}
        \item $B \cap \underset{i \in I}{\cup}X_i \subset \underset{i \in I}{\cup} (B\cap X_i)$ 

        If $x \in B \cap \underset{i \in I }{\cup}X_i$ then $x \in B$ and $x \in \underset{i\in I}{\cup}X_i$. 
        
        I.e. $x$ is at least in one $X_j$ for some $j \in I \implies x \in B \cap X_j$. 
        
        Thus, $x \in \underset{i\in I}{\cup} (B \cap X_i) \implies B \cap \underset{i \in I}{\cup}X_i \subset \underset{i \in I}{\cup} (B\cap X_i)$. 

        \item $B \cap \underset{i \in I}{\cup}X_i \supset \underset{i \in I}{\cup} (B\cap X_i)$ 

        If $x \in \underset{i\in I}{\cup} (B \cap X_i)$, then $x$ is at least in one $B \cap X_j$ for some $j \in I$

                
        \begin{align*}
            &\implies x \in B \text{ and } x \in X_j \\ 
            &\implies x \in B \text{ and } x \in \underset{i \in I}{\cup} X_i \\
            &\implies x \in B \cap \underset{i\in I}{\cup} X_i
        \end{align*}
        
        $\implies B \cap \underset{i \in I}{\cup}X_i \supset \underset{i \in I} {\cup} (B \cap X_i)$. 


        Since, $B \cap \underset{i \in I}{\cup}X_i \subset \underset{i \in I} {\cup} (B \cap X_i)$ and $\implies B \cap \underset{i \in I}{\cup}X_i \supset \underset{i \in I} {\cup} (B \cap X_i)$, we can say that $\implies B \cap \underset{i \in I}{\cup}X_i = \underset{i \in I} {\cup} (B \cap X_i)$. 

    \end{enumerate}
\end{solution}


\newpage

\begin{problem}{3}
    Let $f: X\to Y$ be a function, let $A$ and $B$ be subsets of $X$, and let $C$ and $D$ be subsets of $Y$. Prove that: 
    \begin{enumerate}
    \item $f(A\cap B) \subset f(A) \cap f(B)$
    
    \item $f^{-1}(C \cap D ) =  f^{-1}(C) \cap f^{-1}(D)$

    \end{enumerate}
\end{problem}

\begin{solution}
    \bbni 
    \begin{enumerate}
        \item  Let $x \in f(A \cap B), x = f(y)$ 
        \begin{align*}
            &\implies y \in A, B \\ 
            &\implies x \in f(A),  x \in f(B) \\ 
            &\implies x \in f(A) \cap f(B) \\ 
            &\implies f(A \cap B) \subset f(A) \cap f(B) 
        \end{align*}
        For an arbitrary $f$ the reverse containment isn't true. To show this, consider distinct elements $a \in A, a \notin B, b \in B, b \notin A$ such that for some $c$, $f(a) = c, f(b) = c$. Then $c \in f(A) \cap f(B)$ but $c \notin f (A \cap B)$. I.e. equality won't hold unless $f$ is injective. 

        \item Let $x \in f^{-1}(C \cap D), y  = f(x)$ 
        
        $(\implies)$
        \begin{align*}
            &\implies y \in C,  y \in D \\
            &\implies x \in f^{-1}(C), f^{-1}(D) \\ 
            &\implies x \in f^{-1}(C) \cap f^{-1}(D) \\ 
            &\implies f^{-1}( C \cap D) \subset f^{-1}(C) \cap f^{-1}(D) 
        \end{align*}

        $(\impliedby)$ 

        Let $x \in f^{-1}(C) \cap f^{-1}(D), y = f(x)$
        \begin{align*}
            &\implies x \in f^{-1}(C),  x \in f^{-1}(D)  \\
            &\implies y \in C, D \\ 
            &\implies y \in C \cap D \\ 
            &\implies x \in f^{-1} (C \cap D)
        \end{align*}
    \end{enumerate}
\end{solution}

\newpage

\begin{problem}{4}
    \begin{enumerate}
    \item How many functions are there from a nonempty set $S$ into the $\emptyset$?
    
    \item Show that the notation $\{X_i\}_{i\in I }$ implicitly involves the notion of function.
    \end{enumerate}
\end{problem}


\begin{solution}
    \bbni 

    \begin{enumerate}
        \item There are no functions from a nonempty set to the empty set. A function needs to assign a definite output to every input. Since there are no elements in $\emptyset$, that is impossible. 
        \item The notation $\{X_i\}_{i \in I}$ describes a rule from elements of $I$ to corresponding objects $X_i$. This by definition is a function from $I$ to the set consisting objects $X_i$. 
    \end{enumerate}
\end{solution}

\vspace{3em}

\begin{problem}{5} 
    Prove in detail that for any $a, b \in  \R$: 
    \[ - (a - b) = b - a \]
\end{problem}

\begin{solution}
    \bbni  

    Because $R$ is a field, we can say that, 
    \begin{align*}
        - (a - b) &= - (a) - (-b) \text{ (Field Property 8)} \\ 
        &= -a + b \text{ (Field Property 6)} \\
        &= b - a \text{ (Commutativity)}
    \end{align*}

    Hence, $- (a - b) = b - a$ is true for any $a, b \in \R$.
\end{solution}

\newpage

\begin{problem}{6}
    Show that if $a , b , x, y  \in \R$ and $a <x <b, a <y <b$, then $\mid y- x \mid<b- a $.
\end{problem}

\begin{solution}
    \bbni
    
    $b > a \implies b - a > a - a \implies b - a > 0$. Hence, $b-a$ is always positive. 

    There are 3 cases: 
    \begin{enumerate}
        \item $x = y$ 
        
        Then $y - x < b - a$, is trivially true since $0 < b - a$. 
        \item $x > y$ 
        
        \begin{align*}
            y &< x \\ 
            y - x &< x - x \\
            y  - x &< 0 \\ 
            y  - x &< 0 < b -a\\ 
        \end{align*}
        \item $x < y$ 

        \begin{align*}
            y &< b \\
            a &< x \\
            \implies y-x &< b-a \\
        \end{align*}

    \end{enumerate}
    
\end{solution}

\newpage 

\begin{problem}{7}
    Find the g.l.b. and l.u.b. of  $\left \{1, \frac{1}{2},\frac{1}{3}, \frac{1}{4}, \cdots \right \}$, giving reasons if you can.
\end{problem}


\begin{solution}
    \bbni 

    Let $S = \left \{1, \frac{1}{2}, \frac{1}{3}, \frac{1}{4}, \cdots \right \} = \left \{\frac{1}{n}, n \in \N \right \}$. 
    
    The set $S \subset \R$ is nonempty, and bounded from above since $\forall \frac{1}{n}, \, \frac{1}{n+1} \, \in S, \frac{1}{n} > \frac{1}{n+1}$. Hence $y = l.u.b. \, S$ exists. From the inequality above, we can see that the larger $n$ is, the smaller the resulting fraction is. I.e. the largest value occurs at $n = 1$. Because the maximum element of the set is $1$, \textbf{the least upper bound is $1$}. 

    \bbni 
    $\frac{1}{n} > 0, \forall n > 0$ (7th consequence of the order property). Hence, the set $S \subset \R$ is nonempty, and bounded from below by $0$. I.e. $y = g.l.b. \, S$ exists. To show that $0$ is the greatest lower bound, let $\epsilon > 0$, choose $n > \frac{1}{\epsilon}$. Then $\epsilon > \frac{1}{n}$ (L.U.B. 2). 
    \[\implies 0 < \frac{1}{n}< \epsilon \]
    Hence, no $\epsilon > 0$ can be a lower bound for the set $S$. Therefore, \textbf{the greatest lower bound is} $0$. 
    \todo{}
\end{solution}

\newpage

\begin{problem}{8}
    Prove that if $a \in  \R, a > 1$, then the set $\{a, a^2, a^3, ...\}$ is not bounded from above. (Hint: First find a positive integer $n$ such that $a > 1 + \frac{1}{n}$ and prove that $a^n > \left (  1+ \frac{1}{n} \right)^n \geq 2 $). 
\end{problem}

\begin{solution}
    \bbni 


\end{solution}

\newpage 

\begin{problem}{9}
    If $S_1, S_2$ are nonempty subsets of $\R$ that are bounded from above, prove that \[\text{l.u.b } \{ x + y : x \in S_1, y \in S_2\} = \text{l.u.b. }S_1+ \text{l.u.b. }S_2\]
\end{problem}

\begin{solution}
    \bbni 

     If $S_1, S_2$ are nonempty subsets of $\R$ that are bounded from above, we know that $l.u.b. \, S_1,$ and $l.u.b. \, S_2$ exist. 
     
     Let $S= \{ x + y, x \in S_1, y \in S_2\}$, $l.u.b. \, S_1 = l_1, \, l.u.b. \, S_2 = l_2$. 
     
     By definition $\forall x \in S_1, x \leq l_1$ and $\forall y \in S_2, y \leq l_2$. Adding the inequalities, we get  $x + y \leq l_1 + l_2$. Hence, $l_1 + l_2$ is an upper bound of $S$. 

     Let $\epsilon$ be an arbitrarily small positive number. Since $l_1 = l.u.b. \, S_1$, $\exists x_0 \in S_1$ such that $l_1 - \epsilon/2 < x_0$.
     \[l_1 - \epsilon/2  < x_0 \leq l_1 \] 
     Similarly, since $l_2 = l.u.b. \, S_2$, $\exists y_0 \in S_2$ such that $l_2 - \epsilon/2 < y_0$.

     \[l_2 - \epsilon/2  < y_0 \leq l_2 \] 

     Adding the two inequalities gives, 
     \[ l_1 + l_2 -  \epsilon < x_0 + y_0 \leq l_1 + l_2\]
     Hence,$ l_1 + l_2$ is the least upper bound for the set $S$. 
\end{solution}

\end{document}
