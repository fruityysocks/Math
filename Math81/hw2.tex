\documentclass[12pt]{article}

\usepackage{fullpage}
\usepackage{mdframed}
\usepackage{colonequals}
\usepackage{algpseudocode}
\usepackage{algorithm}
\usepackage[most, breakable]{tcolorbox}
\usepackage[all]{xy}
\usepackage{proof}
\usepackage{mathtools}
\usepackage{bbm}
\usepackage{amssymb}
\usepackage{amsthm}
\usepackage{amsmath}
\usepackage{amsxtra}
\usepackage{enumitem}
\newcommand{\bb}{\mathbb}


\newtheorem{theorem}{Theorem}[section]
\newtheorem{theorem*}{Theorem}
\newtheorem{definition}[theorem]{Definition}
\newtheorem{corollary}{Corollary}[theorem]
\newtheorem{lemma}[theorem]{Lemma}
\newtheorem{prop}[theorem]{Proposition}
\newtheorem{remark}[theorem]{Remark}


\newtheorem*{exercisehelper}{Exercise.}
\newenvironment{exercise}[1]{%
  \IfBlankTF{#1}
    {\renewcommand{\exercisehelper}{\textbf{Exercise} \unskip}}
    {\renewcommand\exercisehelper{\textbf{Exercise #1}}}%
  \exercisehelper
}{\endexercisehelper}

\theoremstyle{remark}
\newtheorem*{solution}{Solution}
\newcommand{\mathcat}[1]{\textup{\textbf{\textsf{#1}}}} % for defined terms

\newenvironment{problem}[1]
{ \begin{tcolorbox}[breakable]\noindent\textbf{Problem #1}.}
{\vskip 6pt \end{tcolorbox}}

\newenvironment{enumalph}
{\begin{enumerate}\renewcommand{\labelenumi}{\textnormal{(\alph{enumi})}}}
{\end{enumerate}}

\newenvironment{enumroman}
{\begin{enumerate}\renewcommand{\labelenumi}{\textnormal{(\roman{enumi})}}}
{\end{enumerate}}

\newcommand{\defi}[1]{\textsf{#1}} % for defined terms



\setlength{\hfuzz}{4pt}

\let\H\relax
\let\P\relax
\newcommand{\H}{\mathbb H}
\newcommand{\P}{\mathbb P}
\newcommand{\C}{\mathbb C}
\newcommand{\N}{\mathbb N}
\newcommand{\Q}{\mathbb Q}
\newcommand{\R}{\mathbb R}
\newcommand{\Z}{\mathbb Z}
\newcommand{\F}{\mathbb F}
\newcommand{\br}{\mathbf{r}}
\newcommand{\RP}{\mathbb{RP}}
\newcommand{\CP}{\mathbb{CP}}
\newcommand{\nbit}[1]{\{0, 1\}^{#1}}
\newcommand{\bits}{\{0, 1\}^{n}}
\newcommand{\bbni}{\bigbreak \noindent}
\newcommand{\norm}[1]{\left\vert\left\vert#1\right\vert\right\vert}
\newcommand{\dbar}{\overline{\partial}}
\let\d\relax
\newcommand{\d}{\partial}
\newcommand{\calO}{\mathcal{O}}
\newcommand{\calF}{\mathcal{F}}
\newcommand{\calG}{\mathcal{G}}
\newcommand{\calH}{\mathcal{H}}
\newcommand{\calE}{\mathcal{E}}
\newcommand{\calC}{\mathcal{C}}
\newcommand{\calD}{\mathcal{D}}

\let\1\relax
\newcommand{\1}{\mathbf{1}}
\newcommand{\fr}[2]{\left(\frac{#1}{#2}\right)}
\newcommand{\todo}[1]{\textcolor{red}{\textbf{TODO:} #1}}
\newcommand{\vecz}{\mathbf{z}}
\newcommand{\vecr}{\mathbf{r}}
\DeclareMathOperator{\Cinf}{C^{\infty}}
\DeclareMathOperator{\Id}{Id}
\DeclareMathOperator{\Ell}{Ell}
\DeclareMathOperator{\CL}{\mathcal{CL}}

\DeclareMathOperator{\Alt}{Alt}
\DeclareMathOperator{\Aut}{Aut}
\DeclareMathOperator{\ann}{ann}
\DeclareMathOperator{\codim}{codim}
\DeclareMathOperator{\End}{End}
\DeclareMathOperator{\Hom}{Hom}
\DeclareMathOperator{\id}{id}
\DeclareMathOperator{\M}{M}
\DeclareMathOperator{\Mat}{Mat}
\DeclareMathOperator{\Ob}{Ob}
\DeclareMathOperator{\opchar}{char}
\DeclareMathOperator{\opspan}{span}
\DeclareMathOperator{\rk}{rk}
\DeclareMathOperator{\sgn}{sgn}
\DeclareMathOperator{\Sym}{Sym}
\DeclareMathOperator{\tr}{tr}
\DeclareMathOperator{\img}{img}
\DeclareMathOperator{\coker}{coker}
\DeclareMathOperator{\Spec}{Spec}
\DeclareMathOperator{\CandE}{CandE}
\DeclareMathOperator{\CandO}{CandO}
\DeclareMathOperator{\argmax}{argmax}
\DeclareMathOperator{\first}{first}
\DeclareMathOperator{\last}{last}
\DeclareMathOperator{\cost}{cost}
\DeclareMathOperator{\dist}{dist}
\DeclareMathOperator{\path}{path}
\DeclareMathOperator{\parent}{parent}
\DeclareMathOperator{\argmin}{argmin}
\DeclareMathOperator{\excess}{excess}
\let\Pr\relax
\DeclareMathOperator{\Pr}{\mathbf{Pr}}
\DeclareMathOperator{\Exp}{\mathbb{E}}
\DeclareMathOperator{\Var}{\mathbf{Var}}
\let\limsup\relax
\DeclareMathOperator{\limsup}{limsup}
%Paired Delims
\DeclarePairedDelimiter\ceil{\lceil}{\rceil}
\let\oldceil\ceil
\renewcommand{\ceil}[1]{\oldceil*{#1}}

\DeclarePairedDelimiter{\floor}{\lfloor}{\rfloor}
\let\oldfloor\floor
\renewcommand{\floor}[1]{\oldfloor*{#1}}





\newcommand{\dagstar}{*}

\newcommand{\tbigwedge}{{\textstyle{\bigwedge}}}
\setlength{\parindent}{0pt}
\setlength{\parskip}{5pt}


\usepackage{listings}
\usepackage{courier}
\usepackage{microtype}


\lstset{
  basicstyle=\footnotesize\ttfamily,
  breaklines=true,
  breakatwhitespace=true
  columns=fullflexible,
  keepspaces=true,
  frame=single,
  escapeinside={(*@}{@*)}
}
\usepackage{euler}

\begin{document}

\title{Math 71: Abstract Algebra}

\author{Prishita Dharampal}
\date{}
\maketitle


\textbf{Credit Statement:} Talked to Sair Shaikh'26, and Math Stack Exchange.

\begin{problem}{1} 
    \textbf{Subgroups of fields.}

    Let $F$ be a field.
    \begin{enumerate}
    \item Let $G$ be a finite abelian group. Prove that $G$ is cyclic if and only if $G$ has at most
    $m$ elements of order dividing $m$ for each $m \mid \#G$.  
    \emph{Hint.} One possible proof uses the structure theorem of finite abelian groups, but you can get away with slightly less.

    \item Prove that every finite subgroup $G$ of the multiplicative group $F^\times = F \setminus \{0\}$ is cyclic.  
    \emph{Hint.} Use the fact that a polynomial of degree $m$ has at most $m$ roots in $F$.

    \item Deduce that if $F$ is a finite field then $F^\times$ is cyclic. For each field $F$ having at most $7$ elements, find an explicit generator of $F^\times$.

    \item Let $p$ be an odd prime. Prove that $-1 \in \mathbb{F}_p^\times$ is a square if and only if $p \equiv 1 \pmod{4}$.

    \item Prove that for any odd prime $p$, the set of nonzero squares is an index $2$ subgroup of $\mathbb{F}_p^\times$.  
    \emph{Hint.} You can use the above results, but there's also a purely combinatorial proof.
    \end{enumerate}
\end{problem}

\begin{solution}
    \bbni 
    \begin{enumerate}
        \item $(\implies)$ 
        
        Assume $G$ is cyclic. Then we know that for every positive integer $m$, $m | |G|$ there exists a subgroup $H$ of $G$ with order $m$. Since any $H$ would have at most $m$ elements of order dividing $m$, there are at most $m$ elements in $G$ with order dividing $m$.

        $(\impliedby)$

        Assume group $G$ has at most $m$ elements of order dividing $m$ for each $m| |G|$.
        \todo{}

        \item Let $G$ be a finite subgroup of the multiplicative group $F^\times$. Then, 
        \item 
        \item  ($\implies$) \\ 
                If $-1$ is a square in $\F_p^\times$, $p \neq 2$ then there exists an element $n \in \F_p$ with order 4 ($|-1 | = 2$). That means if there exists  square root of $-1$ in $\F_p^\times$ then $4 \bigm | |\F_p^\times | \implies 4 \mid p - 1 \implies p \equiv 1$ (mod $4$).  \\ 
        
                ($\impliedby)$ \\ 
                If $p \equiv 1 \text{ (mod $4$)} \implies p -1  \equiv 0 \text{ (mod $4$)} \implies 4 \mid p-1 \implies 4 \mid | \F_p^\times | \implies \exists x \in \F_p^\times$, such that $\mid x \mid = 4$ (converse of Lagrange's Theorem for Abelian Groups). \\
                
                $x^4 = 1 \implies (x^2)^2 = 1 \implies x^2 = \pm 1$.  But if $x^2 = 1$, the order of $x$ would be 2, hence a contradiction. I.e. $x^2$ must be $-1$, and $x = \sqrt{-1}$. 
        
                Hence, $-1 \in \mathbb{F}_p^\times$ is a square if and only if $p \equiv 1 \pmod{4}$. 
        
        \item 
    \end{enumerate}
\end{solution}

\begin{problem}{2} \textbf{Reducibility of $x^4+1$ modulo primes.}
        
    The goal is to prove that $f(x) = x^4 + 1 \in \mathbb{Z}[x]$ is reducible modulo every prime number $p$. You already know (HW\#1) that $f(x)$ is irreducible in $\mathbb{Q}[x]$.
    \begin{enumerate}
        \item Factor $f(x)$ modulo $2$.

        \item Assume that $-1 = u^2$ is a square in $\mathbb{F}_p$. Then use the equality
        \[
        x^4 + 1 = x^4 - u^2
        \]
        to factor $f(x)$ modulo $p$.

        \item Assume that $p$ is odd and $2 = v^2$ is a square in $\mathbb{F}_p$. Then use the equality
        \[
        x^4 + 1 = (x^2 + 1)^2 - (vx)^2
        \]
        to factor $f(x)$ modulo $p$.

        \item Prove that if $p$ is odd and neither $-1$ nor $2$ is a square in $\mathbb{F}_p$, then $-2$ is a square. In this case, factor $f(x)$ modulo any such $p$.  
        \emph{Hint.} For the first part, use the previous problem.

        \item Conclude that $x^4 + 1$ is reducible modulo every prime $p$.
        \end{enumerate}
        \end{problem}

\begin{problem}{3. Field homomorphisms}

    Let $K$ and $K'$ be field extensions of a field $F$.
    \begin{enumerate}
        \item Prove that any $F$-homomorphism $\varphi : K \to K'$ is injective.

        \item Prove that if $K'/F$ is finite and $\varphi : K \to K'$ is an $F$-homomorphism, then $K/F$ is finite.

        \item Assume that both $K$ and $K'$ are finite over $F$, and that $\varphi : K \to K'$ is an $F$-homomorphism. Then $\varphi$ is an $F$-isomorphism if and only if $[K:F] = [K':F]$.

        \item Prove that $f(x) = x^2 - 4x + 2 \in \mathbb{Q}[x]$ is irreducible. Prove that the extensions
        \[
        K = \mathbb{Q}[x]/(f(x)) \quad \text{and} \quad \mathbb{Q}(\sqrt{2})
        \]
        of $\mathbb{Q}$ are $\mathbb{Q}$-isomorphic and exhibit an explicit $\mathbb{Q}$-isomorphism between them.
    \end{enumerate}
\end{problem}

\begin{problem}{4} \textbf{Inverses in a cubic extension.}

    Let $\alpha \approx -1.7693$ be the real root of $x^3 - 2x + 2$. In the extension $\mathbb{Q}(\alpha)/\mathbb{Q}$, write the elements $\alpha^{-1}$ and $(\alpha+1)^{-1}$ explicitly as a polynomial in $\alpha$ with coefficients in $\mathbb{Q}$.  

    \emph{Hint.} Remember the algorithm using the Bézout identity (e.g.\ FT pp.\ 16--17).
\end{problem}

\begin{problem}{5} \textbf{Quadratic extensions.}

    Let $F$ be a field of characteristic $\neq 2$ and let $K/F$ be a field extension of degree $2$.
    \begin{enumerate}
        \item Prove that there exists $\alpha \in K$ with $\alpha^2 \in F$ such that $K = F(\alpha)$. We often write $\alpha = \sqrt{a}$ if $\alpha^2 = a \in F$.  
        \emph{Hint.} Get inspiration from the quadratic formula.

        \item For $a,b \in F^\times$ prove that $F(\sqrt{a}) \cong F(\sqrt{b})$ if and only if $a = u^2 b$ for some $u \in F^\times$.

        \item Deduce that there is a bijection between the set of $F$-isomorphism classes of field extensions $K/F$ with $[K:F] \mid 2$ and the group $F^\times / F^{\times 2}$.

        \item If $F$ is a finite field of characteristic $\neq 2$, prove that $F$ has a unique quadratic extension (up to $F$-isomorphism).
    \end{enumerate}
\end{problem}

\begin{problem}{6} \textbf{Minimal polynomials.}

    For each extension $K/F$ and each element $\alpha \in K$, find the minimal polynomial of $\alpha$ over $F$ (and prove that it is the minimal polynomial).
    \begin{enumerate}
        \item $i$ in $\mathbb{C}/\mathbb{R}$
        \item $i$ in $\mathbb{C}/\mathbb{Q}$
        \item $\dfrac{1+\sqrt{5}}{2}$ in $\mathbb{R}/\mathbb{Q}$
        \item $\sqrt{2} + \sqrt{2}$ in $\mathbb{R}/\mathbb{Q}$
    \end{enumerate}
\end{problem}

\begin{problem}{7} \textbf{Transcendental and algebraic extensions.}

    Let $\pi \in \mathbb{R}$ be the area of a unit circle and let $\alpha = \sqrt{\pi^2 + 2}$. Consider the field $K = \mathbb{Q}(\pi,\alpha)$.

    For the following field extensions, determine whether they are transcendental and/or algebraic and/or finite and/or simple, and if you determine the extension is simple and algebraic, find a simple generator and determine its minimal polynomial.
    \begin{enumerate}
        \item $K/\mathbb{Q}$
        \item $K/\mathbb{Q}(\pi)$
        \item $K/\mathbb{Q}(\alpha)$
        \item $K/\mathbb{Q}(\pi+\alpha)$
    \end{enumerate}
\end{problem}
\end{document}