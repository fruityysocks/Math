\documentclass[12pt]{article}

\usepackage{fullpage}
\usepackage{mdframed}
\usepackage{colonequals}
\usepackage{algpseudocode}
\usepackage{algorithm}
\usepackage[most, breakable]{tcolorbox}
\usepackage[all]{xy}
\usepackage{proof}
\usepackage{mathtools}
\usepackage{bbm}
\usepackage{amssymb}
\usepackage{amsthm}
\usepackage{amsmath}
\usepackage{amsxtra}
\usepackage{enumitem}
\newcommand{\bb}{\mathbb}


\newtheorem{theorem}{Theorem}[section]
\newtheorem{theorem*}{Theorem}
\newtheorem{definition}[theorem]{Definition}
\newtheorem{corollary}{Corollary}[theorem]
\newtheorem{lemma}[theorem]{Lemma}
\newtheorem{prop}[theorem]{Proposition}
\newtheorem{remark}[theorem]{Remark}


\newtheorem*{exercisehelper}{Exercise.}
\newenvironment{exercise}[1]{%
  \IfBlankTF{#1}
    {\renewcommand{\exercisehelper}{\textbf{Exercise} \unskip}}
    {\renewcommand\exercisehelper{\textbf{Exercise #1}}}%
  \exercisehelper
}{\endexercisehelper}

\theoremstyle{remark}
\newtheorem*{solution}{Solution}
\newcommand{\mathcat}[1]{\textup{\textbf{\textsf{#1}}}} % for defined terms

\newenvironment{problem}[1]
{ \begin{tcolorbox}[breakable]\noindent\textbf{Problem #1}.}
{\vskip 6pt \end{tcolorbox}}

\newenvironment{enumalph}
{\begin{enumerate}\renewcommand{\labelenumi}{\textnormal{(\alph{enumi})}}}
{\end{enumerate}}

\newenvironment{enumroman}
{\begin{enumerate}\renewcommand{\labelenumi}{\textnormal{(\roman{enumi})}}}
{\end{enumerate}}

\newcommand{\defi}[1]{\textsf{#1}} % for defined terms



\setlength{\hfuzz}{4pt}

\let\H\relax
\let\P\relax
\newcommand{\H}{\mathbb H}
\newcommand{\P}{\mathbb P}
\newcommand{\C}{\mathbb C}
\newcommand{\N}{\mathbb N}
\newcommand{\Q}{\mathbb Q}
\newcommand{\R}{\mathbb R}
\newcommand{\Z}{\mathbb Z}
\newcommand{\F}{\mathbb F}
\newcommand{\br}{\mathbf{r}}
\newcommand{\RP}{\mathbb{RP}}
\newcommand{\CP}{\mathbb{CP}}
\newcommand{\nbit}[1]{\{0, 1\}^{#1}}
\newcommand{\bits}{\{0, 1\}^{n}}
\newcommand{\bbni}{\bigbreak \noindent}
\newcommand{\norm}[1]{\left\vert\left\vert#1\right\vert\right\vert}
\newcommand{\dbar}{\overline{\partial}}
\let\d\relax
\newcommand{\d}{\partial}
\newcommand{\calO}{\mathcal{O}}
\newcommand{\calF}{\mathcal{F}}
\newcommand{\calG}{\mathcal{G}}
\newcommand{\calH}{\mathcal{H}}
\newcommand{\calE}{\mathcal{E}}
\newcommand{\calC}{\mathcal{C}}
\newcommand{\calD}{\mathcal{D}}

\let\1\relax
\newcommand{\1}{\mathbf{1}}
\newcommand{\fr}[2]{\left(\frac{#1}{#2}\right)}
\newcommand{\todo}[1]{\textcolor{red}{\textbf{TODO:} #1}}
\newcommand{\vecz}{\mathbf{z}}
\newcommand{\vecr}{\mathbf{r}}
\DeclareMathOperator{\Cinf}{C^{\infty}}
\DeclareMathOperator{\Id}{Id}
\DeclareMathOperator{\Ell}{Ell}
\DeclareMathOperator{\CL}{\mathcal{CL}}

\DeclareMathOperator{\Alt}{Alt}
\DeclareMathOperator{\Aut}{Aut}
\DeclareMathOperator{\ann}{ann}
\DeclareMathOperator{\codim}{codim}
\DeclareMathOperator{\End}{End}
\DeclareMathOperator{\Hom}{Hom}
\DeclareMathOperator{\id}{id}
\DeclareMathOperator{\M}{M}
\DeclareMathOperator{\Mat}{Mat}
\DeclareMathOperator{\Ob}{Ob}
\DeclareMathOperator{\opchar}{char}
\DeclareMathOperator{\opspan}{span}
\DeclareMathOperator{\rk}{rk}
\DeclareMathOperator{\sgn}{sgn}
\DeclareMathOperator{\Sym}{Sym}
\DeclareMathOperator{\tr}{tr}
\DeclareMathOperator{\img}{img}
\DeclareMathOperator{\coker}{coker}
\DeclareMathOperator{\Spec}{Spec}
\DeclareMathOperator{\CandE}{CandE}
\DeclareMathOperator{\CandO}{CandO}
\DeclareMathOperator{\argmax}{argmax}
\DeclareMathOperator{\first}{first}
\DeclareMathOperator{\last}{last}
\DeclareMathOperator{\cost}{cost}
\DeclareMathOperator{\dist}{dist}
\DeclareMathOperator{\path}{path}
\DeclareMathOperator{\parent}{parent}
\DeclareMathOperator{\argmin}{argmin}
\DeclareMathOperator{\excess}{excess}
\let\Pr\relax
\DeclareMathOperator{\Pr}{\mathbf{Pr}}
\DeclareMathOperator{\Exp}{\mathbb{E}}
\DeclareMathOperator{\Var}{\mathbf{Var}}
\let\limsup\relax
\DeclareMathOperator{\limsup}{limsup}
%Paired Delims
\DeclarePairedDelimiter\ceil{\lceil}{\rceil}
\let\oldceil\ceil
\renewcommand{\ceil}[1]{\oldceil*{#1}}

\DeclarePairedDelimiter{\floor}{\lfloor}{\rfloor}
\let\oldfloor\floor
\renewcommand{\floor}[1]{\oldfloor*{#1}}





\newcommand{\dagstar}{*}

\newcommand{\tbigwedge}{{\textstyle{\bigwedge}}}
\setlength{\parindent}{0pt}
\setlength{\parskip}{5pt}


\usepackage{listings}
\usepackage{courier}
\usepackage{microtype}


\lstset{
  basicstyle=\footnotesize\ttfamily,
  breaklines=true,
  breakatwhitespace=true
  columns=fullflexible,
  keepspaces=true,
  frame=single,
  escapeinside={(*@}{@*)}
}
\usepackage{euler}



\begin{document}

\title{Math 71: Abstract Algebra}

\author{Prishita Dharampal}
\date{}
\maketitle


\textbf{Credit Statement:} Talked to Sair Shaikh'26, and Math Stack Exchange.


\begin{problem}{1} 
    \textbf{Subgroups of fields.}

    Let $F$ be a field.
    \begin{enumerate}
    \item Let $G$ be a finite abelian group. Prove that $G$ is cyclic if and only if $G$ has at most
    $m$ elements of order dividing $m$ for each $m \mid \#G$.  
    \emph{Hint.} One possible proof uses the structure theorem of finite abelian groups, but you can get away with slightly less.

    \item Prove that every finite subgroup $G$ of the multiplicative group $F^\times = F \setminus \{0\}$ is cyclic.  
    \emph{Hint.} Use the fact that a polynomial of degree $m$ has at most $m$ roots in $F$.

    \item Deduce that if $F$ is a finite field then $F^\times$ is cyclic. For each field $F$ having at most $7$ elements, find an explicit generator of $F^\times$.

    \item Let $p$ be an odd prime. Prove that $-1 \in \mathbb{F}_p^\times$ is a square if and only if $p \equiv 1 \pmod{4}$.

    \item Prove that for any odd prime $p$, the set of nonzero squares is an index $2$ subgroup of $\mathbb{F}_p^\times$.  
    \emph{Hint.} You can use the above results, but there's also a purely combinatorial proof.
    \end{enumerate}
\end{problem}

\begin{solution}
    \bbni 
    \begin{enumerate}
        \item $(\implies)$ 
        
        Assume $G$ is cyclic. Then we know that for every positive integer $m$, $m \big | |G|$ there exists a unique cyclic subgroup $H$ of $G$ with order $m$. Since $H$ has $m$ elements and $\forall h \in H, |h| \big | m$, $H$ would have at most $m$ elements of order dividing $m$. Thus, there are at most $m$ elements in $G$ with order dividing $m$. 

        $(\impliedby)$

        To show that if group $G$ has at most $m$ elements of order dividing $m$ for each $m \big |  |G|$, then it must be cyclic, we prove the contrapositive.
        
        To Show: If a abelian group $G$ is not cyclic, then it has more than $m$ elements of order dividing $m$ for at least one $m \big | |G|$. 


        Then $G$ cannot have a unique subgroup of order $p$ for every prime $p \big | |G|$, since otherwise the product of generators of these cyclic subgroups would generate $G$, making it cyclic. Thus, there exists a prime $p \mid |G|$ such that $G$ has at least two distinct subgroups of order $p$. Each subgroup of order $p$ has exactly $p-1$ non-identity elements of order $p$, and different subgroups of order $p$ intersect trivially. Hence $G$ has at least $(2p -1)$ elements whose order divides $p$. Hence, there exists at least a $p$ such that $G$ has more than $p$ elements of order diving $p$. Hence, proving the contrapositive, and the original proposition. 

        A finite, abelian group $G$ is cyclic if and only if $G$ has at most $m$ elements of order dividing $m$ for each $m \big | |G|$.  

        \item Let $G$ be an arbitrary finite subgroup of the multiplicative group $F^\times$. $G$ is abelian because $F$ is a field. Then every element of order diving $m$ for all $m \big | |G|$, would be a root to the equation 
        \[f(x)=x^m - 1 = 0\]
        Since, $f(x)$ has at most $m$ roots in $F$, there are at most $m$ elements of order dividing $m$ for every $m\big ||G|$. Hence, by subpart (1), $G$ is cyclic. Since $G$ was an arbitrary subgroup, every finite subgroup of the multiplicative group $F^\times$ is also cyclic.

        \item  Assume $F$ is a finite field. $F^\times$ is a finite subgroup of $F^\times$, and hence is cyclic (subpart (2)). 
        
        \begin{enumerate}
            \item Field of order $2$ $ = \{0,1\}$: $\F_2^\times$ is generated by $1$.  
            \item Field of order $3$ $= \{0, 1, 2, 3\}$: $\F_3^\times$ is generated by $2$. 
            \item Field of order $4$ $ = \{0, 1, x, y\}$: $\F_4^\times$ is generated by $x$. 
            \item Field of order $5$ $= \{0, 1, 2, 3, 4\}$: $\F_5^\times$ is generated by $3$. 
            \item Field of order $7$ $ = \{0, 1, 2, 3, 5, 6\}$:   $\F_7^\times$ is generated by $3$. 
        \end{enumerate}
        Note: Since $\F^\times$ is cyclic, any non identity element would generate it. 
        \item  ($\implies$) \\ 
                If $-1$ is a square in $\F_p^\times$, $p \neq 2$ then there exists an element $n \in \F_p$ with order 4 ($|-1 | = 2$). That means if there exists  square root of $-1$ in $\F_p^\times$ then $4 \bigm | |\F_p^\times | \implies 4 \mid p - 1 \implies p \equiv 1$ (mod $4$).  \\ 
        
                ($\impliedby)$ \\ 
                If $p \equiv 1 \text{ (mod $4$)} \implies p -1  \equiv 0 \text{ (mod $4$)} \implies 4 \mid p-1 \implies 4 \mid | \F_p^\times | \implies \exists x \in \F_p^\times$, such that $\mid x \mid = 4$ (converse of Lagrange's Theorem for Abelian Groups). \\
                
                $x^4 = 1 \implies (x^2)^2 = 1 \implies x^2 = \pm 1$.  But if $x^2 = 1$, the order of $x$ would be 2, hence a contradiction. I.e. $x^2$ must be $-1$, and $x = \sqrt{-1}$. 
        
                Hence, $-1 \in \F_p^\times$ is a square if and only if $p \equiv 1 \pmod{4}$. 
        
        \item Let $S$ be the set of nonzero squares in $\F_p^\times$, and $g: \F_p^\times \to S$ such that $\forall a \in \F_p^\times, \, g(a) = a^2$. We can see that both $a \pmod{p}$ and $-a \pmod{p} \in \F_p^\times$ map to the same element $a^2 \in S$, because modulo $p$ the following relations hold: 
        \[a * a = a^2 , \qquad -a * -a = a^2 \]

        We prove that $g$ is strictly a $2:1$ mapping by contradiction. Asssume there exist distinct elements $a, b, c \in \F_p^\times$ such that $a^2 = b^2 = c^2 = z$. Then, consider the polynomial $x^2 - z = 0 \in \F_p^\times[x]$.  Since, a polynomial of degree $2$ has at most $2$ roots in $\F_p^\times$, $a,b,c$ can't be distinct. Hence, a contradiction! And $g$ is strictly a $2:1$ mapping. 

        To show that $S \leq \F_p^\times$ is a subgroup we show that the group axioms hold.
        \begin{enumerate}
            \item Identity: $1^2 = 1$, hence $1 \in S$. 
            \item Associativity: Inherited from $\F_p^\times$. 
            \item Closure under multiplication and inverses: 
            
            We know that for any $a, b \in \F_p^\times$ there exist $y = a^2, z = b^2 \in S$. Then because $\F_p^\times$ is cyclic and hence abelian, 
            \[yz = a^2b^2 = (ab)^2 \in S\]

            for some $ab \in \F_p^\times$. 

            Similarly, for any $a, \bar{a} \in \F_p^\times$, where $a\cdot \bar{a} = 1$, by definition there exist $y = a^2, z = \bar{a}^2 \in S$, such that, 
            \[yz = a^2 \bar{a}^2 = (a\bar{a})^2 = 1\]
        \end{enumerate}

        Hence, $S$ is a subgroup. And since $g$ is a $2:1$ mapping, $|\F_p^\times|/|S| = 2$, i.e. $S$ is an index $2$ subgroup of $\F_p^\times$. 
    \end{enumerate}
\end{solution}

\newpage 

\begin{problem}{2} \textbf{Reducibility of $x^4+1$ modulo primes.}
        
    The goal is to prove that $f(x) = x^4 + 1 \in \mathbb{Z}[x]$ is reducible modulo every prime number $p$. You already know (HW\#1) that $f(x)$ is irreducible in $\mathbb{Q}[x]$.
    \begin{enumerate}
        \item Factor $f(x)$ modulo $2$.

        \item Assume that $-1 = u^2$ is a square in $\mathbb{F}_p$. Then use the equality
        \[
        x^4 + 1 = x^4 - u^2
        \]
        to factor $f(x)$ modulo $p$.

        \item Assume that $p$ is odd and $2 = v^2$ is a square in $\mathbb{F}_p$. Then use the equality
        \[
        x^4 + 1 = (x^2 + 1)^2 - (vx)^2
        \]
        to factor $f(x)$ modulo $p$.

        \item Prove that if $p$ is odd and neither $-1$ nor $2$ is a square in $\mathbb{F}_p$, then $-2$ is a square. In this case, factor $f(x)$ modulo any such $p$.  
        \emph{Hint.} For the first part, use the previous problem.

        \item Conclude that $x^4 + 1$ is reducible modulo every prime $p$.
        \end{enumerate}
\end{problem}

\begin{solution}
    \bbni

    \begin{enumerate}
        \item $x^4 + 1 \pmod{2} \equiv (x^2+1)^2 \pmod{2}$
        \item In $\F_p[x]$, $u^2 = -1$, 
        \[x^4 + 1 = x^4 = u^2 = (x^2 + u)(x^2 - u)\]
        \item In $\F_p[x]$, $v^2 = 2$, 
        \[x^4 + 1 = (x^2 + 1)^2 - (vx)^2 = (x^2 + 1 + vx)(x^2 + 1 - vx)\]
        \item From problem 1.3 we know that for finite fields $F$, $F^\times$ is cyclic. Let $g$ be the generator for the field $\F_p^\times$, where $p$ is some odd prime.
        
        \textbf{Claim:} For all even powers $k$, $g^{k}$ must be a square. 

        \textbf{Proof:} Let $k = 2i$ for some $i$. Consider $g^{k} \in \F_p^\times$,
        \[g^k = g^{2i} = g^{i+i} = g^i \cdot g^i\]
        
        It follows that $g^{k}$ is a square in $\F_p^\times$ with square root $g^i$. 

        
        Let $g^i = -1 \pmod{p}, \, g^j = 2 \pmod{p}$ is not a square for some powers $i, j$. We know that $i, j$ are both odd from the claim above; then $i+j$ would be even, and $g^{i+j} = 2 \cdot (-1) = -2$ is a square in $F^\times$.
        
        In $\F_p[x]$, $w^2 = -2$,
        \[x^4 + 1 = (x^2 - 1)^2 - (wx)^2 = (x^2 - 1 + wx)(x^2 - 1 - wx)\]

        \item To see if $f(x)$ is reducible modulo every prime $p$, we check the following 2 cases, 
        \begin{enumerate}
            \item $p = 2$: from subpart (1) we know that $f(x)$ is reducible modulo $2$. 
            \item $p \neq 2$: from subpart (4) we see that for all odd primes, one of $-1$, $2$, or $-2$ must be a square in $\F_p^\times$. And we know $f(x)$ is factorable modulo $p$ in all three cases (subparts (2),(3),(4)). 
        \end{enumerate}
        Therefore, $f(x)$ is reducible modulo every prime $p$. 
    \end{enumerate}
\end{solution}

\newpage 

\begin{problem}{3. Field homomorphisms}

    Let $K$ and $K'$ be field extensions of a field $F$.
    \begin{enumerate}
        \item Prove that any $F$-homomorphism $\varphi : K \to K'$ is injective.

        \item Prove that if $K'/F$ is finite and $\varphi : K \to K'$ is an $F$-homomorphism, then $K/F$ is finite.

        \item Assume that both $K$ and $K'$ are finite over $F$, and that $\varphi : K \to K'$ is an $F$-homomorphism. Prove that $\varphi$ is an $F$-isomorphism if and only if $[K:F] = [K':F]$.

        \item Prove that $f(x) = x^2 - 4x + 2 \in \mathbb{Q}[x]$ is irreducible. Prove that the extensions
        \[
        K = \mathbb{Q}[x]/(f(x)) \quad \text{and} \quad \mathbb{Q}(\sqrt{2})
        \]
        of $\mathbb{Q}$ are $\mathbb{Q}$-isomorphic and exhibit an explicit $\mathbb{Q}$-isomorphism between them.
    \end{enumerate}
\end{problem}

\begin{solution}
    \bbni

    \begin{enumerate}
        \item An $F$-homomorphism $\varphi : K \to K'$ is a homomorphism such that $\varphi(x) = x, \forall x \in F$. Moreover, the kernel of the $F$-homomorphism $\varphi: K \to K'$ is an ideal of $K$. But since $K$ is a field, it only has two ideals $(0)$ and $K$, and because by definition elements in $F \in K$ are mapped to non-zero elements in $K'$, the kernel must be the zero ideal. 
        
        Since the kernel is zero, the $F$-homomorphism is injective.

        \item  Assume $[K':F] < \infty$, and $\varphi : K \to K'$ is an $F$-homomorphism. From subpart (1) we know that $\varphi$ must be injective. Then by rank-nulity, we know that
         \[dim (K) = dim (ker \varphi) + dim (im \varphi) \leq dim(K')\]
         Since, the kernel is trivial, $dim(K) = dim(im \varphi) \leq dim(K')$. Hence, if $K'$ is finite dimensional, $K$ must be too, i.e. $[K : F] < \infty$. 

        \item  Assume $[K:F], \, [K':F] < \infty$, and $\varphi : K \to K'$ is an $F$-homomorphism. 
        
        ($\implies$)

        Assume $\varphi$ is an $F$-isomorphism. An $F$-isomorphism is in particular is a bijective linear map. Which means that $K$ and $K'$ have the same dimension. That is, $[K : F] = [K' : F]$. 

        ($\impliedby)$

        Assume $[K : F] = [K' : F]$, that is $dim (K) = dim(K')$. We already know (from subpart (1)) that any $F$-homomorphism $\varphi : K \to K'$ is injective. Thus $\varphi(K)$ is an $F$-subspace of $K'$ with 
        \[
        \dim( \varphi(K)) = \dim (K) = [K:F].
        \]
        Since $[K:F] = [K':F] = \dim (K')$, it follows that $\varphi(K) = K'$. Therefore $\varphi$ is surjective. Hence $\varphi$ is both injective and surjective, and therefore an $F$-isomorphism.
 
        \item Using the rational root test, we know that $f(x) = x^2 - 4x +2$ has the folowing possible roots $\frac{p}{q} \in \Q$: $p = \{\pm 1, \pm 2\}, \, q = \{\pm 1\}$, that is $\frac{p}{q} = \{\pm 1, \pm 2\}$. Checking if any of these roots satisfy $f(x)$, 
        \begin{align*}
            f(1) &= 1 - 4 +2 = -1 \neq 0\\ 
            f(-1) &= 1 + 4 +2 = 7 \neq 0\\ 
            f(2) &= 4 - 8 +2 = -2 \neq 0\\ 
            f(-2) &= 4 +8  +2 = 14 \neq 0\\ 
        \end{align*}

        None of them do. Since $f(x)$ is quadratic and has no rational roots it is irreducible in $\Q[x]$. 

        $K = \Q[x]/(f(x)) = \Q[x]/(x^2 - 4x +2)$. The degree of the field extension $K$, is equal to the degree of the minimal polynomial. 
        \[[K:\Q[x]] = 2\]  
        That is, the dimension of $K$ as a $\Q$ vector space is $2$. The dimension of $\Q[\sqrt{2}]$ as a $\Q$ vector space is also $2$. That is, $[Q[\sqrt{2}] : \Q] = [K : \Q] = 2$. Define an $F$-homomorphism $\varphi: K \to \Q[\sqrt{2}]$. From subpart (3), we know that any $F$-homomorphism between field extensions of the same degree is a $F$-isomorphism. Hence, the fields are $\Q$-isomorphic. 
        \[K \cong \Q[\sqrt{2}]\]





        
        Let $\alpha \in K$ be a root to $f(x)$. 
        \[ \alpha^2 - 4\alpha + 2 = 0 \implies (\alpha - 2)^2 = 2\]
        Thus, 
        \[ K = \Q(\alpha)\]
        Define $\varphi$ specifically to be, $ \varphi(x) = 2 + \sqrt{2}$. 
        Since
        \[(2 + \sqrt{2})^2 - 4(2 + \sqrt{2}) + 2 = 0\]
        we have $f(2 + \sqrt{2}) = 0$, so $(f(x)) \subset \ker \varphi$.
    \end{enumerate}
\end{solution}

\newpage


\begin{problem}{4} \textbf{Inverses in a cubic extension.}

    Let $\alpha \approx -1.7693$ be the real root of $x^3 - 2x + 2$. In the extension $\mathbb{Q}(\alpha)/\mathbb{Q}$, write the elements $\alpha^{-1}$ and $(\alpha+1)^{-1}$ explicitly as a polynomial in $\alpha$ with coefficients in $\mathbb{Q}$.  

    \emph{Hint.} Remember the algorithm using the Bézout identity (e.g.\ FT pp.\ 16--17).
\end{problem}

\begin{solution}
    \bbni

    \begin{enumerate}
        \item $\alpha^{-1}$
        
        We know that $\alpha^3 - 2\alpha + 2 = 0$. Upon rearranging we get, 
        \begin{align*}
             \alpha^3 - 2\alpha &= -2  \\ 
             \frac{-1}{2}(\alpha^3 - 2\alpha) &= 1 \\ 
             \alpha \left(\frac{-1}{2}(\alpha^2 - 2)\right) &=1
        \end{align*}

        $\implies \alpha^{-1} = \left(\frac{-1}{2}(\alpha^2 - 2)\right)$. 


        \item $(\alpha + 1)^{-1}$ 
        
        From the original equation we get that, 
        \[ \alpha +1 = x^3 -x + 3\]     
        Let $f(x) = x^3 -2x + 2$, $g(x) = x^3 - x + 3$. We can see that $f(x) \nmid g(x)$ and $g(x) \nmid f(x)$. 
        We use Euclid's Extended Algorithm to obtain both the $gcd$ and bezout's coefficients, 
        \begin{align*}
            f(x) &= (1)g(x) - x - 1 \\ 
            g(x) &= (-x^2+x)(-x-1) + 3 \\ 
            (-x-1) &= \left (\frac{-1}{3}(x+1) \right ) 3
        \end{align*}

        Working backwards, 
        \begin{align*}
            3 &= g(x) - (-x^2 + x)(-x-1) \\ 
            &= g(x) -(-x^2 + x)(f(x) - g(x))\\ 
            &= g(x) +x^2 f(x) - xf(x) -x^2g(x) + xg(x) \\ 
            &= f(x)(x^2-x) + g(x)(1-x^2+x) \\ 
            1 &= \frac{1}{3} (x^2 - x) f(x) + \frac{1}{3}(-x^2 +x +1)g(x) \\
            1 &= \frac{1}{3} (x^2 - x) (x^3 - 2x +2) + \frac{1}{3}(-x^2 +x +1)(x^3-x+3) \\ 
        \end{align*}

        Substituting $\alpha$ in we get,   

         \begin{align*}
            1 &= 0 + \frac{1}{3}(-\alpha^2 +\alpha +1)(\alpha^3-\alpha+3) \\ 
            1 &= \frac{1}{3}(-\alpha^2 +\alpha +1)(\alpha + 1)
        \end{align*}

        The inverse of $(\alpha + 1)$ is $\frac{1}{3}(-\alpha^2 +\alpha +1)$. 

    \end{enumerate}
\end{solution}

\newpage

\begin{problem}{5} \textbf{Quadratic extensions.}

    Let $F$ be a field of characteristic $\neq 2$ and let $K/F$ be a field extension of degree $2$.
    \begin{enumerate}
        \item Prove that there exists $\alpha \in K$ with $\alpha^2 \in F$ such that $K = F(\alpha)$. We often write $\alpha = \sqrt{a}$ if $\alpha^2 = a \in F$.  
        \emph{Hint.} Get inspiration from the quadratic formula.

        \item For $a,b \in F^\times$ prove that $F(\sqrt{a}) \cong F(\sqrt{b})$ if and only if $a = u^2 b$ for some $u \in F^\times$.

        \item Deduce that there is a bijection between the set of $F$-isomorphism classes of field extensions $K/F$ with $[K:F] \mid 2$ and the group $F^\times / F^{\times 2}$.

        \item If $F$ is a finite field of characteristic $\neq 2$, prove that $F$ has a unique quadratic extension (up to $F$-isomorphism).
    \end{enumerate}
\end{problem}

\begin{solution}
    \bbni

    \begin{enumerate}
        \item Since $K/F$ is a degree two field extension, the minimal polynomial for this extension is some monic quadratic, say $f(x) = x^2 + bx + c$. It suffices to adjoin the discriminant of the polynomial to get the field extension. $K = F(\sqrt{b^2 -4c})$, where $b, c \in F$ that is, $b^2 -4c \in F$.
        \item  ($\implies$)
        
        If $F(\sqrt{a}) \cong F(\sqrt{b})$, then there exists a $F$-isomorphism $\varphi:F(\sqrt{a}) \to F(\sqrt{b})$. Since $\varphi$ fixes $F$, and the following property holds:

        \[\varphi(a) = \varphi(\sqrt{a}\sqrt{a}) = \varphi(\sqrt{a})\varphi(\sqrt{a}) = a\]

        $\varphi(\sqrt{a})$ is a squareroot of $\varphi(a) \in F(\sqrt{b})$. Every element in $F(\sqrt{b})$ can be written as $x + y \sqrt{b}$ for some $x,y \in F$. Then, 
        \[\varphi(\sqrt{a}) = x + y\sqrt{b} \implies (\varphi(\sqrt{a}))^2 = x^2 + 2xy\sqrt{b} + y^2b \]

        Since, $(\varphi(\sqrt{a}))^2$ is a perfect square in $F(\sqrt{b})$, \, $2xy\sqrt{b}$ must be $0$. Since, $F$ is not characteristic $2$, either $x =0$ or $y=0$. If $y=0$, then 
        \[(\varphi(\sqrt{a}))^2 = x^2 \implies a = x^2\]
        But $a, x \in F$, and $\varphi$ fixes $F$. So this is not possible. $x$ must be zero. 
        \[(\varphi(\sqrt{a}))^2 =  \varphi (a) = y^2b = a\]

        Swapping $y$ with $u$ we get, 
        \[(\varphi(\sqrt{a}))^2 =  \varphi (a) = u^2b = a\]

        Hence, if $F(\sqrt{a}) \cong F(\sqrt{b})$ then $a = u^2b$, for some $u \in F^\times$. 

        ($\impliedby$)

        Assume $a = u^2b$, where $a,b,u \in F^\times$. Since, $F(\sqrt{a}), F(\sqrt{b})$ are degree two extensions, they are $2$-dimensional $F$-vector spaces with bases 
        \[\{1, \sqrt{a}\} \quad \text{and} \quad \{1, u\sqrt{b}\},\]
        respectively.
        We can define an $F$-linear map
        \[\varphi : F(\sqrt{a}) \to F(\sqrt{b})\]
        by sending bases to bases:
        \[\varphi(1) = 1, \qquad \varphi(\sqrt{a}) = u\sqrt{b}\]
        By linearity, for $x,y \in F$,
        \[\varphi(x + y\sqrt{a}) = x + yu\sqrt{b}\]
        Since $\varphi$ sends basis to basis, the two extensions are isomorphic as $F$-vector spaces. It remains to check if $\varphi$ preserves multiplication: 
        \[\varphi(a) = \varphi(\sqrt{a}\sqrt{a}) = \varphi(\sqrt{a})\varphi(\sqrt{a}) = (u\sqrt{b})^2 = u^2 b = a\]
        And since $\varphi$ fixes $F$, $\forall x, y \in F(\sqrt{a})$,
        \[\varphi(xy) = \varphi(x)\varphi(y)\]
        
        It holds! Hence, $\varphi$ is an $F$-algebra isomorphism: \[F(\sqrt{a}) \cong F(\sqrt{b})\]

        \item  For each $a \in F^\times$, let $K_a = F(\sqrt{a})$. Since we are adjoining a square root, the extension $K_a/F$ has degree at most $2$. Thus every field extension $K/F$ with $[K:F] \mid 2$ is either equal to $F$ or of the form $F(\sqrt{a})$ for some $a \in F^\times$. Thus every such extension is represented by some $a \in F^\times$.
        
        From subpart (2), we know that for any $a, b \in F^\times$,  
        \[ F(\sqrt{a}) \cong F(\sqrt{b}) \quad \iff \quad a = u^2 b \text{ for some } u \in F^\times.\]
        That is, $F(\sqrt{a})$ and $F(\sqrt{b})$ are $F$-isomorphic if and only if $a$ and $b$ represent the same element of the quotient group  $F^\times / F^{\times 2}$. Therefore assigning each such extension to it's represtative class in the quotinent group $F^\times/F^{\times 2}$, defines a bijection between the set of $F$-isomorphism classes of field extensions $K/F$ with $[K:F] \mid 2$ and the group $F^\times / F^{\times 2}$

        \item From Problem (1.5) we know that in finite fields, the set of non-zero squares forms an index $2$ subgroup of $F^\times$. That is, $F^\times/F^{\times 2}$ has two cosets. Moreover from subpart (4) we know that there exists a bijection between the set of $F$-isomorphism classes of field extensions $K/F$ with $[K:F] \mid 2$ and the group $F^\times / F^{\times 2}$. The coset corresponding to the squares maps to the trivial extension $F/F$, and the other one maps to $K/F$ where $K = F(\sqrt{a})$ for some $a$ in the coset. Hence, there exists a unique quadratic extension for $F$ upto isomorphism. 
    \end{enumerate}
\end{solution}

\newpage

\begin{problem}{6} \textbf{Minimal polynomials.}

    For each extension $K/F$ and each element $\alpha \in K$, find the minimal polynomial of $\alpha$ over $F$ (and prove that it is the minimal polynomial).
    \begin{enumerate}
        \item $i$ in $\mathbb{C}/\mathbb{R}$
        \item $i$ in $\mathbb{C}/\mathbb{Q}$
        \item $\dfrac{1+\sqrt{5}}{2}$ in $\mathbb{R}/\mathbb{Q}$
        \item $\sqrt{2} + \sqrt{2}$ in $\mathbb{R}/\mathbb{Q}$
    \end{enumerate}
\end{problem}

\begin{solution}
    \bbni
    \begin{enumerate}
        \item $i$ in $\mathbb{C}/\mathbb{R}$
        
        Minimal Polynomial: $f(x) = x^2 + 1$.

        To prove that $f(x)$ is the minimal polynomial, we first check that $i$ satisfies it, then check if the degree of the extension and the degree of the polynomial match, and then check that it is irreducible over $\R$. 

        \begin{enumerate}
            \item $i^2 + 1 = (-1) + 1 = 0$. Hence, $i$ satisfies $f(x)$. 
            \item The basis of $\C$ over $\R$ is $\{1, i\}$, so $[\C:\R] = 2$. And the degree of the polynomial is also $2$. Hence, the degrees match. 
            \item Assume $f(x)$ is reducible over $\R$, then there exist $a, b \in \R$ such that $x^2 +1 = (x-a)(x-b)$. This implies $a,b$ are roots, and $a^2 = -1$. But since, no real numbers have negative squares, this is not possible. Hence, a contradiction! $f(x)$ is irreducible over  $\R$. 
        \end{enumerate}

        Thus, $f(x) = x^2 +1$ is the minimal polynomial of $i$ in $\C/\R$. 

        \item $i$ in $\mathbb{C}/\mathbb{Q}$
        
        Minimal Polynomial: $f(x) = x^2 + 1$.

        Since $\sqrt{-1} \notin \Q$, there are no rational numbers $\frac{a}{b}$ that satisfy $i - \frac{a}{b} = 0$. Hence, the minimal polynomial cannot have degree 1. So, it must be of degree $2$ or more. $f(x)$ is a degree two irreducible polynomial with no rational roots, and hence must be the minimal polynomial. 

        \item $\dfrac{1+\sqrt{5}}{2}$ in $\mathbb{R}/\mathbb{Q}$
        
        Minimal Polynomial: $f(x) = x^2 - x -1$. 

        To prove that $f(x)$ is the minimal polynomial, we first check that $\left(\frac{1 + \sqrt{5}}{2}\right)$ satisfies it, then check if the degree of the minimal extension of $\Q$ such that it contains $\left(\frac{1 + \sqrt{5}}{2}\right)$ 
        
        and the degree of the polynomial match, and then check that it is irreducible over $\Q$. 

        \begin{enumerate}
            \item $\left (\dfrac{1 + \sqrt{5}}{2} \right )^2 - \dfrac{1 + \sqrt{5}}{2} - 1  = 0$. Hence, $\left(\dfrac{1 + \sqrt{5}}{2}\right)$ satisfies $f(x)$. 
            \item The basis of $\Q[\sqrt{5}]$ over $\Q$ is $\{1, \sqrt{5}\}$, so $[\Q[\sqrt{5}]:\Q] = 2$. And the degree of the polynomial is also $2$. Hence, the degrees match. 
            \item  By the rational root test, the only possible roots are $\pm1$. We check to see if either satisfy $f(x)$: 
            \[f(1) = 1 - 1 -1 = -2, \qquad f(-1) = 1 - (-1) -1 = -1\]
            Neither do. So, $f(x)$ is irreducible over $\Q$. 
        \end{enumerate}

        Thus, $f(x) = x^2 - x -1$ is the minimal polynomial of $\left (\dfrac{1 + \sqrt{5}}{2} \right)$ in $\R/\Q$. 

        \item $\sqrt{2} + \sqrt{2}$ in $\mathbb{R}/\mathbb{Q}$    
        
        Minimal Polynomial: $f(x) = x^2 -8$. 

        To prove that $f(x)$ is the minimal polynomial, we first check that $(\sqrt{2} + \sqrt{2})$ satisfies it, then check if the degree of the minimal extension of $\Q$ such that it contains $(\sqrt{2} + \sqrt{2})$         and the degree of the polynomial match, and then check that it is irreducible over $\Q$. 

        \begin{enumerate}
            \item $(\sqrt{2} + \sqrt{2})^2 - 8 = 2 + 2\sqrt{2}\sqrt{2} + 2 - 8 = 0$. Hence, $(\sqrt{2} + \sqrt{2})$ satisfies $f(x)$. 
            \item The basis of $\Q[\sqrt{2}]$ over $\Q$ is $\{1, \sqrt{2}\}$, so $[\Q[\sqrt{2}]:\Q] = 2$. And the degree of the polynomial is also $2$. Hence, the degrees match. 
            \item  By the rational root test, the possible values for roots $\frac{p}{q} \in \Q$ are $p \in \{\pm 1, \pm 2, \pm 4, \pm 8\}$ and $q \in \{\pm 1\}$, i.e., $\frac{p}{q} \in \{\pm 1, \pm 2, \pm 4, \pm 8\}$. We check if any of these are roots to $f(x)$: 
                \[
                \begin{aligned}
                f(1) &= 1^2 - 8 = -7 \neq 0 \\ 
                f(-1) &= 1^2 - 8 = -7 \neq 0 \\
                f(2) &= 2^{2} - 8  = -4 \neq 0, \\
                f(-2) &= (-2)^{2} - 8  = -4 \neq 0, \\
                f(4) &= 4^{2} - 8 = 8 \neq 0, \\
                f(-4) &= (-4)^{2} - 8 = 8 \neq 0, \\
                f(8) &= 8^{2} - 8 = 56 \neq 0.
                \end{aligned}
                \]
            None of them do. So, $f(x)$ is irreducible over $\Q$. 
        \end{enumerate}

        Thus, $f(x) = x^2 -8$ is the minimal polynomial of $(\sqrt{2} + \sqrt{2})$ in $\R/\Q$. 

    \end{enumerate}
\end{solution}

\newpage


\begin{problem}{7} \textbf{Transcendental and algebraic extensions.}

    Let $\pi \in \mathbb{R}$ be the area of a unit circle and let $\alpha = \sqrt{\pi^2 + 2}$. Consider the field $K = \mathbb{Q}(\pi,\alpha)$.

    For the following field extensions, determine whether they are transcendental and/or algebraic and/or finite and/or simple, and if you determine the extension is simple and algebraic, find a simple generator and determine its minimal polynomial.
    \begin{enumerate}
        \item $K/\mathbb{Q}$
        \item $K/\mathbb{Q}(\pi)$
        \item $K/\mathbb{Q}(\alpha)$
        \item $K/\mathbb{Q}(\pi+\alpha)$
    \end{enumerate}
\end{problem}

\begin{solution}
    \bbni

    \begin{enumerate}
        \item $K/\Q$ - Transcendental, Infinite, and Simple. 
        
        Consider the element $\pi + \alpha \in K$. Trivially, $\Q[\pi + \alpha] \subset \Q[\pi, \alpha]$. To show the opposite containment we check, 
        \begin{align*}
            (\pi + \alpha)^2 &= \pi^2 + \alpha^2 + 2\pi \alpha \\ 
            &= \pi^2 + \pi^2 + 2 + 2\pi \left (\sqrt{\pi^2 + 2} \right )  \\ 
            &= 2 \pi \left ( \pi  + \left (\sqrt{\pi^2 + 2} \right )  \right ) + 2 \\ 
            &= 2 \pi \left ( \pi  + \alpha  \right ) + 2 \\
        \end{align*}
        \[\pi = ((\pi +\alpha)^2 - 2) / (2(\pi + \alpha)), \qquad \alpha = (\pi + \alpha)- \pi\]
        
        $\implies Q[\pi + \alpha] \supset \Q[\pi, \alpha] \implies K = \Q(\pi, \alpha) = \Q(\pi + \alpha)$ 


        Hence, $\Q[\pi, \alpha]$ is a simple extension. 
        
        The extension is transcendental over $\Q$ because $\pi$ satisfies no polynomials in $\Q[x]$. And as all powers of $\pi$ are linearly independent over $\Q$, the extension is infinite over $\Q$.  

        \item $K/\Q(\pi)$ - Algebraic, Finite, and Simple. 

        \begin{enumerate}
            \item $K = (\Q(\pi))(\alpha)$ with generator $\alpha$ 
            \item Minimal polynomial: $f(x) = x^2 - \pi^2 -2$
        \end{enumerate}

        Since, the degree of the minimal polynomial is $2$, the extension also has degree $2$ and hence is finite. 

        \item $K/\Q(\alpha)$ - Algebraic, Finite, and Simple
        \begin{enumerate}
            \item $K = (\Q(\alpha))(\pi)$ with generator $\pi$ 
            \item Minimal polynomial: $f(x) = x^2 - \alpha + 2$
        \end{enumerate}

        Since, the degree of the minimal polynomial is $2$, the extension also has degree $2$ and hence is finite. 

        \item $K/\Q(\pi + \alpha)$ - Algebraic, Finite, and Simple
        \begin{enumerate}
            \item $K = (\Q(\pi +\alpha))(67)$ with generator $67$ 
            \item Minimal polynomial: $f(x) = x - 67$
        \end{enumerate}

        Since, the degree of the minimal polynomial is $1$, the extension also has degree $1$ and hence is finite. 
    \end{enumerate}
\end{solution}
\end{document}