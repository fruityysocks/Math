\documentclass[12pt]{article}

\usepackage{fullpage}
\usepackage{mdframed}
\usepackage{colonequals}
\usepackage{algpseudocode}
\usepackage{algorithm}
\usepackage[most, breakable]{tcolorbox}
\usepackage[all]{xy}
\usepackage{proof}
\usepackage{mathtools}
\usepackage{bbm}
\usepackage{amssymb}
\usepackage{amsthm}
\usepackage{amsmath}
\usepackage{amsxtra}
\usepackage{enumitem}
\newcommand{\bb}{\mathbb}


\newtheorem{theorem}{Theorem}[section]
\newtheorem{theorem*}{Theorem}
\newtheorem{definition}[theorem]{Definition}
\newtheorem{corollary}{Corollary}[theorem]
\newtheorem{lemma}[theorem]{Lemma}
\newtheorem{prop}[theorem]{Proposition}
\newtheorem{remark}[theorem]{Remark}


\newtheorem*{exercisehelper}{Exercise.}
\newenvironment{exercise}[1]{%
  \IfBlankTF{#1}
    {\renewcommand{\exercisehelper}{\textbf{Exercise} \unskip}}
    {\renewcommand\exercisehelper{\textbf{Exercise #1}}}%
  \exercisehelper
}{\endexercisehelper}

\theoremstyle{remark}
\newtheorem*{solution}{Solution}
\newcommand{\mathcat}[1]{\textup{\textbf{\textsf{#1}}}} % for defined terms

\newenvironment{problem}[1]
{ \begin{tcolorbox}[breakable]\noindent\textbf{Problem #1}.}
{\vskip 6pt \end{tcolorbox}}

\newenvironment{enumalph}
{\begin{enumerate}\renewcommand{\labelenumi}{\textnormal{(\alph{enumi})}}}
{\end{enumerate}}

\newenvironment{enumroman}
{\begin{enumerate}\renewcommand{\labelenumi}{\textnormal{(\roman{enumi})}}}
{\end{enumerate}}

\newcommand{\defi}[1]{\textsf{#1}} % for defined terms



\setlength{\hfuzz}{4pt}

\let\H\relax
\let\P\relax
\newcommand{\H}{\mathbb H}
\newcommand{\P}{\mathbb P}
\newcommand{\C}{\mathbb C}
\newcommand{\N}{\mathbb N}
\newcommand{\Q}{\mathbb Q}
\newcommand{\R}{\mathbb R}
\newcommand{\Z}{\mathbb Z}
\newcommand{\F}{\mathbb F}
\newcommand{\br}{\mathbf{r}}
\newcommand{\RP}{\mathbb{RP}}
\newcommand{\CP}{\mathbb{CP}}
\newcommand{\nbit}[1]{\{0, 1\}^{#1}}
\newcommand{\bits}{\{0, 1\}^{n}}
\newcommand{\bbni}{\bigbreak \noindent}
\newcommand{\norm}[1]{\left\vert\left\vert#1\right\vert\right\vert}
\newcommand{\dbar}{\overline{\partial}}
\let\d\relax
\newcommand{\d}{\partial}
\newcommand{\calO}{\mathcal{O}}
\newcommand{\calF}{\mathcal{F}}
\newcommand{\calG}{\mathcal{G}}
\newcommand{\calH}{\mathcal{H}}
\newcommand{\calE}{\mathcal{E}}
\newcommand{\calC}{\mathcal{C}}
\newcommand{\calD}{\mathcal{D}}

\let\1\relax
\newcommand{\1}{\mathbf{1}}
\newcommand{\fr}[2]{\left(\frac{#1}{#2}\right)}
\newcommand{\todo}[1]{\textcolor{red}{\textbf{TODO:} #1}}
\newcommand{\vecz}{\mathbf{z}}
\newcommand{\vecr}{\mathbf{r}}
\DeclareMathOperator{\Cinf}{C^{\infty}}
\DeclareMathOperator{\Id}{Id}
\DeclareMathOperator{\Ell}{Ell}
\DeclareMathOperator{\CL}{\mathcal{CL}}

\DeclareMathOperator{\Alt}{Alt}
\DeclareMathOperator{\Aut}{Aut}
\DeclareMathOperator{\ann}{ann}
\DeclareMathOperator{\codim}{codim}
\DeclareMathOperator{\End}{End}
\DeclareMathOperator{\Hom}{Hom}
\DeclareMathOperator{\id}{id}
\DeclareMathOperator{\M}{M}
\DeclareMathOperator{\Mat}{Mat}
\DeclareMathOperator{\Ob}{Ob}
\DeclareMathOperator{\opchar}{char}
\DeclareMathOperator{\opspan}{span}
\DeclareMathOperator{\rk}{rk}
\DeclareMathOperator{\sgn}{sgn}
\DeclareMathOperator{\Sym}{Sym}
\DeclareMathOperator{\tr}{tr}
\DeclareMathOperator{\img}{img}
\DeclareMathOperator{\coker}{coker}
\DeclareMathOperator{\Spec}{Spec}
\DeclareMathOperator{\CandE}{CandE}
\DeclareMathOperator{\CandO}{CandO}
\DeclareMathOperator{\argmax}{argmax}
\DeclareMathOperator{\first}{first}
\DeclareMathOperator{\last}{last}
\DeclareMathOperator{\cost}{cost}
\DeclareMathOperator{\dist}{dist}
\DeclareMathOperator{\path}{path}
\DeclareMathOperator{\parent}{parent}
\DeclareMathOperator{\argmin}{argmin}
\DeclareMathOperator{\excess}{excess}
\let\Pr\relax
\DeclareMathOperator{\Pr}{\mathbf{Pr}}
\DeclareMathOperator{\Exp}{\mathbb{E}}
\DeclareMathOperator{\Var}{\mathbf{Var}}
\let\limsup\relax
\DeclareMathOperator{\limsup}{limsup}
%Paired Delims
\DeclarePairedDelimiter\ceil{\lceil}{\rceil}
\let\oldceil\ceil
\renewcommand{\ceil}[1]{\oldceil*{#1}}

\DeclarePairedDelimiter{\floor}{\lfloor}{\rfloor}
\let\oldfloor\floor
\renewcommand{\floor}[1]{\oldfloor*{#1}}





\newcommand{\dagstar}{*}

\newcommand{\tbigwedge}{{\textstyle{\bigwedge}}}
\setlength{\parindent}{0pt}
\setlength{\parskip}{5pt}


\usepackage{listings}
\usepackage{courier}
\usepackage{microtype}


\lstset{
  basicstyle=\footnotesize\ttfamily,
  breaklines=true,
  breakatwhitespace=true
  columns=fullflexible,
  keepspaces=true,
  frame=single,
  escapeinside={(*@}{@*)}
}

\begin{document}

\title{Math 71: Abstract Algebra}

\author{Prishita Dharampal}
\date{}
\maketitle


\textbf{Credit Statement:} Talked to Sair Shaikh'26, and Math Stack Exchange.


\begin{problem}{1}
    For $f(x)=x^4-1$ and $g(x) = 3x^2+3x$ find: the quotient and remainder after dividing $f$~by~$g$; the gcd of $f$ and $g$; and the expression of this gcd in the form $af+bg$ for some $a,b \in \Q[x]$.  For the last two, you'll need to recall the Euclidean Algorithm and the Bezout Identity.
\end{problem}


\begin{solution}
    \bbni 

    Quotinent: $\frac{1}{3} (x^2 - x +1)$ \\ 
    Remainder: $ -x - 1$ \\ 
    Using Euclid's Algorithm: 
    \begin{align*}
        x^4 -1 &= (3x^2 + 3x)(\frac{1}{3} (x^2 - x +1)) + ( -x -1) \\ 
        (3x^2 + 3x) &= (- x - 1)(-3x) + 0
    \end{align*}
    $gcd(x^4-1, 3x^2 + 3x) = -x - 1$ 

    Using Bezout's Idenity: 
    \begin{align*}
        (x^4-1, 3x^2 + 3x) &= af + bg \\
        -x - 1 &= f - (\frac{1}{3} (x^2 - x +1))g \\ 
        -x -1 &=  1(x^4 -1) + (- (\frac{1}{3} (x^2 - x +1)))(3x^2 + 3x) 
    \end{align*}

    $a = 1, b = - (\frac{1}{3} (x^2 - x +1))$
\end{solution}


\newpage


\begin{problem}{2}
    Prove that two polynomials $f, g \in \Z[x]$ are relatively prime in $\Q[x]$ (i.e., they share no common nonconstant factor) if and only if the ideal $(f,g)\subset \Z[x]$ contains a nonzero integer.
\end{problem}

\begin{solution}
    \bbni 

    $(\implies)$

    Assume the polynomials $f,g$ are relatively prime in $\Q[x]$. \\
    I.e. $(f,g) = (gcd(f,g)) = (1) = \Q[x]$. Since we are in a euclidean domain, 
    \[1 = af + bg\] 
    for some $a, b$ with rational coefficients. Let $k$ be the product of the denominators of the coefficients of the terms in $a, b$. Then 
    \[k = kaf + kbg\] 
    has integer coefficients. I.e. $kaf, \, kbg \in \Z[x]$, and since $k$ can be expressed as a linear combination of $f$ and  $g$, $k \in (f,g) \subset \Z[x]$. 
    Hence, the ideal $(f,g)\subset \Z[x]$ contains a nonzero integer.

    $(\impliedby)$

    Assume the ideal $(f, g) \subset \Z[x]$ contains a non-zero integer $k$. \\
    Since this ideal is a subset of the ideal generated by $f, g$ in $\Q[x]$, $k \in (f, g) \subset \Q[x]$. But all integers are units in $\Q[x] \implies 1 \in (f, g) \subset \Q[x]$. I.e. for some polynomials $a, b \in \Q[x]$, 
    \[1 = af + bg\]
    Hence, the polynomials $f, g$ are relatively prime in $\Q[x]$. 
\end{solution}

\newpage 

\begin{problem}{3}
    Decide whether each of the following polynomials is irreducible, and
    if not, then find the factorization into monic irreducibles.
    \begin{enumerate}\renewcommand{\itemsep}{3mm}
    \item $x^4+1 \in \R[x]$

    \item $x^4+1 \in \Q[x]$

    \item $x^7 + 66x^6 - 77x + 737 \in \Q[x]$

    \item $x^4+x^3+x^2+x+1 \in \Q[x]$

    \item $x^3+5x^2-9x+3 \in \Q[x]$
    \end{enumerate}
\end{problem}

\begin{solution}
    \bbni 
    \begin{enumerate}
        \item $x^4+1 \in \R[x]$
        \[(x^2 + \sqrt{2}x + 1)(x^2 - \sqrt{2}x + 1)\]
        \item $x^4+1 \in \Q[x]$
        Let $f(x) = x^4 + 1$. Then,
        \[f(y+1) = (y+1)^4 + 1 = y^4 + 4y^3 + 6y^2 + 4y + 2\]
         We can see that $2 | 4, 2 | 6, 2 | 2$, and $4 \nmid 2$. Then by Eisentein's Criterion, the polynomials of the form $f(x)$ are irreducible in $\Q[x]$. 

        \item $x^7 + 66x^6 - 77x + 737 \in \Q[x]$

        We can see that $11 | 66, \, 11 | -77, \, 11 |  737$, and $121 \nmid 737$. Then by Eisentein's Criterion, the polynomial is irreducible in $\Q[x]$.

        \item $x^4+x^3+x^2+x+1 \in \Q[x]$
        
        Let $f(x) = x^4 + x^3 + x^2 + 1$. Then,
        \[f(y+1) = (y+1)^4 + (y + 1)^3 + (y+1)^2 + 1 = y^4 + 5y^3 + 10y^2 + 10y + 5\]
         We can see that $5 | 5, 5 | 10$, and $25 \nmid 5$. Then by Eisentein's Criterion, the polynomials of the form $f(x)$ are irreducible in $\Q[x]$. 
    

        \item $x^3+5x^2-9x+3 \in \Q[x]$

        Assume $\frac{r}{s}$ is a root of the polynomial in the lowest terms. From proposition 11 we know that $r \mid a_n$ and $s \mid a_0$. I.e. $r \mid 1, \, s \mid 3$. The only such candidate is $\frac{1}{3}$.
        Checking, 
        \[ \left (\frac{1}{3} \right )^3 + 5 \left (\frac{1}{3} \right )^2 - 9 \left (\frac{1}{3} \right ) +3 = \frac{16}{27} \]
        Hence, $\frac{16}{27}$ is not a root of the polynomial. By proposition 10, we know that this polynomial (degree 3) is irreducible in $\Q[x]$ (over a field). 
    \end{enumerate}
\end{solution}




\begin{problem}{4}
    \textit{Irreducible polynomials over finite fields.}
    Let $\F_3$ be the field with three elements.
    \begin{enumerate}\renewcommand{\itemsep}{3mm}
    \item Determine all the monic irreducible polynomials of degree $\leq
    3$ in $\F_3[x]$.

    \item Determine the number of monic irreducible polynomials of degree
    $4$ in $\F_3[x]$. \\ 
    \textbf{Hint.} This is easier than determining all of them.
    \end{enumerate}
\end{problem}

\begin{solution}
    \bbni 

    \begin{enumerate}
        \item \begin{enumerate}
            \item Linear Irreducible Polynomials
            
            By definition all monic linear polynomials are irreducible. 
            \begin{align*}
                x  &= 0 \\ 
                x + 1 &= 0 \\ 
                x + 2 &= 0 \\ 
            \end{align*}
            \item Quadratic Irreducible Polynomials
            
            All quadratic polynomials are of the form $x^2 + ax + b = 0 $, where $a,b \in \F_3$. There are 9 such polynomials. By Propopsition 10, we know that polynomials of degree two over a field is reducible if and only if it has a root in the field.         

            Upon checking, we are left with: 
            \begin{align*}
                x^2 + 1 &= 0 \\
                x^2 + 1x + 2 &= 0 \\
                x^2 + 2x + 2 &= 0 \\
            \end{align*}

            \item Cubic Irreducible Polynomials
            All cubic polynomials are of the form $x^3+ ax^2 + bx + c = 0 $, where $a,b,c \in \F_3$. There are 27 such polynomials. By Propopsition 10, we know that polynomials of degree three over a field is reducible if and only if it has a root in the field.            

            Upon checking, we are left with: 
            \begin{align*}
                x^3 + 2x + 1 &= 0 \\
                x^3 + 2x + 2 &= 0 \\
                x^3 + 1x^2  + 2 &= 0 \\
                x^3 + 1x^2 + 2x + 1 &= 0 \\
                x^3 + 1x^2 + 1x + 2 &= 0 \\
                x^3 + 2x^2  + 1 &= 0 \\
                x^3 + 2x^2 + 1x + 1 &= 0 \\
                x^3 + 2x^2 + 2x + 2 &= 0 \\
            \end{align*}
        \end{enumerate}

        \item Quartic Irreducible Polynomials
        
        All cubic polynomials are of the form $x^4+ ax^3 + bx^2 + cx + d = 0 $, where $a,b,c, d \in \F_3$. There are 81 such polynomials. To the irreducibles we first count the reducibles. The reducibles can be classified by the degrees of their factors that is partitions of $4$. 
        \begin{itemize}
            \item 3 + 1 
            
            There are 8 irreducible cubics and 3 irreducible linear polynomials, the number of quartics factored as such are: 
            $8 \cdot 3 = 24$. 
            \item 2 + 2 
            
            There are 3 irreducible quadratics, the number of quartics factored as such are: $3! = 6$ 

            \item 2 + 1 + 1 
            
            There are 3 irreducible quadratics, and 3 irreducible linear polynomials, the number of quartics factored as such are: $3 \cdot (3!) = 18$.  

            \item 1 + 1 + 1 + 1
            
            There are $3$ irreducible linear polynomials and $4$ places to fill, so by stars and bars the number of quartics that can be factored as such are: $\begin{pmatrix}
                6 \\ 2
            \end{pmatrix} = 15$. 
        \end{itemize}

        Then the number of irreducible quartics is \[81 - 24 - 6 - 18- 15 = 18.\]
    \end{enumerate}
\end{solution}

\newpage

\begin{problem}{5(a)}\textit{ Symmetric polynomials.} Let $R$ be a commutative ring with 1 and $R[x_1,\dotsc,x_n]$ the ring
of polynomials in the variables $x_1,\dotsc,x_n$ with coefficients in
$R$.  Consider the symmetric group $S_n$ acting on the set
$\{x_1,\dotsc,x_n\}$ by permutations.  Extend this action linearly to
$R[x_1,x_2,\dotsc,x_n]$; for example, if $\sigma = (123) \in S_3$, then
$$
\sigma\cdot (x_1x_2 - 6x_3^2 + 7x_2x_3^2) = x_2x_3 - 6x_1^2 + 7x_3x_1^2.
$$  
Then this action satisfies $\sigma\cdot(f+g)=\sigma\cdot f +
\sigma\cdot g$ and $\sigma\cdot(fg) = (\sigma\cdot f)(\sigma\cdot g)$
for all $\sigma \in S_n$ and all $f,g \in
R[x_1,\dotsc,x_n]$.

Let $S \subset R[x_1,\dotsc,x_n]$ be the subset fixed under the
action of $S_n$.  Prove that $S$ is a subring with 1.  This is called
the \textbf{ring of symmetric polynomials}.

\end{problem}

\begin{solution}
    \bbni 

    To prove that $S$ is a subring with $1$: 
    \begin{enumerate}
        \item Contains 1

        Since $1$ is a constant polynomial and $S_n$ is acting on the set of variables $\{x_1,\dotsc,x_n\}$, 
        \[\sigma(1) = 1\]
        Hence, $1 \in S$. 

        \item Closed under multiplication
        
        $\forall f, g \in S$ by definition, $\sigma(f) = f, \sigma(g)=g$ and again by definition, 
        \[\sigma(fg) = \sigma(f) \cdot \sigma(g) = f\cdot g\]
        Hence, $f\cdot g \in S$. Since $f, g$ were arbitrary polynomials this holds for any elements in $S$ and $S$ is closed under multiplication.
    \end{enumerate}
\end{solution}

\newpage

\begin{problem}{5(b)} For each $n \geq 0$, define polynomials $e_i \in
R[x_1,\dotsc,x_n]$ by  $e_0=1$ and
$$
e_1  {}= x_1+\dotsm+x_n,
\quad
e_2  {}= \sum_{1 \leq i < j \leq n}x_ix_j,
\quad 
\dotsc, {} \quad
e_n  {}= x_1\dotsm x_n
$$
and $e_k=0$ for $k > n$. In words, $e_k$ is the sum of all distinct
products of subsets of $k$ distinct variables.  Prove that each $e_k$
is a symmetric polynomial.  These are called the \textbf{elementary
symmetric polynomials}.
\end{problem}

\begin{solution}
    \bbni 

    For a given $k$, $1 \leq k \leq n$, 
    \[e_k = \sum_{1 \leq i_1 <  i_2 < \cdots < i_k \leq n}x_{i_1} x_{i_2}\cdots x_{i_k}\]
    
    Let $A$ be the set of all terms in $e_k$.
    \[A = \{x_{i_1} x_{i_2}\cdots x_{i_k} \mid 1 \leq i_1 <  i_2 < \cdots < i_k \leq n \} \]

    Since $\sigma$ as a n-cycle is a bijection on $\{i_1, i_2, \cdots, i_n\}$,  $\forall a \in A, \, \sigma(a)$ is also a product of $k$ distinct variables. And by definition of $A$ all distinct multiples of subsets of $k$ distinct variables, $\sigma(a) \in A$. Hence, $\sigma: A \to A$. 
    
    Also, for any $a \in A$ we can find $b \in A$ such that $\sigma^{-1}(b) = a$. Hence $\sigma$ is surjective. A surjective mapping from $A$ to $A$ is bijective. 

    Hence, $\sigma(e_k)$ only permutes the terms of $e_k$. 
    \[\sigma(e_k) = \sum_{a\in A} \sigma(a) = \sum_{a\in A} a = e_k \]
    $e_k$ is invariant under the action of $\sigma$, and hence is a symmetric polynomial. 

\end{solution}

\newpage


\begin{problem}{5(c)} 
    The \textbf{generic polynomial} of degree $n$ is the polynomial
    $$
    f(x) = (x-x_1)(x-x_2)\dotsm (x-x_n)
    $$
    in the ring $R[x_1,\dotsc,x_n][x]$ of polynomials in $x$ with
    coefficients in $R[x_1,\dotsc,x_n]$.  Prove (by induction) that

    \begin{align*}
        f(x) &= (x-x_1)(x-x_2)\dotsm (x-x_n)=x^n - e_1 x^{n-1} + e_2 x^{n-2} + \dotsm
    + (-1)^ne_n  \\
    &= \sum_{j=0}^n (-1)^{n-j} e_{n-j} x^j.
    \end{align*}

\end{problem}

\begin{solution}
    \bbni

    \textbf{Base case:} $n = 1, R[x_1][x]$ 

    Then, by definition, 
    \[f(x) = x - x_1\]
    
    In $R[x_1][x], \, e_1$ is the sum of all distinct products of subsets of 1 distinct variables. I.e. $e_1 = x_1$. Upon substitution. 
    \begin{align*}
        f(x) &= x- x_1 \\ 
        &= x - e_1 \\ 
        &= x + (-1)^1 e_1 \\ 
        &= (-1)^0e_{0}x + (-1)^1 e_1x^0 \\ 
        &= (-1)^1 e_1x^0 + (-1)^0e_{0}x\\ 
        &= \sum_{j=0}^{1}(-1)^{1-j}e_{1-j}x^j \\ 
    \end{align*}

    Hence, base case holds. 

    \textbf{Notation:} $e_{a,k}$, where $a$ refers to the elementary symmetric polynomials in $R[x_1, \cdots, x_{a}][x]$, or in a ring with $a$ adjoined variables, and $k$ refers to the number of variables in the subset. For example, in $R[x_1 \cdots x_n][x]$, the elementary symmetric polynomial with $j$ elements in the subset as $e_{n,j}$. 


    \textbf{Inductive Hypothesis:} Assume the $n-1$ the case holds. I.e. in $R[x_1, \cdots x_{n-1}][x]$, 
    \begin{align*}
        f(x) &= (x-x_1)(x-x_2)\dotsm (x-x_{n-1})=x^{n-1} - e_1 x^{n-2} + e_2 x^{n-3} + \dotsm + (-1)^{n-1}e_{n-1}  \\
        &= \sum_{j=0}^{n-1} (-1)^{n-1-j} e_{(n-1), (n-1-j)} x^j
    \end{align*}

    \textbf{Inductive Step:} $n=n, R[x_1, \cdots, x_n][x]$ 

    Then, by definition, 
    \[f'(x) = (x-x_1)(x-x_2) \cdots (x-x_n)\]

    Upon substitution, 
    \begin{align*}
        f'(x) &= (x-x_1)(x-x_2)\cdots (x-x_n) \\
        f'(x) &= f(x)(x-x_n) \qquad \text{ where $f(x)$ is from IH}\\ 
        &= \left (\sum_{j=0}^{n-1} (-1)^{n-1-j} e_{(n-1), (n-1-j)} x^j \right )(x-x_n) \\ 
        &= x \left (\sum_{j=0}^{n-1} (-1)^{n-1-j} e_{(n-1), (n-1-j)} x^j \right )- x_n \left (\sum_{j=0}^{n-1} (-1)^{n-1-j} e_{(n-1), (n-1-j)} x^j \right ) \\ 
        \text{(Reindexing j)}\\
        &= \left (x^n + \sum_{j=0}^{n-1} (-1)^{n-j} e_{(n-1), (n-j)} x^{j} \right )-  \left (\sum_{j=0}^{n-1} (-1)^{n-1-j} x_n e_{(n-1), (n-1-j)} x^j \right ) \\ 
        &=  x^n + \sum_{j=0}^{n-1}  x^j \left ((-1)^{n-j} e_{(n-1), (n-j)} + (-1)^{n-j} x_n e_{(n-1), (n-1-j)}\right ) \\
        &= \left ( x^n +  \sum_{j=0}^{n-1}x^j (-1)^{n-j} 
        \left (
        e_{(n-1), (n-j)} +  x_n e_{(n-1), (n-1-j)}
        \right )
        \right )
    \end{align*}

    $x_n e_{(n-1), (n-1-j)}$ represents all the elements in $e_{n, (n-1-j)}$ that contain $x_n$ as $e_{(n-1), (n-1-j)}$ contains all combinations of terms from $x_1, \cdots, x_{n-1}$. And $e_{(n-1), (n-1-j)}$ all elements in $e_{n, (n-1-j)}$ that don't. So, their sum equals $e_{(n), (n-j)}$. Overall this gives, 
    \begin{align*}
        f'(x) 
        &= \left ( x^n +  \sum_{j=0}^{n-1}x^j (-1)^{n-j} 
        e_{(n), (n-j)} 
        \right )  \\
        &= \left ( \sum_{j=0}^{n}x^j (-1)^{n-j} 
        e_{(n), (n-j)} 
        \right ) 
    \end{align*}
    where the last equation follows as $(-1)^{n-n} e_{(n),{n-n}} = 1$. 
\end{solution}


\newpage

\begin{problem}{5(d)} For each $k \geq 1$, define the \textbf{power sums} $p_k =
x_1^k + \dotsm + x_n^k$ in $R[x_1,\dotsc,x_n]$.  Clearly, the power
sums are symmetric.  Verify the following identities by hand:
$$
p_1=e_1, \quad p_2 = e_1p_1 - 2e_2, \quad p_3 = e_1p_2 - e_2p_1 + 3e_3
$$
In general \textbf{Newton's identities} in $R[x_1,\dotsc,x_n]$ are
(recall that $e_k=0$ for $k>n$):
$$
p_k - e_1 p_{k-1} + e_2 p_{k-2} - \dotsm + (-1)^{k-1}e_{k-1}p_1 +
(-1)^k k e_k = 0.
$$
Prove Newton's identities whenever $k \geq n$.\\

    \textbf{Hint.} For each $i$, consider the equation in part (c) for $f(x_i)$ and
sum all these equations together.  This gives Newton's identity for
$k=n$.  Set extra variables to zero to get the identities for $k > n$
from this.  (Fun. Can you come up with a proof when $1 \leq k \leq n$?)
\end{problem}

\begin{solution}
    \bbni
    \begin{enumerate}
        \item $p_1 = e_1$
        
       \[p_1 = x^1_1 + \cdots x^1_n = e_1 \]

        \item $p_2 = e_1p_1 -2e_2$ 
        
        \begin{align*}
            e_1p_1 -2e_2 &= e_1^2 - 2e_2 \\ 
            &=  (x_1 + \cdots + x_n)^2 - 2 \left (\sum_{1\leq i < j \leq n}x_ix_j \right ) \\ 
            &= \left (x_1^2 + \cdots + x_n^2 + 2 \left (\sum_{1\leq i < j \leq n}x_ix_j \right ) \right ) - \left (\sum_{1\leq i < j \leq n}x_ix_j \right ) \\ 
            &= x_1^2 + \cdots + x_n^2 \\ 
            &= p_2
        \end{align*}

        \item $p_3 = e_1p_2 - e_2p_1 + 3e_3$ 
        
        \begin{align*}
            e_1p_2 &= \sum_{1 \leq i, j \leq n} x_ix_j^2  \\
            &=  \sum_{1 \leq i, j \leq n, i = j} x_i^3  + \sum_{1 \leq i < j \leq n} x_ix_j^2 + \sum_{1 \leq j < i \leq n} x_ix_j^2 \\ 
            &=  \sum_{1 \leq i, j \leq n, i = j} x_i^3  + \sum_{1 \leq i < j \leq n} x_ix_j^2 + \sum_{1 \leq i < j \leq n} x_i^2x_j \\ 
            e_2p_1 &= \sum_{\substack{1 \leq i < j\leq n \\ 1 \leq k \leq n}} x_ix_jx_k \\
            &\text{(Using cases: $j<i<k, \, i < k < j, \, i < k < j, \, k = i, \, k = j$) } \\
            &= \sum_{1 \leq k < i < j \leq n} x_ix_jx_k + \sum_{1 \leq i < k < j \leq n} x_ix_jx_k + \sum_{1 \leq  i < j < k \leq n} x_ix_jx_k + \sum_{1 \leq i < j \leq n} x_i^2x_j + \sum_{1 \leq  i < j \leq n} x_ix_j^2\\
            &\text{(After re-indexing, we get:) } \\
            &= 3\sum_{1 \leq  i < j < k \leq n} x_ix_jx_k + \sum_{1 \leq i < j \leq n} x_i^2x_j + \sum_{1 \leq  i < j \leq n} x_ix_j^2\\
            e_3 &= \sum_{1 \leq i < j < k \leq n}x_ix_jx_k \\ 
        \end{align*}

        Substituting values in, 
        \begin{align*}
            e_1p_2 - e_2p_1 + 3e_2 &=  \sum_{1 \leq i, j \leq n, i = j} x_i^3  + \sum_{1 \leq i < j \leq n} x_ix_j^2 + \sum_{1 \leq i < j \leq n} x_i^2x_j \\ 
            &- \left ( 3\sum_{1 \leq  i < j < k \leq n} x_ix_jx_k + \sum_{1 \leq i < j \leq n} x_i^2x_j + \sum_{1 \leq  i < j \leq n} x_ix_j^2 \right ) + 3 \sum_{1 \leq i < j < k \leq n}x_ix_jx_k  \\ 
            &= \sum_{1 \leq i, j \leq n, i = j} x_i^3  \\ 
            & = p_3
        \end{align*}

        \item Newton's identities for $k \geq n$. 
        
        Let $f = (x-x_1)\cdots(x-x_n)$. For each $x_i$, $(x - x_i)$ is a factor of $f$. Thus, $f(x_i) =0$ for all $i$. Summing these from $1$ to $n$ and using part (c) we get, 
        \begin{align*}
            0 &= \sum_{i =1}^n f(x_i) \\  
            &=\sum_{i = 1}^{n}\left (\sum_{j=0}^n (-1)^{n-j} e_{n-j} x^j \right ) \\
            &= \sum_{i = 1}^{n} \left( x_i^n - e_1x^{n-1}_i + e_2x^{n-1} + \cdots + e_{n-1}x (-1)^ne_n \right) \\ 
            &= \sum_{i = 1}^{n} x_i^n -  \sum_{i = 1}^{n} e_1x^{n-1}_i + \sum_{i = 1}^{n} e_2 x^{n-2}_i + \cdots + (-1)^{n-1}\sum_{i=1}^n e_{n-1} x_i  + (-1)^ne_n \\ 
            &= \sum_{i = 1}^{n} x_i^n -  e_1 \sum_{i = 1}^{n} x^{n-1}_i + e_2 \sum_{i = 1}^{n} x^{n-2}_i + \cdots + (-1)^{n-1} e_{n-1} \sum_{i=1}^n x_i + (-1)^ne_n \\ 
            &= p_n -  e_1 p_{n-1} + e_2 p_{n-2} + \cdots + (-1)^{n-1} e_{n-1}p_{1} +  (-1)^ne_n \\ 
        \end{align*}
        

        Consider the ring $R[x_1, \cdots, x_n, \cdots x_k]$. Here the equation, 
            \[p_k -  e_1 p_{k-1} + e_2 p_{k-2} + \cdots + (-1)^{k-n}e_np_{k-n}+ \cdots + (-1)^{k-1} e_{k-1}p_{1} +  (-1)^ke_k = 0\]
        holds. Since, $\forall i > n, e_{i} =0$, 
        
        \[p_k -  e_1 p_{k-1} + e_2 p_{k-2} + \cdots + (-1)^{k-n}e_np_{k-n}= 0\]

    \end{enumerate}
\end{solution}

\newpage


\begin{problem}{6}\textit{Use the force, my Newton!}
\begin{enumerate}\renewcommand{\itemsep}{3mm}
\item If $x,y,z$ are complex numbers satisfying
$$
x+y+z=1, \qquad x^2+y^2+z^2=6, \qquad x^3+y^3+z^3=7,
$$
then prove that $x^n + y^n + z^n$ is rational for any positive integer
$n$.

\item Calculate $x^4+y^4+z^4$.

\item Prove that each of $x,y,z$ are not rational numbers.  
\end{enumerate}  
\end{problem}



\begin{solution}
    \bbni
    \begin{enumerate}
        \item  Base Case: for $n = 1, 2, 3$ we know that $x^n + y^n + z^n$ is rational. 
        
        Induction Hypothesis: Assume $x^k + y^k + z^k$ is rational, $\forall k <n$. 
        
        Inductive Step: For $n$, as $n > 3$, the following holds (from 5(d)), also $e_i = 0, \forall i >3$:  

        \[p_n -  e_1 p_{n-1} + e_2 p_{n-2}  - e_3p_{n-3} = 0\]
        
        And, 
        \[p_n =  e_1 p_{n-1} - e_2 p_{n-2}  + e_3p_{n-3}\]


        In $R[x, y, z]$, $p_n := x^n + y^n + z^n$. 
        
        Note: $e_1 = p_1, e_2 = (e_1p_1 - p_2)/2, e_3 = (p_3 - e_1p_2 + e_2p_1)/3, e_4 = 0$ as $4 > 3$. 

        So, $e_1 = 1, e_2 = -5/2, e_3 = -1/2, e_4 = 0$, which are all rational. And since, $p_{n-1}, p_{n-2}, p_{n-3}$ are also rational (by IH) we have a sum of products of rationals, which is rational as $\Q$ is a field.  

        \item In $R[x, y, z]$, $p_n := x^n + y^n + z^n$
        \begin{align*}
            0  &= p_4 -  e_1 p_{3} + e_2 p_{2} + (-1)^{3} e_{3}p_{1} + (-1)^4e_4 \\ 
            &= p_4 -  e_1(7) + e_2 (6) - e_{3}(1) + e_4 \\ 
            p_4 &= e_17 - e_2 (6) + e_{3}(1) - e_4 
        \end{align*}

        Note: $e_1 = p_1, e_2 = (e_1p_1 - p_2)/2, e_3 = (p_3 - e_1p_2 + e_2p_1)/3, e_4 = 0$ as $4 > 3$. 

        So, we get $e_1 = 1, e_2 = -5/2, e_3 = -1/2, e_4 = 0$. 

        Substituting the values: 
        \begin{align*}
            p_4 &= 7 - (-5/2)(6) + (-1/2) - 0 \\ 
            &= 7 + 15 - 1/2 \\ 
            &= 21.5 = 43/2
        \end{align*}

        \item Let $f(a) = (a - x)(a-y)(a-z)$ in $\Q[n]$. 
        \begin{align*}
            f(a) &= a^3 - a^2 (x+y+z) + a (xy+yz +zx) - xyz \\
            &= a^3 - a^2(e_1) + a (e_2) - e_3 \\
            &= a^3 - a^2 -(5/2)a + 1/2
        \end{align*}
        If $2f(a)$ has a root, then $f(a)$ also has the same root. Rationalizing denominators, 
        \[2f(a)  = 2a^3 - 5a +1\]
        By rational root test, any root of $2f(a)$, say $\left(\frac{p}{q}\right)$ must be such that $p | 2$ and $q|1$. Since two is prime, there is only one root we need to check, namely $\left(\frac{p}{q}\right) = 2$. Checking, 
        \[2f(2)  = 2(2)^3 - 5(2) +1 = 16 - 10 + 1 = 7 \neq 0\]
        Hence, $2$ is not a root, $f(a)$ doesn't have any rational roots and $x, y, z$ must be irrational. 
    \end{enumerate}
\end{solution}


\end{document}
