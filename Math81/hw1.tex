\documentclass[12pt]{article}

\usepackage{fullpage}
\usepackage{mdframed}
\usepackage{colonequals}
\usepackage{algpseudocode}
\usepackage{algorithm}
\usepackage[most, breakable]{tcolorbox}
\usepackage[all]{xy}
\usepackage{proof}
\usepackage{mathtools}
\usepackage{bbm}
\usepackage{amssymb}
\usepackage{amsthm}
\usepackage{amsmath}
\usepackage{amsxtra}
\usepackage{enumitem}
\newcommand{\bb}{\mathbb}


\newtheorem{theorem}{Theorem}[section]
\newtheorem{theorem*}{Theorem}
\newtheorem{definition}[theorem]{Definition}
\newtheorem{corollary}{Corollary}[theorem]
\newtheorem{lemma}[theorem]{Lemma}
\newtheorem{prop}[theorem]{Proposition}
\newtheorem{remark}[theorem]{Remark}


\newtheorem*{exercisehelper}{Exercise.}
\newenvironment{exercise}[1]{%
  \IfBlankTF{#1}
    {\renewcommand{\exercisehelper}{\textbf{Exercise} \unskip}}
    {\renewcommand\exercisehelper{\textbf{Exercise #1}}}%
  \exercisehelper
}{\endexercisehelper}

\theoremstyle{remark}
\newtheorem*{solution}{Solution}
\newcommand{\mathcat}[1]{\textup{\textbf{\textsf{#1}}}} % for defined terms

\newenvironment{problem}[1]
{ \begin{tcolorbox}[breakable]\noindent\textbf{Problem #1}.}
{\vskip 6pt \end{tcolorbox}}

\newenvironment{enumalph}
{\begin{enumerate}\renewcommand{\labelenumi}{\textnormal{(\alph{enumi})}}}
{\end{enumerate}}

\newenvironment{enumroman}
{\begin{enumerate}\renewcommand{\labelenumi}{\textnormal{(\roman{enumi})}}}
{\end{enumerate}}

\newcommand{\defi}[1]{\textsf{#1}} % for defined terms



\setlength{\hfuzz}{4pt}

\let\H\relax
\let\P\relax
\newcommand{\H}{\mathbb H}
\newcommand{\P}{\mathbb P}
\newcommand{\C}{\mathbb C}
\newcommand{\N}{\mathbb N}
\newcommand{\Q}{\mathbb Q}
\newcommand{\R}{\mathbb R}
\newcommand{\Z}{\mathbb Z}
\newcommand{\F}{\mathbb F}
\newcommand{\br}{\mathbf{r}}
\newcommand{\RP}{\mathbb{RP}}
\newcommand{\CP}{\mathbb{CP}}
\newcommand{\nbit}[1]{\{0, 1\}^{#1}}
\newcommand{\bits}{\{0, 1\}^{n}}
\newcommand{\bbni}{\bigbreak \noindent}
\newcommand{\norm}[1]{\left\vert\left\vert#1\right\vert\right\vert}
\newcommand{\dbar}{\overline{\partial}}
\let\d\relax
\newcommand{\d}{\partial}
\newcommand{\calO}{\mathcal{O}}
\newcommand{\calF}{\mathcal{F}}
\newcommand{\calG}{\mathcal{G}}
\newcommand{\calH}{\mathcal{H}}
\newcommand{\calE}{\mathcal{E}}
\newcommand{\calC}{\mathcal{C}}
\newcommand{\calD}{\mathcal{D}}

\let\1\relax
\newcommand{\1}{\mathbf{1}}
\newcommand{\fr}[2]{\left(\frac{#1}{#2}\right)}
\newcommand{\todo}[1]{\textcolor{red}{\textbf{TODO:} #1}}
\newcommand{\vecz}{\mathbf{z}}
\newcommand{\vecr}{\mathbf{r}}
\DeclareMathOperator{\Cinf}{C^{\infty}}
\DeclareMathOperator{\Id}{Id}
\DeclareMathOperator{\Ell}{Ell}
\DeclareMathOperator{\CL}{\mathcal{CL}}

\DeclareMathOperator{\Alt}{Alt}
\DeclareMathOperator{\Aut}{Aut}
\DeclareMathOperator{\ann}{ann}
\DeclareMathOperator{\codim}{codim}
\DeclareMathOperator{\End}{End}
\DeclareMathOperator{\Hom}{Hom}
\DeclareMathOperator{\id}{id}
\DeclareMathOperator{\M}{M}
\DeclareMathOperator{\Mat}{Mat}
\DeclareMathOperator{\Ob}{Ob}
\DeclareMathOperator{\opchar}{char}
\DeclareMathOperator{\opspan}{span}
\DeclareMathOperator{\rk}{rk}
\DeclareMathOperator{\sgn}{sgn}
\DeclareMathOperator{\Sym}{Sym}
\DeclareMathOperator{\tr}{tr}
\DeclareMathOperator{\img}{img}
\DeclareMathOperator{\coker}{coker}
\DeclareMathOperator{\Spec}{Spec}
\DeclareMathOperator{\CandE}{CandE}
\DeclareMathOperator{\CandO}{CandO}
\DeclareMathOperator{\argmax}{argmax}
\DeclareMathOperator{\first}{first}
\DeclareMathOperator{\last}{last}
\DeclareMathOperator{\cost}{cost}
\DeclareMathOperator{\dist}{dist}
\DeclareMathOperator{\path}{path}
\DeclareMathOperator{\parent}{parent}
\DeclareMathOperator{\argmin}{argmin}
\DeclareMathOperator{\excess}{excess}
\let\Pr\relax
\DeclareMathOperator{\Pr}{\mathbf{Pr}}
\DeclareMathOperator{\Exp}{\mathbb{E}}
\DeclareMathOperator{\Var}{\mathbf{Var}}
\let\limsup\relax
\DeclareMathOperator{\limsup}{limsup}
%Paired Delims
\DeclarePairedDelimiter\ceil{\lceil}{\rceil}
\let\oldceil\ceil
\renewcommand{\ceil}[1]{\oldceil*{#1}}

\DeclarePairedDelimiter{\floor}{\lfloor}{\rfloor}
\let\oldfloor\floor
\renewcommand{\floor}[1]{\oldfloor*{#1}}





\newcommand{\dagstar}{*}

\newcommand{\tbigwedge}{{\textstyle{\bigwedge}}}
\setlength{\parindent}{0pt}
\setlength{\parskip}{5pt}


\usepackage{listings}
\usepackage{courier}
\usepackage{microtype}


\lstset{
  basicstyle=\footnotesize\ttfamily,
  breaklines=true,
  breakatwhitespace=true
  columns=fullflexible,
  keepspaces=true,
  frame=single,
  escapeinside={(*@}{@*)}
}

\begin{document}

\title{Math 71: Abstract Algebra}

\author{Prishita Dharampal}
\date{}
\maketitle


\textbf{Credit Statement:} Talked to Sair Shaikh'26, and Math Stack Exchange.


\begin{problem}{1}
For $f(x)=x^4-1$ and $g(x) = 3x^2+3x$ find: the quotient and remainder
after dividing $f$~by~$g$; the gcd of $f$ and $g$; and the expression
of this gcd in the form $af+bg$ for some $a,b \in \Q[x]$.  For the
last two, you'll need to recall the Euclidean Algorithm and the Bezout Identity.
\end{problem}


\begin{problem}{2}
Prove that two polynomials $f, g \in \Z[x]$ are relatively prime in
$\Q[x]$ (i.e., they share no common nonconstant factor) if and only if
the ideal $(f,g)\subset \Z[x]$ contains a nonzero integer.
\end{problem}


\begin{problem}{3}
Decide whether each of the following polynomials is irreducible, and
if not, then find the factorization into monic irreducibles.
\begin{enumerate}\renewcommand{\itemsep}{3mm}
\item $x^4+1 \in \R[x]$

\item $x^4+1 \in \Q[x]$

\item $x^7 + 66x^6 - 77x + 737 \in \Q[x]$

\item $x^4+x^3+x^2+x+1 \in \Q[x]$

\item $x^3+5x^2-9x+3 \in \Q[x]$
\end{enumerate}
\end{problem}


\begin{problem}{4}
\textit{Irreducible polynomials over finite fields.}
Let $\F_3$ be the field with three elements.
\begin{enumerate}\renewcommand{\itemsep}{3mm}
\item Determine all the monic irreducible polynomials of degree $\leq
3$ in $\F_3[x]$.

\item Determine the number of monic irreducible polynomials of degree
$4$ in $\F_3[x]$. \\ 
\textbf{Hint.} This is easier than determining all of them.
\end{enumerate}
\end{problem}


\begin{problem}{5(a)}\textit{Symmetric polynomials.} Let $R$ be a commutative ring with 1 and $R[x_1,\dotsc,x_n]$ the ring
of polynomials in the variables $x_1,\dotsc,x_n$ with coefficients in
$R$.  Consider the symmetric group $S_n$ acting on the set
$\{x_1,\dotsc,x_n\}$ by permutations.  Extend this action linearly to
$R[x_1,x_2,\dotsc,x_n]$; for example, if $\sigma = (123) \in S_3$, then
$$
\sigma\cdot (x_1x_2 - 6x_3^2 + 7x_2x_3^2) = x_2x_3 - 6x_1^2 + 7x_3x_1^2.
$$  
Then this action satisfies $\sigma\cdot(f+g)=\sigma\cdot f +
\sigma\cdot g$ and $\sigma\cdot(fg) = (\sigma\cdot f)(\sigma\cdot g)$
for all $\sigma \in S_n$ and all $f,g \in
R[x_1,\dotsc,x_n]$.

Let $S \subset R[x_1,\dotsc,x_n]$ be the subset fixed under the
action of $S_n$.  Prove that $S$ is a subring with 1.  This is called
the \textbf{ring of symmetric polynomials}.

\end{problem}


\begin{problem}{5(b)} For each $n \geq 0$, define polynomials $e_i \in
R[x_1,\dotsc,x_n]$ by  $e_0=1$ and
$$
e_1  {}= x_1+\dotsm+x_n,
\quad
e_2  {}= \sum_{1 \leq i < j \leq n}x_ix_j,
\quad 
\dotsc, {} \quad
e_n  {}= x_1\dotsm x_n
$$
and $e_k=0$ for $k > n$. In words, $e_k$ is the sum of all distinct
products of subsets of $k$ distinct variables.  Prove that each $e_k$
is a symmetric polynomial.  These are called the \textbf{elementary
symmetric polynomials}.
\end{problem}

\begin{problem}{5(c)}The \textbf{generic polynomial} of degree $n$ is the polynomial
$$
f(x) = (x-x_1)(x-x_2)\dotsm (x-x_n)
$$
in the ring $R[x_1,\dotsc,x_n][x]$ of polynomials in $x$ with
coefficients in $R[x_1,\dotsc,x_n]$.  Prove (by induction) that

\begin{align*}
    f(x) &= (x-x_1)(x-x_2)\dotsm (x-x_n)=x^n - e_1 x^{n-1} + e_2 x^{n-2} + \dotsm
+ (-1)^ne_n  \\
&= \sum_{j=0}^n (-1)^{n-j} e_{n-j} x^j.
\end{align*}

\end{problem}

\begin{problem}{5(d)}For each $k \geq 1$, define the \textbf{power sums} $p_k =
x_1^k + \dotsm + x_n^k$ in $R[x_1,\dotsc,x_n]$.  Clearly, the power
sums are symmetric.  Verify the following identities by hand:
$$
p_1=e_1, \quad p_2 = e_1p_1 - 2e_2, \quad p_3 = e_1p_2 - e_2p_1 + 3e_3
$$
In general \textbf{Newton's identities} in $R[x_1,\dotsc,x_n]$ are
(recall that $e_k=0$ for $k>n$):
$$
p_k - e_1 p_{k-1} + e_2 p_{k-2} - \dotsm + (-1)^{k-1}e_{k-1}p_1 +
(-1)^k k e_k = 0.
$$
Prove Newton's identities whenever $k \geq n$.\\

    \textbf{Hint.} For each $i$, consider the equation in part (c) for $f(x_i)$ and
sum all these equations together.  This gives Newton's identity for
$k=n$.  Set extra variables to zero to get the identities for $k > n$
from this.  (Fun. Can you come up with a proof when $1 \leq k \leq n$?)
\end{problem}


\begin{problem}{6}\textit{Use the force, my Newton!}
\begin{enumerate}\renewcommand{\itemsep}{3mm}
\item If $x,y,z$ are complex numbers satisfying
$$
x+y+z=1, \qquad x^2+y^2+z^2=6, \qquad x^3+y^3+z^3=7,
$$
then prove that $x^n + y^n + z^n$ is rational for any positive integer
$n$.

\item Calculate $x^4+y^4+z^4$.

\item Prove that each of $x,y,z$ are not rational numbers.  
\end{enumerate}  
\end{problem}


\vfill

\end{document}
