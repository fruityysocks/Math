\documentclass[12pt]{article}

\usepackage{listings}
\usepackage{courier}
\usepackage{microtype}

\usepackage{fullpage}
\usepackage{mdframed}
\usepackage{colonequals}
\usepackage{algpseudocode}
\usepackage{algorithm}
\usepackage[most, breakable]{tcolorbox}
\usepackage[all]{xy}
\usepackage{proof}
\usepackage{mathtools}
\usepackage{bbm}
\usepackage{amssymb}
\usepackage{amsthm}
\usepackage{amsmath}
\usepackage{amsxtra}
\usepackage{enumitem}
\newcommand{\bb}{\mathbb}


\newtheorem{theorem}{Theorem}[section]
\newtheorem{theorem*}{Theorem}
\newtheorem{definition}[theorem]{Definition}
\newtheorem{corollary}{Corollary}[theorem]
\newtheorem{lemma}[theorem]{Lemma}
\newtheorem{prop}[theorem]{Proposition}
\newtheorem{remark}[theorem]{Remark}


\newtheorem*{exercisehelper}{Exercise.}
\newenvironment{exercise}[1]{%
  \IfBlankTF{#1}
    {\renewcommand{\exercisehelper}{\textbf{Exercise} \unskip}}
    {\renewcommand\exercisehelper{\textbf{Exercise #1}}}%
  \exercisehelper
}{\endexercisehelper}

\theoremstyle{remark}
\newtheorem*{solution}{Solution}
\newcommand{\mathcat}[1]{\textup{\textbf{\textsf{#1}}}} % for defined terms

\newenvironment{problem}[1]
{ \begin{tcolorbox}[breakable]\noindent\textbf{Problem #1}.}
{\vskip 6pt \end{tcolorbox}}

\newenvironment{enumalph}
{\begin{enumerate}\renewcommand{\labelenumi}{\textnormal{(\alph{enumi})}}}
{\end{enumerate}}

\newenvironment{enumroman}
{\begin{enumerate}\renewcommand{\labelenumi}{\textnormal{(\roman{enumi})}}}
{\end{enumerate}}

\newcommand{\defi}[1]{\textsf{#1}} % for defined terms



\setlength{\hfuzz}{4pt}

\let\H\relax
\let\P\relax
\newcommand{\H}{\mathbb H}
\newcommand{\P}{\mathbb P}
\newcommand{\C}{\mathbb C}
\newcommand{\N}{\mathbb N}
\newcommand{\Q}{\mathbb Q}
\newcommand{\R}{\mathbb R}
\newcommand{\Z}{\mathbb Z}
\newcommand{\F}{\mathbb F}
\newcommand{\br}{\mathbf{r}}
\newcommand{\RP}{\mathbb{RP}}
\newcommand{\CP}{\mathbb{CP}}
\newcommand{\nbit}[1]{\{0, 1\}^{#1}}
\newcommand{\bits}{\{0, 1\}^{n}}
\newcommand{\bbni}{\bigbreak \noindent}
\newcommand{\norm}[1]{\left\vert\left\vert#1\right\vert\right\vert}
\newcommand{\dbar}{\overline{\partial}}
\let\d\relax
\newcommand{\d}{\partial}
\newcommand{\calO}{\mathcal{O}}
\newcommand{\calF}{\mathcal{F}}
\newcommand{\calG}{\mathcal{G}}
\newcommand{\calH}{\mathcal{H}}
\newcommand{\calE}{\mathcal{E}}
\newcommand{\calC}{\mathcal{C}}
\newcommand{\calD}{\mathcal{D}}

\let\1\relax
\newcommand{\1}{\mathbf{1}}
\newcommand{\fr}[2]{\left(\frac{#1}{#2}\right)}
\newcommand{\todo}[1]{\textcolor{red}{\textbf{TODO:} #1}}
\newcommand{\vecz}{\mathbf{z}}
\newcommand{\vecr}{\mathbf{r}}
\DeclareMathOperator{\Cinf}{C^{\infty}}
\DeclareMathOperator{\Id}{Id}
\DeclareMathOperator{\Ell}{Ell}
\DeclareMathOperator{\CL}{\mathcal{CL}}

\DeclareMathOperator{\Alt}{Alt}
\DeclareMathOperator{\Aut}{Aut}
\DeclareMathOperator{\ann}{ann}
\DeclareMathOperator{\codim}{codim}
\DeclareMathOperator{\End}{End}
\DeclareMathOperator{\Hom}{Hom}
\DeclareMathOperator{\id}{id}
\DeclareMathOperator{\M}{M}
\DeclareMathOperator{\Mat}{Mat}
\DeclareMathOperator{\Ob}{Ob}
\DeclareMathOperator{\opchar}{char}
\DeclareMathOperator{\opspan}{span}
\DeclareMathOperator{\rk}{rk}
\DeclareMathOperator{\sgn}{sgn}
\DeclareMathOperator{\Sym}{Sym}
\DeclareMathOperator{\tr}{tr}
\DeclareMathOperator{\img}{img}
\DeclareMathOperator{\coker}{coker}
\DeclareMathOperator{\Spec}{Spec}
\DeclareMathOperator{\CandE}{CandE}
\DeclareMathOperator{\CandO}{CandO}
\DeclareMathOperator{\argmax}{argmax}
\DeclareMathOperator{\first}{first}
\DeclareMathOperator{\last}{last}
\DeclareMathOperator{\cost}{cost}
\DeclareMathOperator{\dist}{dist}
\DeclareMathOperator{\path}{path}
\DeclareMathOperator{\parent}{parent}
\DeclareMathOperator{\argmin}{argmin}
\DeclareMathOperator{\excess}{excess}
\let\Pr\relax
\DeclareMathOperator{\Pr}{\mathbf{Pr}}
\DeclareMathOperator{\Exp}{\mathbb{E}}
\DeclareMathOperator{\Var}{\mathbf{Var}}
\let\limsup\relax
\DeclareMathOperator{\limsup}{limsup}
%Paired Delims
\DeclarePairedDelimiter\ceil{\lceil}{\rceil}
\let\oldceil\ceil
\renewcommand{\ceil}[1]{\oldceil*{#1}}

\DeclarePairedDelimiter{\floor}{\lfloor}{\rfloor}
\let\oldfloor\floor
\renewcommand{\floor}[1]{\oldfloor*{#1}}





\newcommand{\dagstar}{*}

\newcommand{\tbigwedge}{{\textstyle{\bigwedge}}}
\setlength{\parindent}{0pt}
\setlength{\parskip}{5pt}


\usepackage{listings}
\usepackage{courier}
\usepackage{microtype}


\lstset{
  basicstyle=\footnotesize\ttfamily,
  breaklines=true,
  breakatwhitespace=true
  columns=fullflexible,
  keepspaces=true,
  frame=single,
  escapeinside={(*@}{@*)}
}

\begin{document}

\title{Math 71: Abstract Algebra}

\author{Prishita Dharampal}
\date{}
\maketitle

For this lab assignment, you will be compiling short C programs and then
reverse engineering them to see what they look like inside. This assignment
will be performed individually, first on the Unix command line and then in
the GHIDRA reverse engineering framework.

\lstset{
  basicstyle=\footnotesize\ttfamily,
  breaklines=true,
  breakatwhitespace=true
  columns=fullflexible,
  keepspaces=true,
  frame=single,
  escapeinside={(*@}{@*)}
}

\begin{problem}{1} The following is a short C program. Place it in a file named
\texttt{first.c} and compile it with
\texttt{cc -o first first.c} to produce an executable.
You can run this with \texttt{./first}.

\begin{lstlisting}[language=]
#include <stdio.h>

int main(int argc, char **argv) {
    if (argc > 1) {
        printf("Hello world. %s\n", argv[1]);
    } else {
        printf("Aloha!\n");
    }
    return 0;
}
\end{lstlisting}

\begin{enumerate}
  \item What file format and architecture is your file? You can find this by
        running the \texttt{file} command on your executable.
  \item Disassemble your executable with the \texttt{objdump -d} command and
        write down the disassembly of the \texttt{main} function.
  \item Add comments to the disassembly marking:
        \begin{itemize}
          \item the return value (\texttt{0}) of \texttt{main},
          \item the address of the strings \texttt{"Hello world argv[1]"} and
                \texttt{"Aloha!"},
          \item the registers that store \texttt{int argc} and
                \texttt{char **argv}.
        \end{itemize}
        This process is typically called \emph{annotation}.
  \item \texttt{objdump -d} often does not include strings in its listing, but
        \texttt{objdump -D} and other tools will. Include the output lines
        representing the strings, along with the options or actions used to obtain them.
  \item Why are the bytes of the string interpreted as machine language?
  \item Use an ASCII table to decode the bytes of the strings. C strings end
        with a null byte (\texttt{0}), so include every byte from the
        beginning of the string to the null terminator.
\end{enumerate}
\end{problem}

\begin{solution}
    \bbni 

    \begin{enumerate}
        \item The file format is \texttt{Mach -O 64-bit} and the architecture is \texttt{x86\_64}. 
        \item  \begin{lstlisting}
0000000100000470 <_main>:
100000470: 55                          	pushq	%rbp
100000471: 48 89 e5                    	movq	%rsp, %rbp
100000474: 48 83 ec 10                 	subq	$0x10, %rsp
100000478: c7 45 fc 00 00 00 00        	movl	$0x0, -0x4(%rbp)
10000047f: 89 7d f8                    	movl	%edi, -0x8(%rbp)
100000482: 48 89 75 f0                 	movq	%rsi, -0x10(%rbp)
100000486: 83 7d f8 01                 	cmpl	$0x1, -0x8(%rbp)
10000048a: 7e 18                       	jle	0x1000004a4 <_main+0x34>
10000048c: 48 8b 45 f0                 	movq	-0x10(%rbp), %rax
100000490: 48 8b 70 08                 	movq	0x8(%rax), %rsi
100000494: 48 8d 3d 25 00 00 00        	leaq	0x25(%rip), %rdi        ## 0x1000004c0 <_printf+0x1000004c0>
10000049b: b0 00                       	movb	$0x0, %al
10000049d: e8 18 00 00 00              	callq	0x1000004ba <_printf+0x1000004ba>
1000004a2: eb 0e                       	jmp	0x1000004b2 <_main+0x42>
1000004a4: 48 8d 3d 26 00 00 00        	leaq	0x26(%rip), %rdi        ## 0x1000004d1 <_printf+0x1000004d1>
1000004ab: b0 00                       	movb	$0x0, %al
1000004ad: e8 08 00 00 00              	callq	0x1000004ba <_printf+0x1000004ba>
1000004b2: 31 c0                       	xorl	%eax, %eax
1000004b4: 48 83 c4 10                 	addq	$0x10, %rsp
1000004b8: 5d                          	popq	%rbp
1000004b9: c3                          	retq
\end{lstlisting}

\newpage

    \item \begin{lstlisting}
0000000100000470 <_main>:
100000470: 55                          	pushq	%rbp
100000471: 48 89 e5                    	movq	%rsp, %rbp
100000474: 48 83 ec 10                 	subq	$0x10, %rsp
100000478: c7 45 fc 00 00 00 00        	movl	$0x0, -0x4(%rbp)
10000047f: 89 7d f8                    	movl	%edi, -0x8(%rbp) (*@ \\ \textcolor{red}{\#\#copies \texttt{argc} from \texttt{edi} to  \texttt{[rbp - 8]}}@*)
100000482: 48 89 75 f0                 	movq	%rsi, -0x10(%rbp) (*@ \textcolor{red}{\#\#copies \texttt{argv} from \texttt{rsi} to \texttt{[rbp - 10]}}@*)
100000486: 83 7d f8 01                 	cmpl	$0x1, -0x8(%rbp)
10000048a: 7e 18                       	jle	0x1000004a4 <_main+0x34>
10000048c: 48 8b 45 f0                 	movq	-0x10(%rbp), %rax
100000490: 48 8b 70 08                 	movq	0x8(%rax), %rsi
100000494: 48 8d 3d 25 00 00 00        	leaq	0x25(%rip), %rdi        ## 0x1000004c0 <_printf+0x1000004c0>
(*@ \textcolor{red}{\#\#\texttt{`Hello world.'} is stored at \texttt{0x25(\%rip)}}@*)
10000049b: b0 00                       	movb	$0x0, %al
10000049d: e8 18 00 00 00              	callq	0x1000004ba <_printf+0x1000004ba>
1000004a2: eb 0e                       	jmp	0x1000004b2 <_main+0x42>
1000004a4: 48 8d 3d 26 00 00 00        	leaq	0x26(%rip), %rdi        ## 0x1000004d1 <_printf+0x1000004d1>
(*@ \textcolor{red}{\#\#\texttt{`Aloha!'} is stored at \texttt{0x26(\%rip)}}@*)
1000004ab: b0 00                       	movb	$0x0, %al
1000004ad: e8 08 00 00 00              	callq	0x1000004ba <_printf+0x1000004ba>
1000004b2: 31 c0                       	xorl	%eax, %eax
(*@ \textcolor{red}{\#\#return value of main}@*)
1000004b4: 48 83 c4 10                 	addq	$0x10, %rsp
1000004b8: 5d                          	popq	%rbp
1000004b9: c3                          	retq
\end{lstlisting}

        \item  Used \texttt{objdump -D} to get the entire disassembly including the snippet below. 
\begin{lstlisting}
Disassembly of section __TEXT,__cstring:

00000001000004c0 <__cstring>:
1000004c0: 48 65                       	gs
1000004c2: 6c                          	insb	%dx, %es:(%rdi)
1000004c3: 6c                          	insb	%dx, %es:(%rdi)
1000004c4: 6f                          	outsl	(%rsi), %dx
1000004c5: 20 77 6f                    	andb	%dh, 0x6f(%rdi)
1000004c8: 72 6c                       	jb	0x100000536 <_printf+0x100000536>
1000004ca: 64 2e 20 25 73 0a 00 41     	andb	%ah, %cs:0x41000a73(%rip)
1000004d2: 6c                          	insb	%dx, %es:(%rdi)
1000004d3: 6f                          	outsl	(%rsi), %dx
1000004d4: 68 61 21 0a 00              	pushq	$0xa2161                ## imm = 0xA2161

\end{lstlisting}

        \item Using \texttt{-D} with \texttt{objdump} assumes that any symbols found in a code section are code, and disassembles them. 
        \item 
\begin{lstlisting}
48 -> 'H'
65 -> 'e'
6c -> 'l' 
6c -> 'l' 
6f -> 'o' 
20 -> ' ' (white space)
77 -> 'w' 
6f -> 'o' 
72 -> 'r' 
6c -> 'l' 
64 -> 'd' 
2e -> '.'
20 -> ' ' (white space)
25 -> '%' 
73 -> 's'
0a ->  LF (line feed)
00 -> NUL
41 -> 'A' 
6c -> 'l' 
6f -> 'o' 
68 -> 'h'
61 -> 'a' 
21 -> '!' 
0a -> LF (line feed)
00 -> NUL
\end{lstlisting}
    \end{enumerate}
\end{solution}

\newpage

\begin{problem}{2} Let’s try a program that’s a little more complicated, which we’ll call
\texttt{second.c}.

\begin{lstlisting}
#include <stdio.h>

int square(int i) {
    return i * i;
}

int main(int argc, char **argv) {
    for (int i = 0; i < 10; i++) {
        printf("The square of %d is %d.\n", i, square(i));
    }
}
\end{lstlisting}

Compile the program with both \texttt{cc -O3 -o second-fast second.c} and \texttt{cc -Os -o second-small second.c}. The \texttt{-O} flags tell the compiler to optimize the build, with \texttt{-O3} producing the fastest code and \texttt{-Os} producing the smallest code.
\bbni

Disassemble both executables with \texttt{objdump -d}. What is different about their \texttt{main} functions? Which version is easier to understand, and why?

\end{problem}


\begin{solution}
    \bbni 

    Except for the obvious difference in the number of instructions, the \texttt{second-fast objdump} has no loops or calls to \texttt{square()} and is incredibily unreadable. Whereas \texttt{second-small objdump} is easier to understand because its structure matches the source code and one can identify the loop structure from the disassembly. 
\end{solution}

\newpage

\begin{problem}{3}
Our first two programs included symbols, so it was easy to locate
\texttt{main} and identify variable names. For this program, strip the
symbols to create a more realistic executable.

Compile with \texttt{cc -o third third.c}, then run \texttt{objdump -d}
before and after stripping the binary with \texttt{strip third}.

What are the differences between these two \texttt{objdump} outputs, and why?

\begin{lstlisting}
#include <stdio.h>
#include <stdlib.h>

__attribute__((noinline))
void guess(int g) {
    if (g == 42)
        printf("That's right!\n");
    else
        printf("Nope, that's the wrong number.\n");
}

int main(int argc, char **argv) {
    if (argc == 2) {
        guess(atoi(argv[1]));
    } else {
        printf("Give me a number?\n");
    }
}
\end{lstlisting}

Instead of \texttt{objdump}, use GHIDRA for this assignment.

After importing the binary and allowing GHIDRA to auto-analyze it, complete
the following:

\begin{enumerate}
  \item In the Symbols pane, practice navigating between functions.
  \item Observe that after stripping, GHIDRA does not know local function or
        variable names, though it may identify standard library functions such
        as \texttt{atoi()}.
  \item Navigate to the function that calls \texttt{atoi()}, rename it
        \texttt{main}.
  \item Locate and rename the \texttt{guess()} function.
  \item Rename local variables in \texttt{guess()} to match the original
        source code.
  \item Using the decompiler view as a guide, comment each line of the
        assembly code for \texttt{guess()} and paste the commented
        disassembly into your report.
\end{enumerate}
\end{problem}

\begin{solution}
    \bbni 

    
\end{solution}

\begin{problem}{4} For extra credit, repeat the third section using tools such as Binary Ninja,
IDA Pro, Radare2, or Cutter.
\end{problem}
\end{document}