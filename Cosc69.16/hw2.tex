\documentclass[12pt]{article}

\usepackage{listings}
\usepackage{courier}
\usepackage{microtype}

\usepackage{fullpage}
\usepackage{mdframed}
\usepackage{colonequals}
\usepackage{algpseudocode}
\usepackage{algorithm}
\usepackage[most, breakable]{tcolorbox}
\usepackage[all]{xy}
\usepackage{proof}
\usepackage{mathtools}
\usepackage{bbm}
\usepackage{amssymb}
\usepackage{amsthm}
\usepackage{amsmath}
\usepackage{amsxtra}
\usepackage{enumitem}
\newcommand{\bb}{\mathbb}


\newtheorem{theorem}{Theorem}[section]
\newtheorem{theorem*}{Theorem}
\newtheorem{definition}[theorem]{Definition}
\newtheorem{corollary}{Corollary}[theorem]
\newtheorem{lemma}[theorem]{Lemma}
\newtheorem{prop}[theorem]{Proposition}
\newtheorem{remark}[theorem]{Remark}


\newtheorem*{exercisehelper}{Exercise.}
\newenvironment{exercise}[1]{%
  \IfBlankTF{#1}
    {\renewcommand{\exercisehelper}{\textbf{Exercise} \unskip}}
    {\renewcommand\exercisehelper{\textbf{Exercise #1}}}%
  \exercisehelper
}{\endexercisehelper}

\theoremstyle{remark}
\newtheorem*{solution}{Solution}
\newcommand{\mathcat}[1]{\textup{\textbf{\textsf{#1}}}} % for defined terms

\newenvironment{problem}[1]
{ \begin{tcolorbox}[breakable]\noindent\textbf{Problem #1}.}
{\vskip 6pt \end{tcolorbox}}

\newenvironment{enumalph}
{\begin{enumerate}\renewcommand{\labelenumi}{\textnormal{(\alph{enumi})}}}
{\end{enumerate}}

\newenvironment{enumroman}
{\begin{enumerate}\renewcommand{\labelenumi}{\textnormal{(\roman{enumi})}}}
{\end{enumerate}}

\newcommand{\defi}[1]{\textsf{#1}} % for defined terms



\setlength{\hfuzz}{4pt}

\let\H\relax
\let\P\relax
\newcommand{\H}{\mathbb H}
\newcommand{\P}{\mathbb P}
\newcommand{\C}{\mathbb C}
\newcommand{\N}{\mathbb N}
\newcommand{\Q}{\mathbb Q}
\newcommand{\R}{\mathbb R}
\newcommand{\Z}{\mathbb Z}
\newcommand{\F}{\mathbb F}
\newcommand{\br}{\mathbf{r}}
\newcommand{\RP}{\mathbb{RP}}
\newcommand{\CP}{\mathbb{CP}}
\newcommand{\nbit}[1]{\{0, 1\}^{#1}}
\newcommand{\bits}{\{0, 1\}^{n}}
\newcommand{\bbni}{\bigbreak \noindent}
\newcommand{\norm}[1]{\left\vert\left\vert#1\right\vert\right\vert}
\newcommand{\dbar}{\overline{\partial}}
\let\d\relax
\newcommand{\d}{\partial}
\newcommand{\calO}{\mathcal{O}}
\newcommand{\calF}{\mathcal{F}}
\newcommand{\calG}{\mathcal{G}}
\newcommand{\calH}{\mathcal{H}}
\newcommand{\calE}{\mathcal{E}}
\newcommand{\calC}{\mathcal{C}}
\newcommand{\calD}{\mathcal{D}}

\let\1\relax
\newcommand{\1}{\mathbf{1}}
\newcommand{\fr}[2]{\left(\frac{#1}{#2}\right)}
\newcommand{\todo}[1]{\textcolor{red}{\textbf{TODO:} #1}}
\newcommand{\vecz}{\mathbf{z}}
\newcommand{\vecr}{\mathbf{r}}
\DeclareMathOperator{\Cinf}{C^{\infty}}
\DeclareMathOperator{\Id}{Id}
\DeclareMathOperator{\Ell}{Ell}
\DeclareMathOperator{\CL}{\mathcal{CL}}

\DeclareMathOperator{\Alt}{Alt}
\DeclareMathOperator{\Aut}{Aut}
\DeclareMathOperator{\ann}{ann}
\DeclareMathOperator{\codim}{codim}
\DeclareMathOperator{\End}{End}
\DeclareMathOperator{\Hom}{Hom}
\DeclareMathOperator{\id}{id}
\DeclareMathOperator{\M}{M}
\DeclareMathOperator{\Mat}{Mat}
\DeclareMathOperator{\Ob}{Ob}
\DeclareMathOperator{\opchar}{char}
\DeclareMathOperator{\opspan}{span}
\DeclareMathOperator{\rk}{rk}
\DeclareMathOperator{\sgn}{sgn}
\DeclareMathOperator{\Sym}{Sym}
\DeclareMathOperator{\tr}{tr}
\DeclareMathOperator{\img}{img}
\DeclareMathOperator{\coker}{coker}
\DeclareMathOperator{\Spec}{Spec}
\DeclareMathOperator{\CandE}{CandE}
\DeclareMathOperator{\CandO}{CandO}
\DeclareMathOperator{\argmax}{argmax}
\DeclareMathOperator{\first}{first}
\DeclareMathOperator{\last}{last}
\DeclareMathOperator{\cost}{cost}
\DeclareMathOperator{\dist}{dist}
\DeclareMathOperator{\path}{path}
\DeclareMathOperator{\parent}{parent}
\DeclareMathOperator{\argmin}{argmin}
\DeclareMathOperator{\excess}{excess}
\let\Pr\relax
\DeclareMathOperator{\Pr}{\mathbf{Pr}}
\DeclareMathOperator{\Exp}{\mathbb{E}}
\DeclareMathOperator{\Var}{\mathbf{Var}}
\let\limsup\relax
\DeclareMathOperator{\limsup}{limsup}
%Paired Delims
\DeclarePairedDelimiter\ceil{\lceil}{\rceil}
\let\oldceil\ceil
\renewcommand{\ceil}[1]{\oldceil*{#1}}

\DeclarePairedDelimiter{\floor}{\lfloor}{\rfloor}
\let\oldfloor\floor
\renewcommand{\floor}[1]{\oldfloor*{#1}}





\newcommand{\dagstar}{*}

\newcommand{\tbigwedge}{{\textstyle{\bigwedge}}}
\setlength{\parindent}{0pt}
\setlength{\parskip}{5pt}


\usepackage{listings}
\usepackage{courier}
\usepackage{microtype}


\lstset{
  basicstyle=\footnotesize\ttfamily,
  breaklines=true,
  breakatwhitespace=true
  columns=fullflexible,
  keepspaces=true,
  frame=single,
  escapeinside={(*@}{@*)}
}

\begin{document}

\title{Math 71: Abstract Algebra}

\author{Prishita Dharampal}
\date{}
\maketitle

For this lab assignment, you will be compiling short C programs and then
reverse engineering them to see what they look like inside. This assignment
will be performed individually, first on the Unix command line and then in
the GHIDRA reverse engineering framework.

\lstset{
  basicstyle=\footnotesize\ttfamily,
  columns=fullflexible,
  keepspaces=true,
  frame=single
}

\begin{problem}{1} The following is a short C program. Place it in a file named
\texttt{first.c} and compile it with
\texttt{cc -o first first.c} to produce an executable.
You can run this with \texttt{./first}.

\begin{lstlisting}
#include <stdio.h>

int main(int argc, char **argv) {
    if (argc > 1) {
        printf("Hello world. %s\n", argv[1]);
    } else {
        printf("Aloha!\n");
    }
    return 0;
}
\end{lstlisting}

\begin{enumerate}
  \item What file format and architecture is your file? You can find this by
        running the \texttt{file} command on your executable.
  \item Disassemble your executable with the \texttt{objdump -d} command,
        and write down the \texttt{main} function’s disassembly.
  \item Add some comments to the disassembly, marking the return value
        (\texttt{0}) of \texttt{main}, the address of two strings
        (\texttt{"Hello world argv[1]"} and \texttt{"Aloha!"}), and the
        registers that store \texttt{int argc} and \texttt{char **argv}.
        Note we typically will call adding such comments \emph{annotation}.
  \item \texttt{objdump -d} often doesn’t include strings in its listing, but
        \texttt{objdump -D} will. Include the lines of the string in your
        document.
  \item Why are the bytes of the string interpreted as machine language?
  \item Use an ASCII table to decode the bytes of the strings. C strings end
        with a null byte (\texttt{0}), so you should include each and every
        byte from the beginning to the null.
\end{enumerate}

\end{problem}

\begin{problem}{2} Let’s try a program that’s a little more complicated, which we’ll call
\texttt{second.c}.


\begin{lstlisting}
#include <stdio.h>

int square(int i) {
    return i * i;
}

int main(int argc, char **argv) {
    for (int i = 0; i < 10; i++) {
        printf("The square of %d is %d.\n", i, square(i));
    }
}
\end{lstlisting}

\vspace{1em}

\begin{enumerate}
  \item Compile the program with both
        \texttt{cc -O3 second-fast second.c} and
        \texttt{cc -Os second-small second.c}.
        The \texttt{-O} flags tell the compiler to optimize the build, with
        \texttt{-O3} making the fastest code and \texttt{-Os} making the
        smallest code.
  \item Disassemble both of these files with \texttt{objdump -d}. What’s
        different about their \texttt{main} functions? Which one is easier
        to understand, and why?
\end{enumerate}

\end{problem}

\begin{problem}{3}
Our first two programs included symbols, so we could easily tell where
\texttt{main} was and what the variable names were. For this third program,
we will strip the symbols to leave an executable more like those that are
distributed commercially. Compile it with
\texttt{cc -o third third.c} and do \texttt{objdump -d} before and after
stripping the binary with \texttt{strip third}.

\vspace{1em}

What are the differences between these two \texttt{objdump} outputs, and why?


\newpage 

\begin{lstlisting}
#include <stdio.h>
#include <stdlib.h>

__attribute__((noinline))
void guess(int g) {
    if (g == 42)
        printf("That's right!\n");
    else
        printf("Nope, that's the wrong number.\n");
}

int main(int argc, char **argv) {
    if (argc == 2) {
        guess(atoi(argv[1]));
    } else {
        printf("Give me a number?\n");
    }
}
\end{lstlisting}

\vspace{1.5em}

Instead of \texttt{objdump}, we’ll use GHIDRA for this assignment, a free
reverse engineering tool developed by the National Security Agency.

\vspace{1em}

After opening GHIDRA, create a New Project. After the file is imported,
open it and click Yes to have GHIDRA auto-analyze the executable.

\vspace{1em}

\begin{enumerate}
  \item In the Symbols pane, practice navigating between functions.
  \item Because the binary is stripped, GHIDRA does not know local function
        or variable names, though it may identify standard library calls
        such as \texttt{atoi()}.
  \item Navigate to the function that calls \texttt{atoi()}. Rename it
        \texttt{main}.
  \item Find and name the \texttt{guess()} function.
  \item Rename the local variables in \texttt{guess()} to match the
        original source code.
  \item Comment each line of the assembly for \texttt{guess()} and paste
        the commented disassembly into your report.
\end{enumerate}

\end{problem}

\begin{problem}{4} For extra credit, repeat the third section using tools such as Binary Ninja,
IDA Pro, Radare2, or Cutter.

\end{problem}
\end{document}