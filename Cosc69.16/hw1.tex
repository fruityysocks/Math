\documentclass[12pt]{article}
\usepackage{listings}
\usepackage{courier}
\usepackage{microtype}

\usepackage{fullpage}
\usepackage{mdframed}
\usepackage{colonequals}
\usepackage{algpseudocode}
\usepackage{algorithm}
\usepackage[most, breakable]{tcolorbox}
\usepackage[all]{xy}
\usepackage{proof}
\usepackage{mathtools}
\usepackage{bbm}
\usepackage{amssymb}
\usepackage{amsthm}
\usepackage{amsmath}
\usepackage{amsxtra}
\usepackage{enumitem}
\newcommand{\bb}{\mathbb}


\newtheorem{theorem}{Theorem}[section]
\newtheorem{theorem*}{Theorem}
\newtheorem{definition}[theorem]{Definition}
\newtheorem{corollary}{Corollary}[theorem]
\newtheorem{lemma}[theorem]{Lemma}
\newtheorem{prop}[theorem]{Proposition}
\newtheorem{remark}[theorem]{Remark}


\newtheorem*{exercisehelper}{Exercise.}
\newenvironment{exercise}[1]{%
  \IfBlankTF{#1}
    {\renewcommand{\exercisehelper}{\textbf{Exercise} \unskip}}
    {\renewcommand\exercisehelper{\textbf{Exercise #1}}}%
  \exercisehelper
}{\endexercisehelper}

\theoremstyle{remark}
\newtheorem*{solution}{Solution}
\newcommand{\mathcat}[1]{\textup{\textbf{\textsf{#1}}}} % for defined terms

\newenvironment{problem}[1]
{ \begin{tcolorbox}[breakable]\noindent\textbf{Problem #1}.}
{\vskip 6pt \end{tcolorbox}}

\newenvironment{enumalph}
{\begin{enumerate}\renewcommand{\labelenumi}{\textnormal{(\alph{enumi})}}}
{\end{enumerate}}

\newenvironment{enumroman}
{\begin{enumerate}\renewcommand{\labelenumi}{\textnormal{(\roman{enumi})}}}
{\end{enumerate}}

\newcommand{\defi}[1]{\textsf{#1}} % for defined terms



\setlength{\hfuzz}{4pt}

\let\H\relax
\let\P\relax
\newcommand{\H}{\mathbb H}
\newcommand{\P}{\mathbb P}
\newcommand{\C}{\mathbb C}
\newcommand{\N}{\mathbb N}
\newcommand{\Q}{\mathbb Q}
\newcommand{\R}{\mathbb R}
\newcommand{\Z}{\mathbb Z}
\newcommand{\F}{\mathbb F}
\newcommand{\br}{\mathbf{r}}
\newcommand{\RP}{\mathbb{RP}}
\newcommand{\CP}{\mathbb{CP}}
\newcommand{\nbit}[1]{\{0, 1\}^{#1}}
\newcommand{\bits}{\{0, 1\}^{n}}
\newcommand{\bbni}{\bigbreak \noindent}
\newcommand{\norm}[1]{\left\vert\left\vert#1\right\vert\right\vert}
\newcommand{\dbar}{\overline{\partial}}
\let\d\relax
\newcommand{\d}{\partial}
\newcommand{\calO}{\mathcal{O}}
\newcommand{\calF}{\mathcal{F}}
\newcommand{\calG}{\mathcal{G}}
\newcommand{\calH}{\mathcal{H}}
\newcommand{\calE}{\mathcal{E}}
\newcommand{\calC}{\mathcal{C}}
\newcommand{\calD}{\mathcal{D}}

\let\1\relax
\newcommand{\1}{\mathbf{1}}
\newcommand{\fr}[2]{\left(\frac{#1}{#2}\right)}
\newcommand{\todo}[1]{\textcolor{red}{\textbf{TODO:} #1}}
\newcommand{\vecz}{\mathbf{z}}
\newcommand{\vecr}{\mathbf{r}}
\DeclareMathOperator{\Cinf}{C^{\infty}}
\DeclareMathOperator{\Id}{Id}
\DeclareMathOperator{\Ell}{Ell}
\DeclareMathOperator{\CL}{\mathcal{CL}}

\DeclareMathOperator{\Alt}{Alt}
\DeclareMathOperator{\Aut}{Aut}
\DeclareMathOperator{\ann}{ann}
\DeclareMathOperator{\codim}{codim}
\DeclareMathOperator{\End}{End}
\DeclareMathOperator{\Hom}{Hom}
\DeclareMathOperator{\id}{id}
\DeclareMathOperator{\M}{M}
\DeclareMathOperator{\Mat}{Mat}
\DeclareMathOperator{\Ob}{Ob}
\DeclareMathOperator{\opchar}{char}
\DeclareMathOperator{\opspan}{span}
\DeclareMathOperator{\rk}{rk}
\DeclareMathOperator{\sgn}{sgn}
\DeclareMathOperator{\Sym}{Sym}
\DeclareMathOperator{\tr}{tr}
\DeclareMathOperator{\img}{img}
\DeclareMathOperator{\coker}{coker}
\DeclareMathOperator{\Spec}{Spec}
\DeclareMathOperator{\CandE}{CandE}
\DeclareMathOperator{\CandO}{CandO}
\DeclareMathOperator{\argmax}{argmax}
\DeclareMathOperator{\first}{first}
\DeclareMathOperator{\last}{last}
\DeclareMathOperator{\cost}{cost}
\DeclareMathOperator{\dist}{dist}
\DeclareMathOperator{\path}{path}
\DeclareMathOperator{\parent}{parent}
\DeclareMathOperator{\argmin}{argmin}
\DeclareMathOperator{\excess}{excess}
\let\Pr\relax
\DeclareMathOperator{\Pr}{\mathbf{Pr}}
\DeclareMathOperator{\Exp}{\mathbb{E}}
\DeclareMathOperator{\Var}{\mathbf{Var}}
\let\limsup\relax
\DeclareMathOperator{\limsup}{limsup}
%Paired Delims
\DeclarePairedDelimiter\ceil{\lceil}{\rceil}
\let\oldceil\ceil
\renewcommand{\ceil}[1]{\oldceil*{#1}}

\DeclarePairedDelimiter{\floor}{\lfloor}{\rfloor}
\let\oldfloor\floor
\renewcommand{\floor}[1]{\oldfloor*{#1}}





\newcommand{\dagstar}{*}

\newcommand{\tbigwedge}{{\textstyle{\bigwedge}}}
\setlength{\parindent}{0pt}
\setlength{\parskip}{5pt}


\usepackage{listings}
\usepackage{courier}
\usepackage{microtype}


\lstset{
  basicstyle=\footnotesize\ttfamily,
  breaklines=true,
  breakatwhitespace=true
  columns=fullflexible,
  keepspaces=true,
  frame=single,
  escapeinside={(*@}{@*)}
}

\begin{document}

\title{Math 71: Abstract Algebra}

\author{Prishita Dharampal}
\date{}
\maketitle



\begin{problem}{1} For this lab assignment, you will be taking the provided assembly code listing and writing the equivalent in pseudocode followed by a few questions about the listing. The pseudocode doesn’t have to compile, but it should be understandable to someone who knows C.

\vspace{0.5em}

\textbf{Listing}

\vspace{0.5em}

The following listing was generated using \texttt{objdump -M intel -D <file>}.  
The first column is the instruction virtual address. The next column is the machine code, and the final column is the disassembled instruction.

\vspace{1em}

\lstset{
  basicstyle=\footnotesize\ttfamily,
  columns=fullflexible,
  keepspaces=true,
  frame=single
}

\begin{lstlisting}
0000000000001149 <sub_1149>:
1149: f3 0f 1e fa        endbr64
114d: 55                 push rbp
114e: 48 89 e5           mov rbp, rsp
1151: 89 7d fc           mov DWORD PTR [rbp-0x4], edi
1154: 89 75 f8           mov DWORD PTR [rbp-0x8], esi
1157: 8b 55 fc           mov edx, DWORD PTR [rbp-0x4]
115a: 8b 45 f8           mov eax, DWORD PTR [rbp-0x8]
115d: 01 d0              add eax, edx
115f: 5d                 pop rbp
1160: c3                 ret

\end{lstlisting}

\newpage
\begin{lstlisting}

0000000000001161 <sub_1161>:
1161: f3 0f 1e fa        endbr64
1165: 55                 push rbp
1166: 48 89 e5           mov rbp, rsp
1169: 48 83 ec 08        sub rsp, 0x8
116d: 89 7d fc           mov DWORD PTR [rbp-0x4], edi
1170: 81 7d fc ff 00 00 00
                         cmp DWORD PTR [rbp-0x4], 0xff
1177: 7e 11              jle 118a <sub_1161+0x29>
1179: 8b 45 fc           mov eax, DWORD PTR [rbp-0x4]
117c: be ad de 00 00     mov esi, 0xdead
1181: 89 c7              mov edi, eax
1183: e8 c1 ff ff ff     call 1149 <sub_1149>
1188: eb 0f              jmp 1199 <sub_1161+0x38>
118a: b8 fe ca 00 00     mov eax, 0xcafe
118f: 8b 75 fc           mov esi, DWORD PTR [rbp-0x4]
1192: 89 c7              mov edi, eax
1194: e8 b0 ff ff ff     call 1149 <sub_1149>
1199: c9                 leave
119a: c3                 ret
\end{lstlisting}

\begin{enumerate}
  \item Were there any instructions that you hadn't seen before?
  \item Using the Intel Instruction Set Reference, give a brief explanation of what they do.
  \item What is the return value if the argument to the \texttt{sub\_1161} function is \texttt{0x50}? \texttt{0x200}?
\end{enumerate}

\end{problem}

\lstset{
  basicstyle=\footnotesize\ttfamily
}

\begin{solution}
    \bbni 

    \text{Reconstructed Pseudocode:}
        \begin{lstlisting}
        int sub_1149(int a, int b) {
            return a + b;
        }
        \end{lstlisting}

        \begin{lstlisting}
        int sub_1161(int a) {
            if (a <= 0xff){
                return sub_1149(a, 0xcafe);
            }
            else {
                return sub_1149(0xdead, a);
            }

        }
        \end{lstlisting}
    
    \begin{enumerate}
        \item I wasn't familiar with the \texttt{endbr64} instruction.
        \item According to Intel's Website, the processor implements a state machine that tracks indirect \texttt{JMP} and \texttt{CALL} instructions. When \texttt{endbr64} is seen, the state machine moves from \texttt{IDLE} to \texttt{WAIT\_FOR\_ENDBRANCH} state.
        \item The return value for \texttt{sub\_1161(0x50)} is \texttt{0xcafe + 0x50 = 0xcb4e}. The return value for \texttt{sub\_1161(0x200)} is \texttt{0xdead + 0x200 = 0xe0ad}.
    \end{enumerate}

\end{solution}

\end{document}