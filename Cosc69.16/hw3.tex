\documentclass[12pt]{article}

\usepackage{fullpage}
\usepackage{mdframed}
\usepackage{colonequals}
\usepackage{algpseudocode}
\usepackage{algorithm}
\usepackage[most, breakable]{tcolorbox}
\usepackage[all]{xy}
\usepackage{proof}
\usepackage{mathtools}
\usepackage{bbm}
\usepackage{amssymb}
\usepackage{amsthm}
\usepackage{amsmath}
\usepackage{amsxtra}
\usepackage{enumitem}
\newcommand{\bb}{\mathbb}


\newtheorem{theorem}{Theorem}[section]
\newtheorem{theorem*}{Theorem}
\newtheorem{definition}[theorem]{Definition}
\newtheorem{corollary}{Corollary}[theorem]
\newtheorem{lemma}[theorem]{Lemma}
\newtheorem{prop}[theorem]{Proposition}
\newtheorem{remark}[theorem]{Remark}


\newtheorem*{exercisehelper}{Exercise.}
\newenvironment{exercise}[1]{%
  \IfBlankTF{#1}
    {\renewcommand{\exercisehelper}{\textbf{Exercise} \unskip}}
    {\renewcommand\exercisehelper{\textbf{Exercise #1}}}%
  \exercisehelper
}{\endexercisehelper}

\theoremstyle{remark}
\newtheorem*{solution}{Solution}
\newcommand{\mathcat}[1]{\textup{\textbf{\textsf{#1}}}} % for defined terms

\newenvironment{problem}[1]
{ \begin{tcolorbox}[breakable]\noindent\textbf{Problem #1}.}
{\vskip 6pt \end{tcolorbox}}

\newenvironment{enumalph}
{\begin{enumerate}\renewcommand{\labelenumi}{\textnormal{(\alph{enumi})}}}
{\end{enumerate}}

\newenvironment{enumroman}
{\begin{enumerate}\renewcommand{\labelenumi}{\textnormal{(\roman{enumi})}}}
{\end{enumerate}}

\newcommand{\defi}[1]{\textsf{#1}} % for defined terms



\setlength{\hfuzz}{4pt}

\let\H\relax
\let\P\relax
\newcommand{\H}{\mathbb H}
\newcommand{\P}{\mathbb P}
\newcommand{\C}{\mathbb C}
\newcommand{\N}{\mathbb N}
\newcommand{\Q}{\mathbb Q}
\newcommand{\R}{\mathbb R}
\newcommand{\Z}{\mathbb Z}
\newcommand{\F}{\mathbb F}
\newcommand{\br}{\mathbf{r}}
\newcommand{\RP}{\mathbb{RP}}
\newcommand{\CP}{\mathbb{CP}}
\newcommand{\nbit}[1]{\{0, 1\}^{#1}}
\newcommand{\bits}{\{0, 1\}^{n}}
\newcommand{\bbni}{\bigbreak \noindent}
\newcommand{\norm}[1]{\left\vert\left\vert#1\right\vert\right\vert}
\newcommand{\dbar}{\overline{\partial}}
\let\d\relax
\newcommand{\d}{\partial}
\newcommand{\calO}{\mathcal{O}}
\newcommand{\calF}{\mathcal{F}}
\newcommand{\calG}{\mathcal{G}}
\newcommand{\calH}{\mathcal{H}}
\newcommand{\calE}{\mathcal{E}}
\newcommand{\calC}{\mathcal{C}}
\newcommand{\calD}{\mathcal{D}}

\let\1\relax
\newcommand{\1}{\mathbf{1}}
\newcommand{\fr}[2]{\left(\frac{#1}{#2}\right)}
\newcommand{\todo}[1]{\textcolor{red}{\textbf{TODO:} #1}}
\newcommand{\vecz}{\mathbf{z}}
\newcommand{\vecr}{\mathbf{r}}
\DeclareMathOperator{\Cinf}{C^{\infty}}
\DeclareMathOperator{\Id}{Id}
\DeclareMathOperator{\Ell}{Ell}
\DeclareMathOperator{\CL}{\mathcal{CL}}

\DeclareMathOperator{\Alt}{Alt}
\DeclareMathOperator{\Aut}{Aut}
\DeclareMathOperator{\ann}{ann}
\DeclareMathOperator{\codim}{codim}
\DeclareMathOperator{\End}{End}
\DeclareMathOperator{\Hom}{Hom}
\DeclareMathOperator{\id}{id}
\DeclareMathOperator{\M}{M}
\DeclareMathOperator{\Mat}{Mat}
\DeclareMathOperator{\Ob}{Ob}
\DeclareMathOperator{\opchar}{char}
\DeclareMathOperator{\opspan}{span}
\DeclareMathOperator{\rk}{rk}
\DeclareMathOperator{\sgn}{sgn}
\DeclareMathOperator{\Sym}{Sym}
\DeclareMathOperator{\tr}{tr}
\DeclareMathOperator{\img}{img}
\DeclareMathOperator{\coker}{coker}
\DeclareMathOperator{\Spec}{Spec}
\DeclareMathOperator{\CandE}{CandE}
\DeclareMathOperator{\CandO}{CandO}
\DeclareMathOperator{\argmax}{argmax}
\DeclareMathOperator{\first}{first}
\DeclareMathOperator{\last}{last}
\DeclareMathOperator{\cost}{cost}
\DeclareMathOperator{\dist}{dist}
\DeclareMathOperator{\path}{path}
\DeclareMathOperator{\parent}{parent}
\DeclareMathOperator{\argmin}{argmin}
\DeclareMathOperator{\excess}{excess}
\let\Pr\relax
\DeclareMathOperator{\Pr}{\mathbf{Pr}}
\DeclareMathOperator{\Exp}{\mathbb{E}}
\DeclareMathOperator{\Var}{\mathbf{Var}}
\let\limsup\relax
\DeclareMathOperator{\limsup}{limsup}
%Paired Delims
\DeclarePairedDelimiter\ceil{\lceil}{\rceil}
\let\oldceil\ceil
\renewcommand{\ceil}[1]{\oldceil*{#1}}

\DeclarePairedDelimiter{\floor}{\lfloor}{\rfloor}
\let\oldfloor\floor
\renewcommand{\floor}[1]{\oldfloor*{#1}}





\newcommand{\dagstar}{*}

\newcommand{\tbigwedge}{{\textstyle{\bigwedge}}}
\setlength{\parindent}{0pt}
\setlength{\parskip}{5pt}


\usepackage{listings}
\usepackage{courier}
\usepackage{microtype}


\lstset{
  basicstyle=\footnotesize\ttfamily,
  breaklines=true,
  breakatwhitespace=true
  columns=fullflexible,
  keepspaces=true,
  frame=single,
  escapeinside={(*@}{@*)}
}

\begin{document}

\title{Math 71: Abstract Algebra}

\author{Prishita Dharampal}
\date{}
\maketitle



\begin{problem}{1}

\begin{lstlisting}
00000000000011a0 <f>:
11a0: 48 8b 16              mov    rdx,QWORD PTR [rsi]
11a3: 48 03 17              add    rdx,QWORD PTR [rdi]
11a6: 48 89 17              mov    QWORD PTR [rdi],rdx
11a9: 48 8b 06              mov    rax,QWORD PTR [rsi]
11ac: 48 29 d0              sub    rax,rdx
11af: 48 89 06              mov    QWORD PTR [rsi],rax
11b2: 48 01 07              add    QWORD PTR [rdi],rax
11b5: 48 f7 1e              neg    QWORD PTR [rsi]
11b8: c3                    ret
\end{lstlisting}
    For your convenience, here is the same function written in AT\&T assembly style:

\begin{lstlisting}
00000000000011a0 <f>:
11a0: 48 8b 16              mov    (%rsi),%rdx
11a3: 48 03 17              add    (%rdi),%rdx
11a6: 48 89 17              mov    %rdx,(%rdi)
11a9: 48 8b 06              mov    (%rsi),%rax
11ac: 48 29 d0              sub    %rdx,%rax
11af: 48 89 06              mov    %rax,(%rsi)
11b2: 48 01 07              add    %rax,(%rdi)
11b5: 48 f7 1e              negq   (%rsi)
11b8: c3                    ret
\end{lstlisting}

    \begin{enumerate}
        \item What does this function do?
        \item What does this function really do? What is its most likely intended effect? Assume the behavior that is the simplest to describe (though not on all values).
        \item \textit{(Extra credit; note change)} Does the simplest behavior break? If so, give examples. 
        \item \textbf{[Poll]} What looks better to you in this example: AT\&T assembly style or Intel assembly style?
    \end{enumerate}
\end{problem}

\begin{solution}
    \bbni 
    
    \begin{enumerate}
        \item  
        \begin{lstlisting}
(*@ \textcolor{red}{let rsi = x, rdi = y}@*)
00000000000011a0 <f>:
11a0: 48 8b 16              mov    (%rsi),%rdx (*@\textcolor{red}{ ; stores rsi value in rdx $\quad rdx = y$}@*)
11a3: 48 03 17              add    (%rdi),%rdx (*@\textcolor{red}{ ; adds rdi and rdx and stores sum in rdx $\quad rdx =  x + y$}@*)
11a6: 48 89 17              mov    %rdx,(%rdi) (*@\textcolor{red}{ ; moves rdx value to rdi $\quad rdi = x+y$}@*)
11a9: 48 8b 06              mov    (%rsi),%rax (*@\textcolor{red}{ ; stores rsi value in rax $\quad rax = y$}@*)
11ac: 48 29 d0              sub    %rdx,%rax (*@\textcolor{red}{ ; subtracts rdx from rax and stores in rax $\quad rax =  y - (x + y)$}@*)
11af: 48 89 06              mov    %rax,(%rsi) (*@\textcolor{red}{ ; stores rax value in rsi $\quad rsi = -x$}@*)
11b2: 48 01 07              add    %rax,(%rdi) (*@\textcolor{red}{ ; adds rax and rdi and stores in rdi $\quad rdi = (x+y) + (-x)$}@*)
11b5: 48 f7 1e              negq   (%rsi) (*@\textcolor{red}{ ; negates rsi value $\quad rsi = -(-x)$}@*)
11b8: c3                    ret
\end{lstlisting}

        The function swaps the \texttt{\%rsi} and \texttt{\%rdi} values using basic arithmatic. 
        \item The function's most likely intended behavior is to swap the two arguements without using a third (temporary) variable.
        \item  Since the function deals with qword arguments, it would break with a segmentation fault if someone passes a null value. The code also breaks for the arguments: \texttt{\%rsi = 3, \, \%rdi = 9223372036854775807}, outputting \texttt{ \%rsi = -1, \%rdi = 3}. 
        \item I like AT\&T more.
    \end{enumerate}
\end{solution}
\end{document}