\documentclass[12pt]{article}
\usepackage{graphicx}
\graphicspath{ {./images/} }

\usepackage{fullpage}
\usepackage{mdframed}
\usepackage{colonequals}
\usepackage{algpseudocode}
\usepackage{algorithm}
\usepackage[most, breakable]{tcolorbox}
\usepackage[all]{xy}
\usepackage{proof}
\usepackage{mathtools}
\usepackage{bbm}
\usepackage{amssymb}
\usepackage{amsthm}
\usepackage{amsmath}
\usepackage{amsxtra}
\usepackage{enumitem}
\newcommand{\bb}{\mathbb}


\newtheorem{theorem}{Theorem}[section]
\newtheorem{theorem*}{Theorem}
\newtheorem{definition}[theorem]{Definition}
\newtheorem{corollary}{Corollary}[theorem]
\newtheorem{lemma}[theorem]{Lemma}
\newtheorem{prop}[theorem]{Proposition}
\newtheorem{remark}[theorem]{Remark}


\newtheorem*{exercisehelper}{Exercise.}
\newenvironment{exercise}[1]{%
  \IfBlankTF{#1}
    {\renewcommand{\exercisehelper}{\textbf{Exercise} \unskip}}
    {\renewcommand\exercisehelper{\textbf{Exercise #1}}}%
  \exercisehelper
}{\endexercisehelper}

\theoremstyle{remark}
\newtheorem*{solution}{Solution}
\newcommand{\mathcat}[1]{\textup{\textbf{\textsf{#1}}}} % for defined terms

\newenvironment{problem}[1]
{ \begin{tcolorbox}[breakable]\noindent\textbf{Problem #1}.}
{\vskip 6pt \end{tcolorbox}}

\newenvironment{enumalph}
{\begin{enumerate}\renewcommand{\labelenumi}{\textnormal{(\alph{enumi})}}}
{\end{enumerate}}

\newenvironment{enumroman}
{\begin{enumerate}\renewcommand{\labelenumi}{\textnormal{(\roman{enumi})}}}
{\end{enumerate}}

\newcommand{\defi}[1]{\textsf{#1}} % for defined terms



\setlength{\hfuzz}{4pt}

\let\H\relax
\let\P\relax
\newcommand{\H}{\mathbb H}
\newcommand{\P}{\mathbb P}
\newcommand{\C}{\mathbb C}
\newcommand{\N}{\mathbb N}
\newcommand{\Q}{\mathbb Q}
\newcommand{\R}{\mathbb R}
\newcommand{\Z}{\mathbb Z}
\newcommand{\F}{\mathbb F}
\newcommand{\br}{\mathbf{r}}
\newcommand{\RP}{\mathbb{RP}}
\newcommand{\CP}{\mathbb{CP}}
\newcommand{\nbit}[1]{\{0, 1\}^{#1}}
\newcommand{\bits}{\{0, 1\}^{n}}
\newcommand{\bbni}{\bigbreak \noindent}
\newcommand{\norm}[1]{\left\vert\left\vert#1\right\vert\right\vert}
\newcommand{\dbar}{\overline{\partial}}
\let\d\relax
\newcommand{\d}{\partial}
\newcommand{\calO}{\mathcal{O}}
\newcommand{\calF}{\mathcal{F}}
\newcommand{\calG}{\mathcal{G}}
\newcommand{\calH}{\mathcal{H}}
\newcommand{\calE}{\mathcal{E}}
\newcommand{\calC}{\mathcal{C}}
\newcommand{\calD}{\mathcal{D}}

\let\1\relax
\newcommand{\1}{\mathbf{1}}
\newcommand{\fr}[2]{\left(\frac{#1}{#2}\right)}
\newcommand{\todo}[1]{\textcolor{red}{\textbf{TODO:} #1}}
\newcommand{\vecz}{\mathbf{z}}
\newcommand{\vecr}{\mathbf{r}}
\DeclareMathOperator{\Cinf}{C^{\infty}}
\DeclareMathOperator{\Id}{Id}
\DeclareMathOperator{\Ell}{Ell}
\DeclareMathOperator{\CL}{\mathcal{CL}}

\DeclareMathOperator{\Alt}{Alt}
\DeclareMathOperator{\Aut}{Aut}
\DeclareMathOperator{\ann}{ann}
\DeclareMathOperator{\codim}{codim}
\DeclareMathOperator{\End}{End}
\DeclareMathOperator{\Hom}{Hom}
\DeclareMathOperator{\id}{id}
\DeclareMathOperator{\M}{M}
\DeclareMathOperator{\Mat}{Mat}
\DeclareMathOperator{\Ob}{Ob}
\DeclareMathOperator{\opchar}{char}
\DeclareMathOperator{\opspan}{span}
\DeclareMathOperator{\rk}{rk}
\DeclareMathOperator{\sgn}{sgn}
\DeclareMathOperator{\Sym}{Sym}
\DeclareMathOperator{\tr}{tr}
\DeclareMathOperator{\img}{img}
\DeclareMathOperator{\coker}{coker}
\DeclareMathOperator{\Spec}{Spec}
\DeclareMathOperator{\CandE}{CandE}
\DeclareMathOperator{\CandO}{CandO}
\DeclareMathOperator{\argmax}{argmax}
\DeclareMathOperator{\first}{first}
\DeclareMathOperator{\last}{last}
\DeclareMathOperator{\cost}{cost}
\DeclareMathOperator{\dist}{dist}
\DeclareMathOperator{\path}{path}
\DeclareMathOperator{\parent}{parent}
\DeclareMathOperator{\argmin}{argmin}
\DeclareMathOperator{\excess}{excess}
\let\Pr\relax
\DeclareMathOperator{\Pr}{\mathbf{Pr}}
\DeclareMathOperator{\Exp}{\mathbb{E}}
\DeclareMathOperator{\Var}{\mathbf{Var}}
\let\limsup\relax
\DeclareMathOperator{\limsup}{limsup}
%Paired Delims
\DeclarePairedDelimiter\ceil{\lceil}{\rceil}
\let\oldceil\ceil
\renewcommand{\ceil}[1]{\oldceil*{#1}}

\DeclarePairedDelimiter{\floor}{\lfloor}{\rfloor}
\let\oldfloor\floor
\renewcommand{\floor}[1]{\oldfloor*{#1}}





\newcommand{\dagstar}{*}

\newcommand{\tbigwedge}{{\textstyle{\bigwedge}}}
\setlength{\parindent}{0pt}
\setlength{\parskip}{5pt}


\usepackage{listings}
\usepackage{courier}
\usepackage{microtype}


\lstset{
  basicstyle=\footnotesize\ttfamily,
  breaklines=true,
  breakatwhitespace=true
  columns=fullflexible,
  keepspaces=true,
  frame=single,
  escapeinside={(*@}{@*)}
}

\begin{document}

\newcommand{\HH}{\mathbb{H}}
\renewcommand{\ll}{\mathcal{L}}

\title{Math 71: Abstract Algebra}

\author{Prishita Dharampal}
\date{}
\maketitle


\textbf{Credit Statement:} Talked to Sair Shaikh'26, and Math Stack Exchange.
\\

\begin{problem}{1}
    An element $e \in  R$ is called an idempotent if $e^2 = e$. Assume $e$ is an idempotent in $R$ and $er = re$ for all $r \in R$. Prove that $Re$ and $R(1 - e)$ are two-sided ideals of $R$ and that $R \cong Re \times R (1-e)$. Show that $e$ and $1 - e$ are identities for the subrings $Re$ and $R(1 - e)$ respectively.
\end{problem}

\begin{solution}
    \bbni 
    \begin{enumerate}
        \item $Re$ is a two-sided ideal.  \\ 
        A two-sided ideal is an abelian subgroup under addition and closed under multiplication. 
        \begin{enumerate}
            \item Showing that $Re$ is an abelian subgroup: 
            \[Re = \{r_1e + r_2 e + \cdots + r_ne \mid \forall r_i \in R,\, n \in \Z^+ \}\]
            Using the distributive law: 
            \[Re = \{(r_1 + r_2  + \cdots + r_n)e \mid \forall r_i \in R,\, n \in \Z^+ \}\] 
            And $(R,+)$ is an abelian group.  
            \item Showing that $Re$ is closed under multiplication: \\ 
            Let $r \in R$, $r_ie \in R$, 
            \begin{enumerate}
                \item $r(r_ie) = (rr_i) e$, and by definition $rr_i  e \in Re$. 
                \item  $(r_ie)r = (r_ir)e$, again, by definition $r_ire \in Re$.
                \item $r_1e \cdot r_2 e = r_1r_2e^2 = r_1r_2e \in Re$, $\forall r_1e, r_2e \in Re$. 
            \end{enumerate}
        \end{enumerate}
        Hence, $Re$ a two-sided ideal.

        \item $e$ is an identity for $Re$. \\ 
        $\forall re \in Re, \, re. e = re^2 = re, \text{ and }  ere = re^2 = re$. Hence, the element is unchanged under multiplication by $e$, i.e., $e$ is an identity. 

        \item $R(1-e)$ is a two-sided ideal.  \\ 
        A two-sided ideal is an abelian subgroup under addition and closed under multiplication. 
        \begin{enumerate}
            \item Showing that $R(1-e_)$ is an abelian subgroup: 
            \[R(1-e) = \{r_1(1-e) + r_2 (1-e) + \cdots + r_n(1- e) \mid \forall r_i \in R,\, n \in \Z^+ \}\]
            But, using the distributive law: 
            \[Re = \{(r_1 + r_2  + \cdots + r_n)(1-e) \mid \forall r_i \in R,\, n \in \Z^+ \}\] 
            And $(R,+)$ is an abelian group.  
            \item Showing that $R(1-e)$ is closed under multiplication: \\ 
            Let $r \in R$, $r_i(1-e) \in R$, 
            \begin{enumerate}
                \item $r(r_i(1-e)) = (rr_i) (1-e)$, and by definition $rr_i (1- e) \in R(1-e)$. 
                \item  $(r_i(1-e))r =  (r_i - r_ie)r = r_i r - r_ire = r_ir(1-e) $, again, by definition $r_ir(1-e) \in R(1-e)$.
                \item $r_1(1-e), r_2(1-e) \in R(1-e)$
                \begin{align*}
                    r_1(1-e) \cdot r_2 (1-e) &= (r_1 - r_1e)(r_2 - r_2 e) \\ 
                    &= r_1r_2 - r_1r_2e - r_1er_2 + r_1r_2e^2  \\ 
                    &= r_1r_2 - r_1r_2e - r_1r_2e + r_1r_2e  \\ 
                    &= r_1r_2 (1-e) \in Re
                \end{align*} 
            \end{enumerate}
        \end{enumerate}
        Hence, $R(1-e)$ a two-sided ideal.

        \item $(1-e)$ is an identity for $R(1-e)$. \\ 
        We note the following, 
        \[(1-e) ^2 = 1 - e - e + e = 1-e \qquad(1-e)r = r - er = r - re = r(1-e)\]
        Then $\forall r(1-e) \in R(1-e), \, r(1-e). (1-e) = r(1-e), \text{ and }  (1-e)r(1-e) = r(1-e)^2 = r(1-e)$. Hence, the element is unchanged under multiplication by $(1-e)$, i.e., $(1-e)$ is an identity. 

        \item $R \cong Re \times R(1-e)$ \\ 
        We can see that the ideals $(e), (1-e)$ are comaximal  
        \[(e) + (1-e) = (e + 1 -e) = (1) = R\] 
        Then, by the Chinese Remainder Theorem, 
        \[R / (Re  \cap R(1-e)) \cong R / Re \times R / R(1-e)\]
        
        \textbf{Claim:} $Re \cap R (1-e) = {0}$. \\ 
        Assume $Re \cap R (1-e) \neq {0}$, then  there exists a non-zero $x \in Re \cap R (1-e)$. I.e $x = r_1 e = r_2 - r_2e$ for some $r_1, r_2 \in R$. Then, 
        \begin{align*}
            r_1e &= r_2(1-e) \\
            r_1e^2 &= r_2 (e - e^2) \\ 
            r_1e &= r_2 (0) = 0
        \end{align*}
        Hence, $x =0$, which is a contradiction since we assumed $x$ to be a non-zero element in the intersection. 

        \textbf{Claim:} $R/Re \cong R (1-e)$ \\ 
        Define the map $\varphi : R\to R(1-e)$, such that $r \to r (1-e)$. The kernel of this map $ker(\varphi) = \{a(1-e) = 0, \, \forall a\in R\}$. 
            \[a = a\cdot 1 = a \cdot (e + 1 - e) = ae + a (1-e) = ae\]
        $\implies ker(\varphi) \subseteq Re$. 
        For the reverse inclusion, 
        \[(re)(1-e) = re -re^2 = 0\]
        $\implies Re \subseteq ker(\varphi)$
        $\implies Re = ker(\varphi)$.  \\
        By the First Isomorphism Theorem, 
        \[R / Re \cong R(1-e) \]

        \textbf{Claim:} $R/R(1-e) \cong Re$ \\ 
        Define the map $\varphi : R\to Re$, such that $r \to r e$. The kernel of this map $ker(\varphi) = \{ae = 0, \, \forall a\in R\}$. 
            \[a = a\cdot 1 = a \cdot (e + 1 - e) = ae + a (1-e) = a(1-e)\]
        $\implies ker(\varphi) \subseteq R(1-e)$. 
        For the reverse inclusion, 
        \[(r(1-e)) e = re -re^2 = 0\]
        $\implies R(1-e) \subseteq ker(\varphi)$
        $\implies R(1-e) = ker(\varphi)$.  \\
        By the First Isomorphism Theorem, 
        \[R / R(1-e) \cong Re \]
        Then we can re-write 
            \[R / (Re  \cap R(1-e)) \cong R / Re \times R / R(1-e)\]
        as 
            \[R /\{0\} = R \cong R(1-e) \times Re\]
        Hence proved. 
    \end{enumerate}
\end{solution}

\newpage

\begin{problem}{2}
     Let $R$ and $S$ be rings with identities. Prove that every ideal of $R \times S$ is of the form $I  \times J$ where $I$ is an ideal of $R$ and $J$ is an ideal of $S$.
\end{problem}

\begin{solution}
    \bbni
    \bbni
    Let $K$ be an ideal in $R \times S$, where $K = \{(x,y)\}$. Then define $I, J$ to be the ideals generated by all elements in $K$, where $y=0$, and  $x = 0$ respectively. 
    \[I = \{(x: (x,0) \in K\} \qquad J = \{(y: (0,y) \in K\}\]
    \begin{enumerate}
        \item $K \subseteq I \times J$ \\
        Let $(x,y) \in K$. Since an ideal is multiplicatively closed, \[(x,y)(1,0) = (x, 0) \in K\] But $(x,0) \in I$ (by definition of I). Similarly,  \[(x,y)(0,1) = (0, y) \in K\] But $(0,y) \in J$ (by definition of J). 
        Then, $(x,y) \in I \times J$. 

        \item $I \times J \subseteq K$ \\ 
        Let $(x, y) \in I \times J$, where $x, y$ are generators. Then $(x, 0) \in K$, $(0, y) \in K$ (by definition). And since $K$ is closed under addition $(x,0) + ( 0,y) = ( x, y) \in K$. 

        Let $(x', y') \in I \times J$, where $(x', y')$ are arbitrary elements. By definition $x', y'$ are finite sums of the products of the generators of the ideal and the ring elements, 
        \[x' = \sum_{i=1} ^n r_ix_i \qquad y' =\sum_{i=1} ^n s_iy_i \]
        Then we can represent $(x'y')$ as 
        \[(x',y') = (r_1,0)(x_1, 0) + \cdots + (r_n,0)(x_n, 0) + (0, s_1)(0, y_1) + \cdots + (0, s_m)(0, y_m) \]
        I.e. $(x', y') \in K$. 
    \end{enumerate}
    Hence, $ K = I\times J$. Since, $K$ was an arbitrary ideal, this is true for any ideal of $R \times S$. 
\end{solution}

\newpage 

\begin{problem}{3}
    (A Public Key Code) Let $N$ be a positive integer. Let $M$ be an integer relatively prime to $N$ and let $d$ be an integer relatively prime to $\varphi(N)$, where $\varphi$ denotes Euler’s $\varphi$-function. Prove that if $M_1 \equiv M^d \pmod{N}$ then  $M \equiv M_1^{\,d'} \pmod{N}$ where $d'$ is the inverse of $d$ modulo $\varphi(N)$; that is, $dd' \equiv 1 \pmod{\varphi(N)}$.

    \medskip
    
    \noindent\textbf{Remark.} This result is the basis for a standard Public Key Code. Suppose $N = pq$ is the product of two distinct large primes (each on the order of 100 digits, for example). If $M$ is a message, then $M_1 \equiv M^d \pmod{N}$ is a scrambled (encoded) version of $M$, which can be unscrambled (decoded) by computing $M_1^{\,d'} \pmod{N}$. These powers can be computed quite easily even for
    large values of $M$ and $N$ by successive squarings. The values of $N$ and $d$ (but not $p$ and $q$) are made publicly known (hence the name), and then anyone with a message $M$ can send their encoded message $M^d \pmod{N}$. To decode the message it seems necessary to determine $d'$, which requires the determination of the value $\varphi(N) = \varphi(pq) = (p - 1)(q - 1)$ (no one has as yet proved that there is no other decoding scheme, however). The success of this method as a code rests on the necessity of determining the factorization of $N$ into primes, for which no sufficiently efficient algorithm exists. For example, the most naive method of checking all factors up to $\sqrt{N}$ would here require on the order of $10^{100}$ computations, or approximately 300 years even at 10 billion computations per second, and of course one can always increase the
    size of $p$ and $q$.
\end{problem}

\newpage

\begin{problem}{4}
    Let $R$ be the quadratic integer ring $\mathbb{Z}[\sqrt{-5}]$. Define the ideals 
    \[I_2 = (2,\; 1 + \sqrt{-5}), \qquad 
    I_3 = (3,\; 2 + \sqrt{-5}), \qquad 
    I_3' = (3,\; 2 - \sqrt{-5}).\]

    \begin{enumerate}
        \item[(a)] Prove that $I_2$, $I_3$, and $I_3'$ are nonprincipal ideals in $R$. 
        \item[(b)] Prove that the product of two nonprincipal ideals can be principal by showing that $I_2^2$ is the principal ideal generated by $2$, i.e. $ I_2^2 = (2)$.
        \item[(c)] Prove similarly that  $I_2 I_3 = (1 - \sqrt{-5}) \qquad \text{and} \qquad I_2 I_3' = (1 + \sqrt{-5})$  are principal ideals. Conclude that the principal ideal $(6)$ is the product of $4$ ideals: $(6) = I_2^2 I_3 I_3'$
    \end{enumerate}
\end{problem}

\begin{solution}
    \bbni
    \begin{enumerate}
        \item [(a)]
            \begin{enumerate}
                    \item [1.] Assume $I_2$ is principal. Then there must be some element $x$ that generates the whole ideal: \[I_2 = (2, 1+ \sqrt{-5}) = (x) \]
                    In particular, $x \mid 2$ and $x \mid (1 + \sqrt{-5}) \implies N(x) \mid N(2)$ and $N(x) \mid N(1 + \sqrt{-5})$. 
            
                    $N(2) = 4, N(1 - \sqrt{-5}) = 6$, i.e $N(x) = 1$ or $N(x) = 2$. If $N(x) = 1$, then $x$ is a unit and the ideal generated is the whole ring. Hence, a contradiction ($1 \notin I_2$). If $N(x) = 2$ then $x = a + b\sqrt{-5}$ such that $a^2 + 5b^2 = 2$, since $a^2, b^2 > 0$, there exist no integer solutions for this equation. I.e $N(x) \neq 2$. Since, neither of the solutions work $I_2$ is not principal. 
            
                    \item [2.] Similarly, assume $I_3$ is principal. Then there exists some element $x$ that generates the whole ideal: \[I_3 = (3, 2+ \sqrt{-5}) = (x) \]
                     In particular, $x \mid 3$ and $x \mid (2 + \sqrt{-5}) \implies N(x) \mid N(3)$ and $N(x) \mid N(2 + \sqrt{-5})$. 
                     $N(3) = 9, N(2 - \sqrt{-5}) = 9$, i.e $N(x) = 1$, $N(x) = 3$ $N(x) = 9$. If $N(x) = 1$, then $x$ is a unit and the ideal generated is the whole ring. Hence, a contradiction ($1 \notin I_2$). If $N(x) = 3$ then $x = a + b\sqrt{-5}$ such that $a^2 + 5b^2 = 3$, since $a^2, b^2 > 0$, there exist no integer solutions for this equation. $\implies N(x) \neq 3$. If $N(x) = 9$ then $x = a + b\sqrt{-5}$ such that $a^2 + 5b^2 = 9$. There are two integer solutions: 
                     \begin{enumerate}
                         \item $a = 3, b =0$ 
                         Then $I_3 = (3)$, which is not true because $2 + \sqrt{-5} \notin (3)$. 
                         \item $ a= \pm 2, b = \pm1$.  
                         Then $I_3 = (\pm 2 \pm \sqrt{ -5})$, which is not true because $3 \notin (\pm 2 \pm \sqrt{ -5})$ (Since $3 / (\pm 2 \pm \sqrt{ -5}) \notin R$). 
                     \end{enumerate}
                     Since, neither of the solutions work $I_3$ is not principal. 
            
                     \item [3.] The same argument holds for $I_3'$ because $N(2 + \sqrt{-5}) = N(2 - \sqrt{-5})$. 
            \end{enumerate}
    
        \item [(b)] $I_2^2 = I_2\cdot I_2 = ( 2 \cdot 2, 2 \cdot (1 + \sqrt{-5}),  (1 + \sqrt{-5}) \cdot (1 + \sqrt{-5})) = (4, \, 2 + 2 \sqrt{-5}, \, 4 + 2 \sqrt{-5})$. 
        Since the ideal is an abelian group, $I = (a, b) \supseteq ( a - b) \implies I_2^2 \supseteq (-2) = (2)$, And the opposite containment is also true because $2 \mid 4, \, 2 \mid (2 + \sqrt{-5}), \, 2 \mid (4 + 2 \sqrt{-5})$. 
        Hence, $I_2^2 = (2)$, and is principal. 

        \item [(c)] 
        \begin{enumerate}
            \item [(i)] $I_2I_3 = (1  - \sqrt{-5})$
            \begin{align*}
                I_2I_3 &= (2 \cdot 3, 2 \cdot (2 + \sqrt{-5}), (1 + \sqrt{-5}) \cdot 3, (1 + \sqrt{-5}) (2 + \sqrt{-5}) \\
                &= (6, \, 4 + 2 \sqrt{-5}, \, 3 + 3 \sqrt{-5}, -3 + 3\sqrt{-5})\\
            \end{align*}

        \begin{enumerate}
            \item $ 6 = (1 - \sqrt{-5}) (1 + \sqrt{-5})$ 
            \item $ 4 + 2 \sqrt{-5} = (1 - \sqrt{-5}) (-1 + \sqrt{-5})$ 
            \item $ 3 + 3 \sqrt{-5} = (1 - \sqrt{-5}) (-1 + 2\sqrt{-5})$ 
            \item $ -3 + 3 \sqrt{-5} = (1 - \sqrt{-5}) (-3)$ 
        \end{enumerate}
        Same as above, since the ideal is an abelian group, $I = (a, b) \supseteq ( a - b) \implies I_2I_3 \supseteq (1  - \sqrt{-5})$, And the opposite containment is also true because $(1  - \sqrt{-5})$ divides all the generators. 
        Hence, $I_2I_3 = (1  - \sqrt{-5})$, and is principal. 

        \item  [(ii)] $I_2I_3' = (1  + \sqrt{-5})$ 
        \begin{align*}
            I_2I_3' &= (2 \cdot 3, 2 \cdot (2 - \sqrt{-5}), (1 + \sqrt{-5}) \cdot 3, (1 + \sqrt{-5}) (2 - \sqrt{-5}) \\
            &= (6, \, 4 - 2 \sqrt{-5}, \, 3 + 3 \sqrt{-5}, 7 + \sqrt{-5})\\
        \end{align*}

        \begin{enumerate}
            \item $ 6 = (1 + \sqrt{-5}) (1 - \sqrt{-5}) $ 
            \item $ 4 - 2 \sqrt{-5} = (1 + \sqrt{-5}) (-3 - \sqrt{-5})$ 
            \item $ 3 + 3 \sqrt{-5} = (1+ \sqrt{-5}) (3)$ 
            \item $ 7  +  \sqrt{-5} = (1 - \sqrt{-5}) (2 - \sqrt{-5})$ 
        \end{enumerate}
        Same as above, since the ideal is an abelian group, $I = (a, b) \supseteq ( a - b) \implies I_2I_3' \supseteq (1  + \sqrt{-5})$, And the opposite containment is also true because $(1  + \sqrt{-5})$ divides all the generators. 
        Hence, $I_2I_3' = (1  + \sqrt{-5})$, and is principal. 

        \item [(iii)] Since $\Z[\sqrt{-5}]$ is commutative $I_2^2I_3I_3' = I_2I_3I_2I_3'$ and we know that  \[I_2I_3 = (1  - \sqrt{-5}) \quad \text { and } \quad I_2I_3' = (1  + \sqrt{-5})\]then $I_2I_3I_2I_3' = ((1 - \sqrt{-5}) (1 + \sqrt{-5})) =(6) $.
        \end{enumerate}
    \end{enumerate}
\end{solution}

\newpage

\begin{problem}{5}
    Prove that $(x, y)$ and $(2, x, y)$ are prime ideals in $\Z[x, y]$ but only the latter ideal is a maximal ideal. 
\end{problem}

\begin{solution}
    \bbni 
    \bbni 
    We know that an ideal $P$ is prime in a ring $R$ if $R/P$ is an integral domain, and the ideal is maximal if $R/P$ is a field. 
    \begin{enumerate}
        \item Consider the homomorphism $\varphi: \Z[x,y] \to \Z$ such that $\varphi (ax + by + c) = c$. It's kernel is all polynomials with zero constant term, i.e, precisely the ideal $(x, y)$. 
        \[\Z[x,y] / (x,y) \cong \Z\]
        Since, $\Z$ is an integral domain, $(x, y)$. However, because $\Z$ is not a field, $(x,y)$ is not maximal. 
        
        \item Consider the homomorphism $\varphi: \Z[x,y] \to \Z/2\Z$ such that $\varphi (ax + by + c) = c$ (mod 2). It's kernel is all polynomials with even constant term, i.e, precisely the ideal $(2, x, y)$.  \[\Z[x,y] / (x,y) \cong \Z/2\Z\]
        Since, $\Z/2\Z$ is field, $(2,x, y)$ is maximal and thus also prime.
    \end{enumerate} 

  
    
    % To check if $(x, y)$ is an ideal in $\Z[x,y]$, we need to check if $((x,y), +)$ is an abelian group and if $\forall a \in (x,y), \forall r\in \Z[x,y], ar, ra \in (x, y)$
    % \begin{enumerate}
    %     \item Identity: $0x + 0y \in (x,y)$ 
    %     \item Closed under addition and inverses: \\ 
    %     $\forall ax + by, \,cx +dy \in (x,y)$ we can see that 
    %     \[ax + by - (cx +dy) = (a-c) x -  (b-d) y\]
    %     which is an element of $(x,y)$. 
    %     \item Associativity:  \\
    %     $\forall ax + by, \,cx +dy, \, gx + fy \in (x,y)$
    %     \begin{align*}
    %         ax + by + (cx +dy + gx + fy) &= ax + by + (c + g) x + (d + f) y \\ 
    %         &= (a + c + g) x + (b + d + f) y \\ 
    %     \end{align*}
    %     and, 
    %     \begin{align*}
    %         (ax + by + cx +dy) + gx + fy &= (a + c )x + (b+d)y + gx + f y \\ 
    %         &= (a + c + g) x + (b + d + f) y \\ 
    %     \end{align*}

    %     \item Closure under multiplication with elements of $\Z[x,y]$: \\ 
    %     Let $ax + by \in (x, y),  \alpha \in \Z[x,y]$. Then $(ax + by) \alpha = ax \cdot \alpha + by \cdot \alpha$, and $\alpha(ax + by) = \alpha \cdot ax + \alpha \cdot by $. Since, $\Z[x,y]$ does not have negative powers of $x, y$, the only way these terms will not consist of a power of $x, y$ is if $\alpha$ is zero, in which case the whole term equals zero but zero is also in the ideal. 
    % \end{enumerate}
    % Hence, $(x,y)$ is an ideal. Now, upon quotient-ing $\Z[x,y]$ with $(x,y)$ the kernel consists of all terms with $x$ or $y$, i.e., all non constant terms. Then, $\Z[x,y] / (x,y) = \Z$, which is an integral domain. Hence, $(x,y)$ is a prime ideal in $\Z[x,y]$. However, because $\Z$ is not a field, $(x,y)$ is not maximal. 

    % Similarly, we can see that $(2, x, y)$ is an abelian group of polynomials in $x$ and $y$ with even constant terms. Checking for closure under multiplication with elements of $\Z[x,y]$: 
    %     Let $ax + by  + 2p\in (2, x, y),$ where $p$ is any integer, and $\alpha \in \Z[x,y]$. Then $(ax + by + 2p) \alpha = ax \cdot \alpha + by \cdot \alpha + 2p \cdot \alpha$, and $\alpha(ax + by) = \alpha \cdot ax + \alpha \cdot by + \alpha \cdot 2p $. Since, $\Z[x,y]$ does not have negative powers of $x, y$, the only way these terms will not consist of a power of $x, y$ is if $\alpha$ is zero, in which case the whole term equals zero but zero is also in the ideal. Also, the constant term will always be even because of the $2$ in the equation. 
        
        % Hence, $(2, x,y)$ is an ideal. Now, upon quotient-ing $\Z[x,y]$ with $(2, x,y)$ the kernel consists of all terms with $x$, $y$ or even constant terms. That leaves us with only odd constant terms in $\Z$. Then, $\Z[x,y] / (2, x,y) = \Z/2\Z$, which is a field. Hence, $(2, x,y)$ is a maximal ideal in $\Z[x,y]$. Since all fields are integral domains $(2, x, y)$ is also prime. 
\end{solution}

\newpage

\begin{problem}{6}

    Call a positive integer $n$ \emph{special} if there exists an integer $m$ with $1 < m < n$ such that
    \[
    1 + 2 + \cdots + (m-1) \;=\; (m+1) + \cdots + n.
    \]
    For example, $n = 8$ is special with $m = 6$, while $n = 7$ is not special. Find all positive integers that are special. \textbf{Hint.} Relate the pairs $(n,m)$ to integer solutions of 
    $a^2 - 2b^2 = 1$, i.e. units in $\mathbb{Z}[\sqrt{2}]$ of norm $1$. You can use the book’s description of the units in $\mathbb{Z}[\sqrt{2}]$.
\end{problem}

\begin{solution}
    \bbni 
    \bbni 
    Using the formula $\sum_{i=1}^n i = \frac{n(n+1)} {2}$ we get: 
    \[\frac{(m-1)m}{ 2} = \frac{n(n+1)}{2} - \frac{m(m+1)}{2} \implies n^2 + n - 2m^2 = 0 \]
    Completing the square: 
    \begin{align*}
         n^2 + n - 2m^2 &= 0 \\ 
          n^2 + n - \frac{1}{4} + \frac{1}{4}- 2m^2 &= 0 \\ 
          \left(n + \frac{1}{2} \right)^2 - 2m^2 &= \frac{1}{4} \\
          (2n +1)^2 - 2(2m)^2 &= 1
    \end{align*}

    Let $a = 2n +1$, $b = 2m$, then the equation looks like $a^2 - 2b^2 = 1$. \\  For any $\alpha = x + y \sqrt 2 \in \Z[\sqrt{2}]$, the norm looks like $(x + y\sqrt{2})(x-y\sqrt{2}) = x^2 - 2y^2$. Hence, $(a, b)$ are units (with norm +1) in $\Z[\sqrt{2}]$. We know that the full group of units of $\Z[\sqrt2] = \{\pm(1 + \sqrt{2})^n | n \in \Z\}$ (Pg. 230). The norm for the positive case, $(1 + \sqrt{2})^n$ is $-1$, and the norm for the negative case, $(-1  - \sqrt{2})^n$ is $1$. Hence, $(- 1 -  \sqrt{2})^n, n \in \Z$ are all solutions to our equation.
\end{solution}

\newpage

\begin{problem}{7}
    
    Prove the following presentations:
    \begin{enumerate}
        \item[(a)] $A_4 = \langle x, y \mid x^2 = y^3 = (xy)^3 = 1 \rangle$
        \item[(b)] $S_4 = \langle x, y \mid x^2 = y^3 = (xy)^4 = 1 \rangle$
        \item[(c)] $A_5 = \langle x, y \mid x^2 = y^3 = (xy)^5 = 1 \rangle$
    \end{enumerate}
\end{problem}

\begin{solution}
    \bbni 
    \begin{enumerate}
        \item[(a)] Consider the subgroup $H = \langle (12)(34), (123)\rangle \in A_4$. There are at least 7 distinct elements in this subgroup ($\{1, a, b, ab, ba, b^2, ab^2\}, a = (12)(34), b  = (123)$) by Lagrange's Theorem,  $H = A_4$. ($H \neq S_4$ because we cannot get a 4-cycle by multiplying a 2-2-cycle and a 3-cycle). Next, let $G$ be the group with the presentation $\langle x, y \mid x^2 = y^3 = (xy)^3 = 1 \rangle$. Then we can define a map $\varphi: G \to A_4$, which maps $x \to (12)(34)$, $y \to(123)$. 
        \begin{enumerate}
            \item $x^2 = (12)(34)(12)(34) = (12)^2 (34)^2 = 1$. 
            \item $y^3 = (123)^3 = 1$. 
            \item $(xy)^3 = ((12)(34)(123))^3 = ((1)(243))^3 = (243)^3 = 1$
        \end{enumerate}

        Since, the generators map to the generators and all of the relations hold, $\varphi$ is an isomorphism. I.e. $A_4 = \langle x, y \mid x^2 = y^3 = (xy)^3 = 1 \rangle$ is a valid presentation. 
        
        \item[(b)] We know that $S_n = \langle (12),(12\cdots n)\rangle \implies S_4 = \langle (12)(1234) \rangle$. Also note, $(12)(234) = (2341) = (1234)$, ie, we can write $S_4 = \langle (12), (234)\rangle$\\ 

        Next, let $G$ be the group with the presentation $\langle x, y \mid x^2 = y^3 = (xy)^4 = 1 \rangle$. Then we can define a map $\varphi: G \to S_4$, which maps $x \to (12)$, $y \to(234)$. Then we check the relations:  
         \begin{enumerate}
            \item $x^2 = (12)^2= 1$. 
            \item $y^3 = (234)^3 = 1$. 
            \item $(xy)^4 = ((12)(234))^4 = (1243)^4 = 1$
        \end{enumerate}

        Since, the generators map to the generators and all of the relations hold, $\varphi$ is an isomorphism. I.e. $S_4 = \langle x, y \mid x^2 = y^3 = (xy)^4 = 1 \rangle$ is a valid presentation. 

        
        \item[(c)] Consider the subgroup $H = \langle (12)(34), (135)\rangle \leq A_5$. $(12)(34) (135) = (14352)$, and $(135)(12)(34) = (12345)$. 

        Let $G$ be the group with the presentation $\langle x, y \mid x^2 = y^3 = (xy)^5 = 1 \rangle$. Then we can define a map $\varphi: G \to H$, which maps $x \to (12)(34)$, $y \to(135)$. 
        \begin{enumerate}
            \item $x^2 = (12)(34)(12)(34) = (12)^2 (34)^2 = 1$. 
            \item $y^3 = (135)^3 = 1$. 
            \item $(xy)^5 = ((12)(34)(135))^5 = (14352)^5 =  1$
        \end{enumerate}

        Then, $\varphi$ is a homomorphism. Since, the generators map to the generators and all of the relations hold, $\varphi$ is an
        isomorphism. 

        From the presentation, we know that $G$ contains an elements of order $2,3,5$. By Cauchy's Theorem, we know that the order of $G$ is at least $30$. Since, $G \leq A_5 \implies | G| \bigm | 60$. If $| G | =30 \implies [A_5 : G] = 2 $, implying that $A_5$ has normal subgroup, but $A_5$ is simple, hence, $|G| \neq 30$. Then, the subgroup $G$ must equal $A_5$.
        
    \end{enumerate}
\end{solution}

\newpage

\begin{problem} {8}

    For a field $F$, denote by $\mathrm{PGL}_2(F)$ and $\mathrm{PSL}_2(F)$ the quotients of 
    $\mathrm{GL}_2(F)$ and $\mathrm{SL}_2(F)$ by their respective normal subgroups consisting of the nonzero scalar matrices of the identity.
    \begin{enumerate}
        \item[(a)] Prove that for any field $F$, the kernel of the action of $\mathrm{GL}_2(F)$ on the set of 
        lines through the origin in $F^2$ is exactly the subgroup of nonzero scalar matrices. Deduce that the induced action of $\mathrm{PGL}_2(F)$ on the set of lines is faithful.
    
        \item[(b)] Prove that if $F_q$ is a finite field with $q$ elements, then the number of lines through the origin in the vector space $F_q^2$ is $q+1$.
    
        \item[(c)] Recall, from Problem Set \#2, the construction of the field $F_4$ of order $4$. Prove that
        \[\mathrm{PGL}_2(\F_4) \cong \mathrm{PSL}_2(\F_4) \cong \mathrm{SL}_2(\F_4) \cong A_5\]
        \textbf{Hint.} Consider the action on the set of lines through the origin in $\F_4^2$.
    
        \item[(d)] Prove that $\mathrm{PGL}_2(\F_5) \cong S_5$ and that $\mathrm{PSL}_2(\F_5) \cong A_5$.
        \textbf{Hint.} The action on the set of lines through the origin in the vector space $\F_5^2$ gives an injective homomorphism $\mathrm{PGL}_2(\F_5) \to S_6$. Count the number of $(2,2,2)$-cycles in $S_6$ not in the image, then let 
        $\mathrm{PGL}_2(\F_5)$ act on them by conjugation to obtain a new permutation representation.
    \end{enumerate}
\end{problem}

\begin{solution}
    \bbni
    \begin{enumerate}
        \item [(a)] Let $X$ be the set of lines passing through the origin in $F^2$. We can represent an element in this set (a line) as $a + \lambda b$, where $a,  b$ are points on the line and $\lambda \in F$. Since, the lines pass through the origin $a = 0$, and $b = \begin{pmatrix} p \\ q \end{pmatrix}$. Then let $\Vec{v} \in X, \vec{v} = \begin{pmatrix} \lambda p \\ \lambda q \end{pmatrix}$
        
        The action of $GL_2(F)$ on the set of lines through the origin in $F^2$, $X$, looks like $\varphi: GL_2(F) \times X \to X$. The kernel of this action $ker (\varphi)$ are elements in $GL_2(F)$ that fix all elements of $X$. (Note: fixing a line through the origin means it only scalar multiplies each vector.)

        Let $K \in ker (\varphi)$, then $K = \begin{pmatrix}
            k_1 & k _2 \\ k_3 & k_4
        \end{pmatrix}$, such that $Kx  = \alpha \vec{v}$, for some $\alpha$. Then consider the basis vectors:  
        \begin{enumerate}
            \item $\vec{v} = \begin{pmatrix}
                1 \\ 0
            \end{pmatrix}$, $K\vec{v} = \alpha \vec{v} \implies \begin{pmatrix}
            k_1 & k _2 \\ k_3 & k_4
            \end{pmatrix} \begin{pmatrix} 1 \\ 0 
            \end{pmatrix} = \begin{pmatrix}
                k_1  \\ k_3 
            \end{pmatrix} \implies  k_3 = 0$

            \item $\vec{v} = \begin{pmatrix}
                0 \\ 1
            \end{pmatrix}$, $K\vec{v} = \alpha \vec{v} \implies \begin{pmatrix}
            k_1 & k _2 \\ 0 & k_4
            \end{pmatrix} \begin{pmatrix} 0 \\ 1
            \end{pmatrix} = \begin{pmatrix}
                k_2  \\ k_4 
            \end{pmatrix} \implies  k_2 = 0$

            \item $\vec{v} = \begin{pmatrix}
                1 \\ 1
            \end{pmatrix}$, $K\vec{v} = \alpha \vec{v} \implies \begin{pmatrix}
            k_1 & 0 \\ 0 & k_4
            \end{pmatrix} \begin{pmatrix} 1 \\ 1
            \end{pmatrix} = \begin{pmatrix}
                k_1  \\ k_4 
            \end{pmatrix}  \implies \begin{pmatrix}
                k_1  \\ k_4 
            \end{pmatrix} = \alpha\begin{pmatrix}
                1  \\ 1 
            \end{pmatrix} \\ \\ \implies k_1 = k_4$
        \end{enumerate}

        I.e. any matrix $K \in ker(\varphi)$ is a scalar matrix; $ker(\varphi) \subseteq $ Set of Scalar Matrices. To show equality, we must show opposite containment: Set of Scalar Matrices $\subseteq ker (\varphi)$

        Let $A$ be a scalar matrix, then \\ 
        \[A = \begin{pmatrix}
            a & 0 \\ 0 & a 
        \end{pmatrix} \implies A\vec{v} =\begin{pmatrix}
            a & 0 \\ 0 & a 
        \end{pmatrix}\begin{pmatrix}
            x \\y
        \end{pmatrix} \implies \begin{pmatrix}
            ax \\ay 
        \end{pmatrix} \implies a\begin{pmatrix}
            x \\ y 
        \end{pmatrix} \implies A \in ker(\varphi) \]
        \\ 
        Since $A$ was an arbitrary scalar matrix, we can say that the \\ Set of Scalar Matrices $\subseteq ker (\varphi) \implies$ Set of Scalar Matrices $= ker (\varphi)$. 

        By the First Isomorphism Theorem, 
        \[GL_2(F) / ker (\varphi) \cong im(\varphi)\]
        
        Since, $PGL_2(F)$ is the quotient of $GL_2(F)$ by it's normal subgroup consisting of non-zero scalar matrices, we can re-write the above equation as, 
        \[GL_2(F) / ker (\varphi) = PGL_2(F)  \cong im(\varphi)\]

        I.e the kernel of the induced action of $PGL_2(F)$ on $X$ is trivial, hence the action is faithful. 
                
        \item [(b)] A line in $F_q^2$ can be represented as $a + \lambda b$, where $a,  b$ are two points on the line, and $\lambda \in F_q$. If the line passes through the origin, we know that one of the points is $(0,0) \implies p = 0$.
        That is, all lines passing through the origin in $F_q^2$ can be represented as $\lambda b$. Since all lines intersect at $(0,0)$ we know that a given line has $(q-1)$ unique points.  
        
        There are $q^2$ points in $F_q^2$. Excluding $(0,0)$, there are $q^2 - 1$ points. Then we can determine the number of unique lines through the origin in $F_q^2$ by $(q^2-1)/(q-1) = q + 1$. 
        Hence proved. 

        \item [(c)] From Problem Set 2, $\F_4 = \{0, 1, x, y   \mid x^2 = y, y^2 = x, 1+x = y, 1+y =x\}$
        \begin{enumerate}
            \item [(i)] $PSL_2(\F_4) \cong A_5$ \\
            We know that the order of $GL_2(F_q) =  (q^{2}-1)(q^{2}-q)$, where $F_q$ is a finite field. Order of $GL_2(\F_4) = 180$. 
            Using part $(a)$, the kernel of the action of $GL_2(\F_4)$ on the set of lines through the origin in $\F_4^2$ is the subgroup of nonzero scalar matrices, $ker =  \{ \lambda I, \lambda \in \F_4^\times \} \implies | ker | = 3$.

            $GL_2(\F_4)/ker \cong PGL_2(\F_4)$ (by definition of $PGL_2(F)$).  \\ 
            $|PGL_2(\F_4) | = |GL_2(\F_4)| / 3 = 60$. 

            Consider the action of $PGL_2(\F_4)$ on the set of all lines through the origin in $\F_4^2$. From part (b), we know that $\F_4^2$ has 5 lines. We then define a homomorphism $\varphi: PGL_2(\F_4) \to S_5$ with trivial kernel (the action is faithful, part(a)). 

            
            By the First Isomorphism Theorem, $PGL_2(\F_4)$ is isomorphic to a subgroup of $S_5$ of order 60. Since $A_5$ is the only one subgroup of $S_5$ of order 60, we can say $PSL_2(\F_4) \cong A_5$. 

            
            \item [(ii)] $PSL_2(\F_4) \cong SL_2(\F_4)$ \\ 

            $PSL_2(\F_4)$ by definition is the the quotient of $SL_2(\F_4)$ by it's normal subgroup consisting of the non-zero scalar matrices of the identity. 
            
            $ det(\lambda I) = \lambda^2 det (I)$, $0^2, \, x^2 , \, y^2 \neq 1 \implies \lambda  =1 $

            The only non-zero scalar matrix in $SL_2(\F_4)$ is the identity matrix. Then the normal group that $SL_2(\F_4)$ will be quotiented by is the trivial subgroup.
            \[SL_2(\F_4) / \{I\}  \cong SL_2(\F_4)  = PSL_2(\F_4) \]

            \item [(iii)] $PGL_2(\F_4) \cong PSL_2(\F_4)$ \\ 
            $PGL_2(\F_4)$ is defined as $GL_2(\F_4) / \langle \lambda I\rangle$. Let $[A]$ be a coset in $PGL_2(\F_4)$, $[A] = A \lambda I$. 
            \[det(A\lambda I) = det(A) \cdot det (\lambda I) = det(A) \cdot \lambda^2  \]
            Let the representative of the equivalence class $[A]$ be $A\lambda$, where $\lambda = \sqrt{\frac{1}{det(A)}}$, since every element has a square root in $\F_4$. Then the determinant of $[A], \forall[A] \in PGL_2(\F_4)$ is 1. 
            Then we can define a homomorphism $\varphi: PGL_2(\F_4) \to PSL_2(\F_4)$, mapping $[A]$ to the corresponding matrix in $PSL_2(\F_4)$. The kernel of this homomorphism $ker (\varphi) = \{\lambda I  = I\} \implies \lambda =1$. Hence, the kernel is trivial. 
            
            The order of $SL_2(F_q) =q^3 -q \implies  SL_2(\F_4) = 60 \implies PSL_2(\F_4) = 60$. Also from part (i), we know that the order of $PSL_2(\F_4)$ is 60. Hence, we can say that $\varphi$ is an isomorphism, $PGL_2(\F_4) \cong PSL_2(\F_4)$. 
        \end{enumerate}

        \[PGL_2(\F_4) \cong PSL_2(\F_4) \cong SL_2(\F_4) \cong A_5\]

    \end{enumerate}
\end{solution}

\newpage


\begin{problem}{9}

Some linear algebra over the field of order $9$.
\begin{enumerate}
    \item[(a)] Prove that $\F_3[i] = \{0, \pm1, \pm i, \pm1 \pm i\}$, where $i^2 =-
    1$ and all other arithmetic is done modulo 3, is a field of order 9, which we call $\F_9$.

    \item[(b)] Prove that $\mathrm{PGL}_2(\F_9)$ is not isomorphic to $S_6$, even though they have the same order. \textbf{Hint.} Use linear algebra to bound the sizes of certain conjugacy classes in  $\mathrm{PGL}_2(\F_9)$ and compare to what is known in $S_6$.

    \item[(c)] Prove, on the other hand, that $\mathrm{PSL}_2(\F_9) \cong A_6$. \\
    \textbf{Hint.} Find a subgroup of $\mathrm{PSL}_2(\F_9)$ isomorphic to $A_5$, then let it act on its  $6$ cosets to obtain a permutation representation.
\end{enumerate}
\end{problem}

\begin{solution}
    \bbni
    \begin{enumerate}
        \item [(a)] Consider the Euclidean Domain $\Z[i]$, and the ideal generated by $3, (3)$. We know that all EDs are PIDs (Proposition 1, Chapter 8). Then to show that $(3)$ is maximal we only need to prove that it is prime. Assume $3$ is not irreducible, i.e. 
        \[\exists a + bi, c+di \in \Z[i]: (a + bi)(c +di) = 3 \implies N(a+bi)N(c + di) = N(3) = 9 = (a^2 + b^2) ( c^2 + d^2)\]
        Then, wlog, the possible values for $a^2 + b^2, c^2 + d^2$ are: $1,9$ or $3,3$. WLOG, we can see that $a^2 + b^2 \neq 3, a,b \in \Z$. However, if $a^2 + b^2 = 1$, then $a + bi$ is a unit. Thus, a contradiction. 

        $3$ is prime in $\Z[i]$, and hence the ideal generated by $3$ is maximal. 

        We know that the quotient of a ring by a maximal ideal is a field. An element in the quotient $\Z[i]/(3)$ looks like $a + bi$ where $a, b \in \Z/3\Z$. That is, 
        \begin{align*}
            \Z[i] / (3) &= \{0, 1, 2, i, 2i, 1 + i, 1 + 2i, 2 + i, 2+ 2i  \} 
            \\ &= \{0, 1, -1, i, -i, 1 + i, 1 - i, -1 + i, -1 - i  \}, \quad 2 \equiv -1  \text { (mod 3)} \\
            &= \{0, \pm1, \pm i, \pm1 \pm i\} 
        \end{align*}

        where $i^2 = -1$. Then we can say that $\Z[i]/(3) \cong \F_3[i]$. Also, by counting, we can see $\mid \F_3 [i] \mid = 9$. \\ 

        Hence, $F_3[i]$ is a field of order 9, where arithmetic is done modulo 3 and $i^2 = -1$. 
        
    \end{enumerate}
\end{solution}

\end{document}