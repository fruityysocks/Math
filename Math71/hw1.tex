\documentclass[12pt]{article}

\usepackage{fullpage}
\usepackage{mdframed}
\usepackage{colonequals}
\usepackage{algpseudocode}
\usepackage{algorithm}
\usepackage[most, breakable]{tcolorbox}
\usepackage[all]{xy}
\usepackage{proof}
\usepackage{mathtools}
\usepackage{bbm}
\usepackage{amssymb}
\usepackage{amsthm}
\usepackage{amsmath}
\usepackage{amsxtra}
\usepackage{enumitem}
\newcommand{\bb}{\mathbb}


\newtheorem{theorem}{Theorem}[section]
\newtheorem{theorem*}{Theorem}
\newtheorem{definition}[theorem]{Definition}
\newtheorem{corollary}{Corollary}[theorem]
\newtheorem{lemma}[theorem]{Lemma}
\newtheorem{prop}[theorem]{Proposition}
\newtheorem{remark}[theorem]{Remark}


\newtheorem*{exercisehelper}{Exercise.}
\newenvironment{exercise}[1]{%
  \IfBlankTF{#1}
    {\renewcommand{\exercisehelper}{\textbf{Exercise} \unskip}}
    {\renewcommand\exercisehelper{\textbf{Exercise #1}}}%
  \exercisehelper
}{\endexercisehelper}

\theoremstyle{remark}
\newtheorem*{solution}{Solution}
\newcommand{\mathcat}[1]{\textup{\textbf{\textsf{#1}}}} % for defined terms

\newenvironment{problem}[1]
{ \begin{tcolorbox}[breakable]\noindent\textbf{Problem #1}.}
{\vskip 6pt \end{tcolorbox}}

\newenvironment{enumalph}
{\begin{enumerate}\renewcommand{\labelenumi}{\textnormal{(\alph{enumi})}}}
{\end{enumerate}}

\newenvironment{enumroman}
{\begin{enumerate}\renewcommand{\labelenumi}{\textnormal{(\roman{enumi})}}}
{\end{enumerate}}

\newcommand{\defi}[1]{\textsf{#1}} % for defined terms



\setlength{\hfuzz}{4pt}

\let\H\relax
\let\P\relax
\newcommand{\H}{\mathbb H}
\newcommand{\P}{\mathbb P}
\newcommand{\C}{\mathbb C}
\newcommand{\N}{\mathbb N}
\newcommand{\Q}{\mathbb Q}
\newcommand{\R}{\mathbb R}
\newcommand{\Z}{\mathbb Z}
\newcommand{\F}{\mathbb F}
\newcommand{\br}{\mathbf{r}}
\newcommand{\RP}{\mathbb{RP}}
\newcommand{\CP}{\mathbb{CP}}
\newcommand{\nbit}[1]{\{0, 1\}^{#1}}
\newcommand{\bits}{\{0, 1\}^{n}}
\newcommand{\bbni}{\bigbreak \noindent}
\newcommand{\norm}[1]{\left\vert\left\vert#1\right\vert\right\vert}
\newcommand{\dbar}{\overline{\partial}}
\let\d\relax
\newcommand{\d}{\partial}
\newcommand{\calO}{\mathcal{O}}
\newcommand{\calF}{\mathcal{F}}
\newcommand{\calG}{\mathcal{G}}
\newcommand{\calH}{\mathcal{H}}
\newcommand{\calE}{\mathcal{E}}
\newcommand{\calC}{\mathcal{C}}
\newcommand{\calD}{\mathcal{D}}

\let\1\relax
\newcommand{\1}{\mathbf{1}}
\newcommand{\fr}[2]{\left(\frac{#1}{#2}\right)}
\newcommand{\todo}[1]{\textcolor{red}{\textbf{TODO:} #1}}
\newcommand{\vecz}{\mathbf{z}}
\newcommand{\vecr}{\mathbf{r}}
\DeclareMathOperator{\Cinf}{C^{\infty}}
\DeclareMathOperator{\Id}{Id}
\DeclareMathOperator{\Ell}{Ell}
\DeclareMathOperator{\CL}{\mathcal{CL}}

\DeclareMathOperator{\Alt}{Alt}
\DeclareMathOperator{\Aut}{Aut}
\DeclareMathOperator{\ann}{ann}
\DeclareMathOperator{\codim}{codim}
\DeclareMathOperator{\End}{End}
\DeclareMathOperator{\Hom}{Hom}
\DeclareMathOperator{\id}{id}
\DeclareMathOperator{\M}{M}
\DeclareMathOperator{\Mat}{Mat}
\DeclareMathOperator{\Ob}{Ob}
\DeclareMathOperator{\opchar}{char}
\DeclareMathOperator{\opspan}{span}
\DeclareMathOperator{\rk}{rk}
\DeclareMathOperator{\sgn}{sgn}
\DeclareMathOperator{\Sym}{Sym}
\DeclareMathOperator{\tr}{tr}
\DeclareMathOperator{\img}{img}
\DeclareMathOperator{\coker}{coker}
\DeclareMathOperator{\Spec}{Spec}
\DeclareMathOperator{\CandE}{CandE}
\DeclareMathOperator{\CandO}{CandO}
\DeclareMathOperator{\argmax}{argmax}
\DeclareMathOperator{\first}{first}
\DeclareMathOperator{\last}{last}
\DeclareMathOperator{\cost}{cost}
\DeclareMathOperator{\dist}{dist}
\DeclareMathOperator{\path}{path}
\DeclareMathOperator{\parent}{parent}
\DeclareMathOperator{\argmin}{argmin}
\DeclareMathOperator{\excess}{excess}
\let\Pr\relax
\DeclareMathOperator{\Pr}{\mathbf{Pr}}
\DeclareMathOperator{\Exp}{\mathbb{E}}
\DeclareMathOperator{\Var}{\mathbf{Var}}
\let\limsup\relax
\DeclareMathOperator{\limsup}{limsup}
%Paired Delims
\DeclarePairedDelimiter\ceil{\lceil}{\rceil}
\let\oldceil\ceil
\renewcommand{\ceil}[1]{\oldceil*{#1}}

\DeclarePairedDelimiter{\floor}{\lfloor}{\rfloor}
\let\oldfloor\floor
\renewcommand{\floor}[1]{\oldfloor*{#1}}





\newcommand{\dagstar}{*}

\newcommand{\tbigwedge}{{\textstyle{\bigwedge}}}
\setlength{\parindent}{0pt}
\setlength{\parskip}{5pt}


\usepackage{listings}
\usepackage{courier}
\usepackage{microtype}


\lstset{
  basicstyle=\footnotesize\ttfamily,
  breaklines=true,
  breakatwhitespace=true
  columns=fullflexible,
  keepspaces=true,
  frame=single,
  escapeinside={(*@}{@*)}
}

\begin{document}

\title{Math 71: Abstract Algebra}

\author{Prishita Dharampal}
\date{}
\maketitle


Sources: Talked to Sair Shaikh '26, Frank Gallo'27, and Math Stack Exchange. 
\begin{problem}{1}
    Let $G$ be a group and $a_1, a_2, \ldots, a_r \in G$. We say that $a_1, \ldots, a_r$ pairwise commute if $a_i$ commutes with $a_j$ for all $i$ and $j$. We say that $a_1, \ldots, a_r$ are rank independent if $a_1^{e_1} \ldots a_r^{e_r} =1$ implies that $e_i$ is a multiple of $\mid a_i \mid$ for all i. The aim of this problem is to prove: 
    \begin{prop}
        Let $G$ be a group and $a_1, a_2, \ldots a_r \in G$ be pairwise commuting rank independent elements of finite order. Then $\mid a_1 \ldots a_r \mid = lcm(\mid a_1 \mid, \ldots, \mid a_r \mid)$. 
    \end{prop}
    \begin{enumerate}
        \item (DF 1.1 Exercise 24) If $a$ and $b$ are commuting elements, prove that $(ab) ^n = a^nb^n$ for all $n \in \Z$. Hint: Do induction on $n$.
        \item If $a_1, \ldots, a_r$ are pairwise commuting elements, prove that $(a_1\ldots a_r)^n = a_1^n \ldots a_r^n$. Hint: Do induction on $r$.
        \item If $a_1, \ldots, a_r$ are pairwise commuting elements of finite order (not necessarily rank independent), prove that $\mid a_1 \ldots a_r \mid$ divides $lcm(\mid a_1 \mid, \ldots, \mid a_r \mid)$. Hint: Raise $a_1 \ldots a_r$ to the power $lcm(\mid a_1 \mid, \ldots, \mid a_r \mid)$. 
        \item Prove the proposition. Hint: Do induction on $r$; for the base case $r = 1$ there is not much to say, and then you should realize that (after a bit of juggling with least common multipliers) the induction step just boils down to the case $r = 2$. Hint (for a different proof): Use the above characterization of the lcm to prove that $lcm(\mid a_1 \mid, \ldots, \mid a_n \mid)$ divides $\mid a_1 \ldots a_n \mid$. In any method you choose, be sure to highlight where the rank independence condition is used!
        \item Show that disjoint cycles in $S_n$ are rank independent, then deduce DF 1.3 Exercise 15.
    \end{enumerate}
\end{problem}

\newpage
\begin{solution}
    \bbni
    \bbni
    \begin{enumerate}
        \item If $a$ and $b$ are commuting elements, then 
        \begin{align*}
            (ab) ^ n &= \underbrace{(ab)(ab)(ab)\ldots(ab)}_{n-times} \\
            (ab)^n &= \underbrace{a(ba)b(ab)\ldots(ab)}_{n-times} \\
            (ab)^n &= \underbrace{a(ab)b(ab)\ldots(ab)}_{n-times} \\
            (ab)^n &= \underbrace{a^2b^2(ab)\ldots(ab)}_{n-times} \\
            (ab)^n &= \underbrace{a^2ab^2b\ldots(ab)}_{n-times} \\
        \end{align*}
        \text{if we commute all of the elements to have all $a$ together, we will be left with}\\
        \begin{align*}
            (ab)^n &= a^nb^n \\
        \end{align*}

        \item If all elements in $a \in G$ are pairwise commuting, then let $n$ be the product of all $a \in G$ such that: 
        \[n = a_1a_2\ldots a_r\]
        then because $a_i$ commutes with $a_j$ for all $i$ and $j$, 
        \begin{align*}
        \implies n &= a_2a_1a_3\ldots a_r \\
        \implies n &= a_2a_3a_1\ldots a_r \\
        \end{align*}
        upon doing this $r$ times, 
        \[n = a_2a_3\ldots a_ra_1\]
        and one could do this process $n$ times to move any element $n$ places in the equation, so we can see that the group $G$ is abelian. So we can say, 
        \begin{align*}
            (a_1\ldots a_r)^n &= \underbrace{(a_1\ldots a_r)(a_1\ldots a_r)}_{n-times}\\
            \implies (a_1\ldots a_r)^n &= (\underbrace{a_1\ldots a_1}_{\text{n-times}})(\underbrace{a_2\ldots a_2}_{\text{n-times}})\ldots (\underbrace{a_r\ldots a_r}_{\text{n-times}}) \\
            \implies (a_1\ldots a_r)^n &= a_1^n\ldots a_r^n
        \end{align*}
        \item Let $e_i, 1\leq i \leq r$ be the orders of elements $a_i$, $n = a_1\ldots a_r$, $k = \mid n \mid$, then: 
        \begin{align*}
            n^k &= (a_1\ldots a_r)^k = 1 \\
            n^k &= a^k_1\ldots a_r^k = 1 \\ 
        \end{align*}
        We can say that $k = lcm (\mid a_1 \mid \ldots \mid a_r\mid)b, b \in \Z$
        
        
        
        Let $e_i, 1\leq i \leq r$ be orders of elements from $a_1$ to $a_r$ and let $n = a_1\ldots a_r$. 
        F
        Also, we know that $a_i^{e_i} = a_i^{be_i} = 1, b \in \Z^+$ so, 
        \[e = e_1\ldots e_r\]
        \[n^e = (a_1\ldots a_r)^e = 1\]

        for all unique $e_i$ because for any $\mid a_i \mid = \mid a_j \mid = e_i, (a_ia_j)^{e_i} = a_i^{e_i}a_j^{e_i} = 1$. Essentially, $e$ is the lcm of $e_1, \ldots, e_r$. 
        
        Also, given that $a_1\ldots a_r$ are not necessarily rank independent there might exist, for some $a_i$ or combination of $a_i$s, an inverse in $a_1\ldots a_r$. Such that for some $a_i, a_j \in n, a_ia_j = 1$ (and their relative positions don't matter because the group is abelian). 
        % idk how to write that if that is the case then e might be smaller than product of all orders because the elements would cancel each other out and then e would be smaller than the lcm of the orders. either ways lcm of the orders would be divisible by e

        \item From above we can see that if all $a_i \in G$ are pairwise commuting, rank independent elements of finite order then: 
        \begin{align*}
            e & = lcm(e_1, \ldots, e_r) \\ 
            \mid a_1 \ldots a_r \mid &= lcm (\mid a_1\mid \ldots \mid a_r \mid). 
        \end{align*}

        \item Let $\sigma = m_1m_2\ldots m_n$, where $m_i$ is a cycle, and $n$ is the total number of cycles, be the permutation of disjoint cycles in $S_n.$ Also, by definition if the cycles are disjoint then they contain no common elements, which means that they don't interact with each other. Thus there are no possible inverse pairs $\in \sigma$. If  $\sigma^x = 1$ then $(m_i)^x = 1, \forall i$, so $\mid m_i \mid \large \mid x$. \\
        So, the cycles are rank independent.  \\ 
        Disjoint cycles are commutative, using subpart (4) we can say that because $\sigma$ is rank indepedent and commutative, then $x = lcm (\mid m_i\mid), \forall i$, and $m_i$. The order of each cycle is equal to it's length. So $x = lcm(\text{lengths of the disjoint cycles)}.$
        \[\]

        % no possible inverses cause no common elements => for the cycles to = 1 they all respectively have to be 1 => 
        
    \end{enumerate}
\end{solution}

\newpage
\begin{problem}{2}
        $D_{2n} = \langle r, s \mid r^n =s^2 =1, rs = sr^{-1} \rangle$ \\
    Use the generators and relations above to show that every element of $D_{2n}$ which is not a power of $r$ has order 2. Deduce that $D_{2n}$ is generated by the two elements $s$ and $sr$, both of which have order 2. 
\end{problem}

\begin{solution}
\bbni \\
    Elements in $D_{2n}$ have the following orders: 
    \begin{enumerate}
        \item Given, $\mid r \mid = n$. \\
        \item Given, $\mid s \mid = 2$. \\
        \item All other elements are multiples of $rs$, let $n$ be the order of $rs,$ and let $n$ be even: \\
        \begin{align*}
            e  &= (rs)^n\\  
            &= \underbrace{(rs)(rs)\ldots(rs)}_{n-times} \\ 
            &= \underbrace{(sr^{-1})(rs)(sr^{-1})(rs) \ldots (rs)}_{n-times} \qquad \text{(replacing every alternate elements with $sr^{-1}$}) \\
            &= \underbrace{s(r^{-1}r)(ss)(r^{-1}r)s \ldots rs)}_{n-times} \\
            &= ses = ss = s^2 
        \end{align*} 
        $\implies \mid rs \mid \large\mid n$ \\
        The smallest $n$ possible is $2$, and we know that $n \neq 1$ because if $x^1 = e \implies x = e$, and $rs \neq e$. Thus, $\mid rs \mid = 2$. 
        Hence, all elements of $D_{2n}$ that are not powers of $r$ have  order $2$.
    \end{enumerate}

        \bbni 

        For $D_{2n} = \langle r, s \mid r^n = s^2  = 1, rs = sr^{-1}\rangle$ to be generated by elements $s, sr$ we should be able to obtain the relations in the presentation from the new generators. \\ 
        So, 
        \begin{enumerate}
            \item $r = s(sr) \implies r^n = s^n (sr)^n = 1$
            \item $s^2 = s \cdot s = 1$ 
            \newpage
            \item $s(s^n(sr)^n)  = s(s^{2n}r^n) = sr^n\implies s(s^{-1}(sr)^{-1}) = s(er^{-1}) = sr^{-1}$ \\ 
            Then the relation, $rs = sr^{-1}$ can be expressed as: 
            \[s\cdot sr \cdot s = s(s^{-1}(sr)^{-1})\] 
            \[ers = s(s^{-1}s^{-1}r^{-1}) = s(s^{2\cdot -1}r^{-1})  = s(er^{-1})\]
            \[rs = sr^{-1}\]            
        \end{enumerate}
        Hence, $D_{2n} = \langle s, sr \mid r^n = s^2 = 1, rs - sr^{-1}\rangle.$
\end{solution}

 \newpage
\begin{problem}{3}
    Show that $\langle a, b \mid a^2 = b^2 = (ab)^n = 1 \rangle$ gives a presentation $D_{2n}$ in terms of the two generators $a = s$ and $b=sr$ of order 2 computed in the question above. [Show that the relations for $r$ and $s$ follow from the relations for $a$ and $b$ and, conversely, the relations for $a$ and $b$ follow from those for $r$ and $s$.]
\end{problem}

\begin{solution}
    \bbni \\
    \begin{enumerate}
        \item To show that $D_{2n} = \langle s, sr \mid r^n = s^2 = 1, rs = sr^{-1}\rangle$ gives $\langle a, b \mid a^2 = b^2 = (ab)^n = 1 \rangle$ if $a = s, b = sr$: 
            \begin{enumerate}
                \item $s^2 = 1 \implies a^2 = 1$, directly from presentations. 
                \item Every element of $D_{2n}$ that is not a power of $r$ has order 2 (from question 2) then, \\
                \[(sr)^2 = 1 \implies b^2 = 1\] 
                \item $s \cdot sr = r$ and $r^n = 1$ \\ 
                $(s\cdot sr)^n = s^{2n}r^n =e\cdot r^n = 1$. \\ 
                $\implies (ab)^n = 1$
            \end{enumerate}
            Hence relations for $a, b$ follow from relations for $s,sr$. 

        \bbni 
        
        \item To show that $\langle a, b \mid a^2 = b^2 = (ab)^n = 1 \rangle$ gives $D_{2n} = \langle s, sr \mid r^n = s^2 = 1, rs = sr^{-1}\rangle$ if $a = s, b = sr$: 
        \begin{enumerate}
            \item $a^2 = 1 \implies s^2 = 1$, directly from presentations. 
            \item $(ab)^n = 1 \implies (s\cdot sr)^n = (s^2r)^n = (r)^n  = 1$.
            \item $aba = s \cdot sr \cdot s = s^2 rs = rs$, and \\
            $a(ab)^{-1} = aa^{-1}b^{-1} = s \cdot s^{-1}\cdot (sr)^{-1} = sr^{-1}$ \\
            Thus, 
            $aba = a(ab)^{-1} \implies rs = sr^{-1}$
        \end{enumerate}
        Hence relations for $s, sr$ follow from relations for $a, b$. 
    \end{enumerate}
\end{solution}

\newpage
\begin{problem}{4}
    Prove that if $\sigma$ is the m-cycle $(a_1a_2\ldots a_m)$, then for all \\ $i \in \{1, 2, \ldots, m\},  \text{ } \sigma^i (a_k) = a_{k+i}$, where $k + i$ is replaced by its least residue mod $m$ when $k+i > m$. Deduce that $\mid \sigma \mid = m$.     
\end{problem}

\begin{solution}
    \bbni 
    \bbni 
    In a m-cycle, $\sigma(a_k) = a_{k+1}$ with the exception that $\sigma(a_m) = a_1$. To prove $\sigma^i(a_k) = a_{k+i}$ through induction, let's consider: \\  
    $i = 1$\\
    \[\sigma^1(a_k) = a_{k+1}\]
    This is true by definition. If $k = m$, 
    \[\sigma^1(a_m) = a_{m+1}\] 
    But by definition of m-cycle, $\sigma(a_m) = a_1$, so 
    \[\sigma^1(a_m) = a_{m+1} = a_1\]
    which is the least positive residue mod m for m + 1. \\
    Inductive Case: Assuming $\sigma^i (a_k) = a_{k+i \text{ (least positive residue mod m)}}$ holds for $i$, 
    \begin{align*}
        \sigma^{i+1} (a_k) &= \sigma(\sigma^i(a_k)) \\
        \sigma^{i+1} (a_k) &= \sigma(a_{k+i}) \\
        \sigma^{i+1} (a_k) &= a_{k+i+1}\\    
    \end{align*}
    It also holds for $i+1$, and $k+i$ is replaced by it's least positive residue mod m. \\ 
    The identity for $\sigma$ would be a cycle such that no permutation occurs: $\sigma^x(a_k) = a_k$. But we know that $\sigma^x(a_k) = a_{k+x}$, so the lease positive residue mod m for $x$ should be m, i.e., $x \in \{m, 2m, \ldots\}$. But order by definition is the least positive integer that gives the identity element upon applying the group action that number of times. So $x = m \implies \mid \sigma \mid = m$. 
\end{solution}

\newpage

\begin{problem}{5}
    Let $\sigma$ be the m-cycle $(1 \space 2 \ldots m).$ Show that $\sigma^i$ is also an m-cycle if and only if $i$ is relatively prime to $m$. 
\end{problem}

\begin{solution}
    \bbni
    \bbni    
    Showing that $(i, m) \neq 1 \implies$ not m-cycle. \\
    From the question above we know that the permutations would look like: 
    \[1 \to i+1,  \, i+ 1 \to 2i + 1, \, 2i +1 \to 3i + 1, \ldots\]
    Then suppose there exists a $k$ such that $ki + 1 \to 2$
    \begin{align*}
        ki + 1 &\equiv 2 \text{ (mod m)} \\
        ki &\equiv 1 \text{ (mod m)}
    \end{align*}
    But we know that it is not possible for numbers that are not relatively prime (from HW0, Problem 4). Hence, for no will $\sigma^i(a_1) = a_2$ or $\sigma^i(a_k)$ will never equal $a_{k+1}$. So there will at least be 2-disjoint cycles, and $\sigma^i$ is not an m-cycle. \\

    If $(i, m) \neq 1 \implies$ not m-cycle then by contrapositive m-cycle $\implies (i, m) = 1.$
    \\

    Showing that $(i, m) = 1 \implies$ m-cycle. \\

    We know from HW0, Problem 5 that $ki \equiv 1 \text{ (mod m)}$ exists, so for some $k, \, \sigma^i(a_1) = a_2$ or $\sigma^i(a_k)$ will equal $a_{k+1}, \, \text{(mod m) } \forall k \in \{ 1, 2, \ldots, m\} $, hence the cycle will be m elements long. 
 \end{solution}

 \newpage

\begin{problem}{6}
    Show that an element has order 2 in $S_n$ if and only if its cycle decomposition is a product of commuting 2-cycles. 
\end{problem}

\begin{solution}
    \bbni \\
    Showing that an element has order 2 in $S_n \implies$ that the cycle decomposition is a product of commuting 2-cycles. 
    \bbni 
    Let $\sigma \in S_n$. $\sigma$ can be expressed as a product of disjoint commuting cycles such that: 
    \[\sigma = \sigma_1\sigma_2\ldots\sigma_m\]
    If $\mid \sigma \mid = 2$, 
    \[\sigma^2 = (\sigma_1 \sigma_2\ldots \sigma_m)^2= e\]
    \[\sigma^2 = \sigma_1^2 \sigma_2^2\ldots \sigma_m^2 = e\]
    $\implies \sigma_i^2 = e \implies\mid \sigma_i \mid = 2$
    And we know from Problem 4 that for an m-cycle the order of $\sigma$ is m, thus the length of $\sigma_i$ 2.  \\ 
    
    Showing that the product of commuting 2-cycles $\implies$ an element has order 2 in $S_n$
    Let n be the product of the commuting 2-cycles, such that: 
    \[n = n_1n_2\ldots n_m\]
    Again from Problem 4, we know that for an m-cycle the order of an n-cycle is n. Then for the 2-cycles in n, $\mid n_i \mid = 2, \, \forall 1 \leq i \leq m$. 
    Also because the cycles commute, $e = n_1^2n_2^2\ldots n_m^2$ can be written as $e = (n_1n_2\ldots n_m)^2 = n^2.$
    Thus, the product of commuting 2-cycles is an element of order 2 in $S_n$. 
\end{solution}
\bbni 
\begin{problem}{7}
    Show that if $n$ is not prime then $\Z /n \Z$ is not a field. 
\end{problem}

\begin{solution}
\bbni \\
For $\Z/n\Z$ to be a field, $(\Z/n\Z, +)$ should be an abelian group and $((\Z/n\Z - \{0\}), \cdot)$ should also be an abelian group. If $((\Z/n\Z - \{0\}), \cdot)$ is an abelian group it must contain the identity element $e$ and $c^{-1} \, \forall c: cc^{-1} = e.$, and follow the axioms of associativity and commutativity. 
\begin{enumerate}
    \item In $((\Z/n\Z - \{0\}), \cdot)$ $e = 1$ because $\forall c \in \Z/n\Z^\times \, c \cdot 1 = c.$ 
    \item For $c$ to have an inverse in the group, there must exist a $a$ such that $a \cdot c = 1$ (mod n). 
    Again from HW0 Problems 4 $\&$ 5, we know that $a \cdot c \equiv 1$ (mod n) is only possible if $1 \leq a \leq n$ is co-prime with n. But if $\forall a < n: (a, n) = 1$ then n is prime. 
\end{enumerate}
\end{solution}

\newpage

\begin{problem}{8}
    Let $H(F) = \left \{ \begin{pmatrix} 1 & a & b \\ 0 & 1 & c \\ 0 & 0 &1 \end{pmatrix} \mid a,b, c \in F \right \}$\textemdash   called the Heisenberg group over F. Let $X= \begin{pmatrix} 1 & a & b \\ 0 & 1 & c \\0 & 0 & 1 \end{pmatrix}$ and $Y = \begin{pmatrix} 1 & d & e \\ 0 & 1 & f \\ 0 & 0 & 1 \end{pmatrix}$ be elements of $H(F).$

    \begin{enumerate}
        \item Compute the matrix product $XY$ and deduce that the $H(F)$ is closed under matrix multiplication. Exhibit explicit matrices such that $XY \neq YX$ (so that $H(F)$ is always non-abelian.
        \item Find an explicit formula for the matrix inverse $X^{-1}$ and deduce that $H(F)$ is closed under inverses. 
        \item Prove the associative law of $H(F)$ and deduce that $H(F)$ is a group of order $\mid F \mid ^3$. (Do not assume that matrix multiplication is associative). 
        \item Find the order of each element of the finite group $H(\Z/2\Z)$.
        \item Prove that every nonidentity element of the group $H(\R)$ has infinite order.
    \end{enumerate}    
\end{problem}

\begin{solution}
\bbni 
    \begin{enumerate}
        \item
        \[ XY = \begin{pmatrix}
            1 & a & b \\ 0 & 1 & c \\ 0 & 0 & 1
        \end{pmatrix} \begin{pmatrix}
            1 & d & e \\ 0 & 1 & f \\ 0 & 0 & 1
        \end{pmatrix} = \begin{pmatrix}
            1 & d + a & e + af + b \\ 0 & 1 & f+c \\ 0 & 0 & 1
        \end{pmatrix}
        \]

        
        \[ YX = \begin{pmatrix}
            1 & d & e \\ 0 & 1 & f \\ 0 & 0 & 1
        \end{pmatrix}\begin{pmatrix}
            1 & a & b \\ 0 & 1 & c \\ 0 & 0 & 1
        \end{pmatrix} = \begin{pmatrix}
            1 & d + a & e + dc + b \\ 0 & 1 & f+c \\ 0 & 0 & 1
        \end{pmatrix}
        \]

        \[af \neq dc \implies XY \neq YX\]

        Thus the group is not abelian. 

        The results of both $XY$ and $YX$ give us matrices expressed using sums of $a, b, c, d, e, f \in F$ (from definition). Because the elements are in field $F$, by additive closure of field $F$ the sum of the elements should also be in $F$, then the resulting matrices belong to $H(F)$. Thus, the $H(F)$ is closed under matrix multiplication. 

        \item $I_n = XX^{-1}$, let $X^{-1} = Y$ then: 
        \[ XY = \begin{pmatrix}
            1 & a & b \\ 0 & 1 & c \\ 0 & 0 & 1
        \end{pmatrix} \begin{pmatrix}
            1 & d & e \\ 0 & 1 & f \\ 0 & 0 & 1
        \end{pmatrix} = \begin{pmatrix}
            1 & d + a & e + af + b \\ 0 & 1 & f+c \\ 0 & 0 & 1
        \end{pmatrix}
        \]
        but $I_n = XY$ 
        \[ \begin{pmatrix}
            1 & 0 & 0 \\ 0 & 1 & 0 \\ 0 & 0 & 1
        \end{pmatrix} = \begin{pmatrix}
            1 & d + a & e + af + b \\ 0 & 1 & f+c \\ 0 & 0 & 1
        \end{pmatrix}
        \]

        \begin{enumerate}
            \item $d + a = 0$
            $\implies d = - a$
            \item $f + c = 0 $
            $\implies f = - c$
            \item $e + af + b = 0$
            $\implies e = - a(-c) - b = ac  - b$
        \end{enumerate}
        
        \[Y = X^{-1} = \begin{pmatrix}
            1 & -a & ac - b \\ 0 & 1 & -c \\ 0 & 0 & 1
        \end{pmatrix}\]

        $\implies$ for any $X, \exists \, X^{-1} \in H(F) \implies  H(F)$ is closed under inverses. 

        \item To see if $H(F)$ is associative, consider three matrices \[X = \begin{pmatrix}
            1 & a & b \\ 0 & 1 & c \\ 0 & 0 & 1
        \end{pmatrix}, \,Y = \begin{pmatrix}
            1 & d & e \\ 0 & 1 & f \\ 0 & 0 & 1
        \end{pmatrix}, \,Z = \begin{pmatrix}
            1 & g & h \\ 0 & 1 & i \\ 0 & 0 & 1
        \end{pmatrix}\]
        then
        \[(XY)Z =\begin{pmatrix}
            1 & d + a & e + af + b \\ 0 & 1 & f+c \\ 0 & 0 & 1
        \end{pmatrix}\begin{pmatrix}
            1 & g & h \\ 0 & 1 & i \\ 0 & 0 & 1
        \end{pmatrix} = \begin{pmatrix}
            1 & d + a + g & h + i(d + a) + e + af + b \\ 0 & 1 & f+c + i\\ 0 & 0 & 1
        \end{pmatrix} \]

        and 

        \[X(YZ) =\begin{pmatrix}
            1 & a & b \\ 0 & 1 & c \\ 0 & 0 & 1
        \end{pmatrix}\begin{pmatrix}
            1 & g + d & h + di + e \\ 0 & 1 & i + f \\ 0 & 0 & 1
        \end{pmatrix} = \begin{pmatrix}
            1 & a + g + d & h + di + e + a(i+f) + b \\ 0 & 1 & c + f + i\\ 0 & 0 & 1
        \end{pmatrix} \]

        By associativity of $F$, and commutativeness of the multiplicative group operation: 
        \begin{enumerate}
            \item d + a + g = a + g + d 
            \item f + c + i = c + f + i 
            \item h + id + ia + e + af = h + di + e + ai + af + b
        \end{enumerate}
        $\implies (XY)Z = X(YZ) \implies H(F)$ is associative. 

        From subparts 1, 2, and 3, we know that $H(F)$ is associative, and is closed under matrix multiplication, and has inverses for all $X \in H(F)$. It's can also be trivially seen that $I_n \in H(F): a, b, c = 0$. So $H(F)$ is a group. \\ 
        The order of a group is its cardinality. For any $X  \in H(F)$, $X$ can have any combination of $a, b, c: a, b, c \in F$. If the order of the $F$ is $\mid F \mid$ then $a, b, c$ can each have $\mid F \mid$ possible values ($\mid F \mid \cdot \mid F \mid \cdot \mid F \mid$). So the order of $\mid H(F) \mid = \mid F \mid ^3$. 

        \item Let $X \in H(\Z /2 \Z) =  \begin{pmatrix}
            1 & a & b \\ 0 & 1 & c \\ 0 & 0 & 1
        \end{pmatrix} $. And we know that $\Z/2\Z = \{0, 1\}$. Then, \\ 
        \[\begin{pmatrix}
            1 & a & b \\ 0 & 1 & c \\ 0 & 0 & 1
        \end{pmatrix}\begin{pmatrix}
            1 & a & b \\ 0 & 1 & c \\ 0 & 0 & 1
        \end{pmatrix} = \begin{pmatrix}
            1 & a + a & b + ac + b \\ 0 & 1 & c + c \\ 0 & 0 & 1
        \end{pmatrix} = \begin{pmatrix}
            1 & 2a & 2b + ac \\ 0 & 1 & 2c \\ 0 & 0 & 1
        \end{pmatrix} = \begin{pmatrix}
            1 & 0 & ac \\ 0 & 1 & 0 \\ 0 & 0 & 1
        \end{pmatrix} \text{ (mod 2)}\]

        If $a = 0$ or $c = 0$, $\mid X \mid = 2$. Else, 

         \[\begin{pmatrix}
            1 & 0 & ac \\ 0 & 1 & 0 \\ 0 & 0 & 1
        \end{pmatrix} \begin{pmatrix}
            1 & 0 & ac \\ 0 & 1 & 0 \\ 0 & 0 & 1
        \end{pmatrix}  = \begin{pmatrix}
            1 & 0 & 0 \\ 0 & 1 & 0 \\ 0 & 0 & 1
        \end{pmatrix}\]


        $\mid X \mid \leq 4 \implies \mid X \mid \bigl\vert  4$ but we know that $1, 2 \neq \mid X \mid \implies \mid X \mid = 4. $  \\ 
        Also if $X = I_n \implies \mid X\mid = 1$
        The order of elements in  $H(\Z/2\Z) = \{1, 2, 4\}$. 

        \item Let $X \in H(\R) =  \begin{pmatrix}
            1 & a & b \\ 0 & 1 & c \\ 0 & 0 & 1
        \end{pmatrix} $. \\
        If $a = b = c = 0$ then $\mid X \mid = 1.$ Else, 
        let $\mid X \mid = n$, where $n$ is some positive integer and $y$ is any integer. Then by working out $X^n$ for $n = {1, 2, 3, \ldots} $ we can observe the following pattern: 
        \[X^n = \begin{pmatrix}
            1 & na & nb + yac \\ 0 & 1 & nc \\ 0 & 0 & 1
        \end{pmatrix} \]
        For $\mid X \mid = n$, $na = nc = nb + yac = 0$. So we have the following cases: 
        \begin{enumerate}
            \item If $a \neq 0 $ or $c \neq 0$ then $na$ and $nc$ respectively cannot be zero for any \\ $x \in \R \implies \mid X \mid = \infty$ 
            \item If $a = 0$ and $c = 0$ then $na = nc = yac = 0$. Still, for the resulting matrix to be an identity matrix $nb = 0$. If $b \neq 0$ then $nb$ cannot be zero for any \\ $x \in \R \implies \mid X \mid = \infty$ 
        \end{enumerate}
        So $X$ only has finite order if $a = b = c = 0$ but if that is the case then $X$ is an identity matrix. Thus, all nonidentity elements of the group $H(\R)$ have infinite order.       
    \end{enumerate}
\end{solution}

\newpage
\begin{problem}{9}
    If $\varphi : G \rightarrow H$ is an isomorphism, prove that $\mid \varphi(x) \mid = \mid x \mid $ for all $x \in G$. Deduce that any two isomorphic groups have the same number of elements of order $n$ for each $n \in \Z^+$. Is the result true if $\varphi$ is only assumed to be a homomorphism?
\end{problem}

\begin{solution}
\bbni \\
    If $\varphi: G \to H$ is an isomorphism then we know that 
    $ker(\varphi) = \{e_G\} \implies \varphi(e_G) = e_H$. Now, let $\mid g \mid = n, g \in G$. 
    Then, 
    \[ \varphi (g^n) = \varphi(e_G) = e_H \]
    But by definition of a homomorphism we also know, 
    \[\varphi(g^n) = \varphi (\underbrace{g \ldots g)}_{n-times} = \underbrace{\varphi(g)\ldots \varphi(g)}_{n-times} =  (\varphi(g))^n \] 

    \[\implies \varphi(g^n) = (\varphi(g))^n = e_H\]
    \[\implies \mid \varphi(g) \mid \leq \mid g \mid \]

    Also, because $\varphi: G \to H$ is an isomorphism we know that $\exists \varphi^{-1} : H \to G$ and let $\mid \varphi(g) \mid = m$. Then 
    \[\varphi^{-1}((\varphi(g))^m) = e_H = g^m\]
    \[\implies m \geq \mid g \mid\]
    Now we have, $m \geq n$ and $n \geq m \implies m = n, \forall g \in G.$

    Isomorphic groups are bijective, so if $G$ has $x$ elements of order $n$, and all elements of $G$ that have order $n$ are mapped to elements of $H$ that have order $n$, then $G$ and $H$ have $x$  elements of of order n. \\

    The result is not necessarily true if $\varphi$ is only assumed to be an homomorphism because $\varphi$ doesn't have an inverse, so we only know that $m \leq n.$
\end{solution}

\newpage

\begin{problem}{10}
    Prove that the multiplicative groups $\R - \{0\}$ and $\C - \{0\}$ are not isomorphic.  
\end{problem}

\begin{solution}
    \bbni \\ 
    We know from Q9 that if $\varphi: G \to H$ is an isomorphism then $\mid \varphi(g) \mid  =  \mid g \mid$. Let $G = \R - \{0\}$ and $H = \C - \{0\}$. We know that for $i \in \C, \mid i \mid = 4. $ So if the groups are isomorphic, there must exist an element $x \in \R$ such that $\mid x \mid = 4.$ But for $\R$: 
    \begin{enumerate}
        \item $\mid x \mid = 1$ only for the identity element
        \item $\mid x \mid  = 2$ if $x = -1$. 
        \item  $\forall x \in \R, x > 1$, and $n \in \Z^+, x^n = 1$. \\
        But $x^n = \underbrace{x\ldots x}_{n-times}$ and we can see that if $x > 1 \implies x^n > 1 \implies x^n \neq 1$
        So there $\nexists n: x^n  = 1 \implies \mid x \mid = \infty$.
        \item $\forall x \in \R, x < -1$, and $n \in \Z^+, x^n = 1$. \\
        But $x^n = \underbrace{x\ldots x}_{n-times}$ and we can see that if $x < -1, x^n$ alternates between being $x^n > 1$ (for even powers) and $x^n<-1$(for odd powers). In both cases $x^n \neq 1$
        So there $\nexists n: x^n  = 1 \implies \mid x \mid = \infty$.
        \item  $\forall x \in \R, -1 < x < 1$, and $n \in \Z^+, x^n = 1$. \\
        But $0^n = 0$ always, and the absolute value of everything else raised to any positive integer $n$ would be smaller than the original value. 
        \[abs(x) > abs(x^n) \implies 1 - x^n > 1- x \]              So there $\nexists n: x^n  = 1 \implies \mid x \mid = \infty$.

    \end{enumerate}
    Hence $\mid x \mid = \{1, 2, \infty\}, x\in \R$. So there does not exist any element in $\R$ with order 4, and thus $\sigma$ cannot be an isomorphism.   
\end{solution} 

\newpage
\begin{problem}{11}
    Prove that the additive groups $\Z$ and $\Q$ are not isomorphic. 
\end{problem} 

\begin{solution}
    \bbni \\
    Suppose $\Z$ and $\Q$ are isomorphic, and $\Z = \langle 1\rangle$, where $\Z$ can be generated by adding/subtracting 1 multiple times to itself. Then there must exist some $p/q$ such that $\Q = \langle p/q \rangle$. \\
    But $\exists p/2q \in \Q$ that cannot be generated from $p/q$ by adding/subtracting multiple times it from itself ($p/2q$ is not an integral multiple of $p/q)$. Hence, $\Z$ and $\Q$ are not isomorphic. 
\end{solution}

\bbni

\begin{problem}{12}
    Prove that $D_{24}$ and $S_4$ are not isomorphic.
\end{problem}

\begin{solution}
    \bbni \\ 
    We know from Q9 that if $\varphi: G \to H$ is an isomorphism then $\mid \varphi(g) \mid  =  \mid g \mid$. But from the presentation of a dihedral group we know that that $r, s \in D_{24}: r^{12} = s^2 = e \implies   \mid r \mid = 12$. However, the maximum length an m-cycle in $S_4$ can have is $4,$ thus the maximum order an element in $S_n$ can have is 4. 
    Thus, in $\varphi: D_{24} \to S_4$, $\exists g \in D_{24}: \, \mid \varphi(g) \mid  \neq  \mid g \mid$  $\implies \varphi$ is not isomorphic. 
\end{solution}

\newpage
\begin{problem}{13}
    Let $d \in \Z$ nonsquare. Prove that $\Q(\sqrt{d}) = \{ a + b\sqrt{d} \in \C \mid a, b \in \Q\}$ is a field under addition and multiplication of complex numbers. Hint: You can take for granted that $\sqrt{d}$ is irrational.
\end{problem}

\begin{solution}
    \bbni 
    \bbni 
    We know that $\Q (\sqrt{d}) \in \C$. So $\Q(\sqrt{d})$ inherits associativity and commutativity from the field $\C$. Then we only need to show that $\Q(\sqrt{d})$ is closed under inverses, multiplication, and addition to prove that $\Q(\sqrt{d})$ is a field. 
    \begin{enumerate}
        \item if $a, b = 0$ \\
            $a + b\sqrt{d} = 0 + 0 = 0$ \\ 
            $\implies 0 \in \Q(\sqrt{d})$
            \bbni 
            if $a = 1, b = 0$ \\
            $a + b\sqrt{d} = 1 + 0 = 1$ \\ 
            $\implies 1 \in \Q(\sqrt{d})$ \\
            Hence, $\Q(\sqrt{d})$ is closed under inverses. 
        \item Let $x + y \sqrt{d}$ be the additive inverse of $a+ b \sqrt{d}$ then, \\
        \begin{align*}
            a + b\sqrt{d} + x + y\sqrt{d} &= 0\\
            a + x + b\sqrt{d} + y\sqrt{d} &= 0 \\
            (a + x)1 +  (b + y)\sqrt{d} &= 0 \\ 
            \implies a = -x \\ 
            \implies b = -y \\             
        \end{align*}

        $a, b \in \Q$, so the additive inverse of $a + b\sqrt{d}$  is  $(-a - b\sqrt{d})$. Thus, $(\Q\sqrt{d})$ is closed under addition. 

        \newpage
        \item Let $x$ be the multiplicative inverse of $a+ b \sqrt{d}$ then, \\
        \begin{align*}
           ( a + b\sqrt{d} )(x + y\sqrt{d})&= 1\\
           x + y\sqrt{d}&= 1 / ( a + b\sqrt{d} ) \\
           x + y\sqrt{d} &= \frac{1}{a + b \sqrt{d}} \cdot \frac{a - x + y\sqrt{d}\sqrt{d}}{a - b\sqrt{d}}\\ 
           x + y\sqrt{d} &= \frac{a - b\sqrt{d}}{a^2 - b^2d} \\
           x + y\sqrt{d} &= \frac{a}{a^2 - b^2d} + \frac{-b}{a^2 -b^2d} \cdot \sqrt{d} \\
           \implies x = \frac{a}{a^2 - b^2d}, &\qquad y = \frac{-b}{a^2 -b^2d}  
        \end{align*}
        We can see that $x, y \in \Q$ because $a, b \in \Q$. So,
        \[ (a+b\sqrt{d})^{-1} = \frac{a}{a^2 - b^2d} + \frac{-b}{a^2 -b^2d} \cdot \sqrt{d} \in \Q(\sqrt{d})\]        
        Thus, $\Q(\sqrt{d})$ is closed under multiplication. 
    \end{enumerate}
    Hence, $\Q(\sqrt{d})$ is a field under addition and multiplication of complex numbers. 
\end{solution}

\newpage
\begin{problem}{14}
    Remind yourself (or learn about) the field of complex numbers $\C = \{ z = x + iy: x + y \in \R, i^2 = -1\}.$ Prove that the complex conjugation $z = x + iy \mapsto \Bar{z} = x - iy $ is a homomorphism of the additive group $\C \mapsto \C$ and the multiplicative group $\C^\times \mapsto \C^\times$. Prove that the absolute value $z \mapsto \mid z \mid = \sqrt{z\Bar{z}}$ is a homomorphism of the multiplicative groups $\C^\times \mapsto \R^\times$. Let $U = \{ z \in \C \, : \ \mid z \mid = 1 \}$ be the unit circle. Prove that the map $\R \mapsto U$ defined by $\theta \mapsto e^{i\theta}$ is a group homomorphism. 
\end{problem}

\begin{solution}
    \bbni \\
    \begin{enumerate}
        \item To show that $\sigma : \C \to \C$ are holomorphic where $\sigma$ is complex conjugation we need to show that $\varphi(ab) = \varphi(a)\varphi(b)$ where $a, b \in \C.$ Let $a = x + yi, b  = p + qi$.
        \begin{align*}
            \varphi(ab) &= \varphi(a)\varphi(b) \\ 
            \varphi(x + yi + p + qi) &= \varphi(x+yi) + \varphi(p + qi) \\ 
            \varphi ((x+p) + (y+q)i) &= (x- yi) + (p -qi) \\
            (x + p) - (y+q)i &=(x + p ) - (y+q)i \\
        \end{align*}
        Thus, $\sigma$ is a homomorphism from $\C \to \C$. 

        \item To show that $\sigma : \C^\times \to \C^\times$ are homomorphic where $\sigma$ is the absolute value of $z$ we need to show that $\varphi(ab) = \varphi(a)\varphi(b)$ where $a, b \in \C.$ Let $a = x + yi, b  = p + qi$.
        \begin{align*}
            \varphi(ab) &= \varphi(a)\varphi(b) \\ 
            \varphi((x + yi) ( p + qi)) &= \varphi(x+yi) \cdot \varphi(p + qi) \\ 
            \varphi (xp + pyi + xqi - yq) &= (x- yi) \cdot (p -qi) \\
            \varphi ((xp -yq) + (py + xq)i) &= (x- yi) \cdot (p -qi) \\
            \varphi((xp - yq) + (py +xq)i) &= xp - pyi -xqi - yq \\
            xp - yq - pyi -xqi &= xp -pyi -xqi -yq 
        \end{align*}
        Thus, $\sigma$ is a homomorphism from $\C^\times \to \C^\times$. 

        \item To show that $\sigma : \C^\times \to \R^\times$ are holomorphic where $\sigma$ is the absolute value of $z$ we need to show that $\varphi(ab) = \varphi(a)\varphi(b)$ where $a, b \in \C.$ Let $a = x + yi, b  = p + qi$.
        \begin{align*}
            \varphi(ab) &= \varphi(a)\varphi(b) \\ 
            \varphi((x + yi) ( p + qi)) &= \varphi(x+yi) \cdot \varphi(p + qi) \\ 
            \varphi (xp + pyi + xqi -yq) &= (x^2 + y^2) \cdot (p^2 +q^2) \\
            \varphi ((xp-yq) + (py + xq)i) &= (x^2 + y^2) \cdot (p^2 +q^2) \\
            (xp-yq)^2 + (py + xq)^2 &= (x^2 + y^2) \cdot (p^2 +q^2)\\ 
            x^2p^2 + y^2q^2 -2xypq + p^2y^2 + x^2q^2 +2xypq &= x^2p^2 + y^2p^2 +x^2q^2 + y^2q^2 \\
            x^2p^2 + y^2q^2 + p^2y^2 + x^2q^2 &= x^2p^2 + y^2p^2 +x^2q^2 + y^2q^2
        \end{align*}
        Thus, $\sigma$ is a homomorphism from $\C^\times \to \R^\times$. 

        \item To show that $\sigma : \R \to U$ are homomorphic where $\sigma(\theta) = e^{i\theta}$  we need to show that $\varphi(ab) = \varphi(a)\varphi(b)$ where $a, b \in \R.$
        \begin{align*}
            \varphi(ab) &= \varphi(a)\varphi(b) \\ 
            \varphi(a+b) &= e^{ia}e^{ib} \\
            e^{i(a+b)} &= e^{ia+ib} \\
            e^{ia+ib} &= e^{ia+ib}
        \end{align*}
        Thus, $\sigma$ is a homomorphism from $\R \to U$. 

    \end{enumerate}
\end{solution}
\end{document}