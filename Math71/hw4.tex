\documentclass[12pt]{article}

\usepackage{fullpage}
\usepackage{mdframed}
\usepackage{colonequals}
\usepackage{algpseudocode}
\usepackage{algorithm}
\usepackage[most, breakable]{tcolorbox}
\usepackage[all]{xy}
\usepackage{proof}
\usepackage{mathtools}
\usepackage{bbm}
\usepackage{amssymb}
\usepackage{amsthm}
\usepackage{amsmath}
\usepackage{amsxtra}
\usepackage{enumitem}
\newcommand{\bb}{\mathbb}


\newtheorem{theorem}{Theorem}[section]
\newtheorem{theorem*}{Theorem}
\newtheorem{definition}[theorem]{Definition}
\newtheorem{corollary}{Corollary}[theorem]
\newtheorem{lemma}[theorem]{Lemma}
\newtheorem{prop}[theorem]{Proposition}
\newtheorem{remark}[theorem]{Remark}


\newtheorem*{exercisehelper}{Exercise.}
\newenvironment{exercise}[1]{%
  \IfBlankTF{#1}
    {\renewcommand{\exercisehelper}{\textbf{Exercise} \unskip}}
    {\renewcommand\exercisehelper{\textbf{Exercise #1}}}%
  \exercisehelper
}{\endexercisehelper}

\theoremstyle{remark}
\newtheorem*{solution}{Solution}
\newcommand{\mathcat}[1]{\textup{\textbf{\textsf{#1}}}} % for defined terms

\newenvironment{problem}[1]
{ \begin{tcolorbox}[breakable]\noindent\textbf{Problem #1}.}
{\vskip 6pt \end{tcolorbox}}

\newenvironment{enumalph}
{\begin{enumerate}\renewcommand{\labelenumi}{\textnormal{(\alph{enumi})}}}
{\end{enumerate}}

\newenvironment{enumroman}
{\begin{enumerate}\renewcommand{\labelenumi}{\textnormal{(\roman{enumi})}}}
{\end{enumerate}}

\newcommand{\defi}[1]{\textsf{#1}} % for defined terms



\setlength{\hfuzz}{4pt}

\let\H\relax
\let\P\relax
\newcommand{\H}{\mathbb H}
\newcommand{\P}{\mathbb P}
\newcommand{\C}{\mathbb C}
\newcommand{\N}{\mathbb N}
\newcommand{\Q}{\mathbb Q}
\newcommand{\R}{\mathbb R}
\newcommand{\Z}{\mathbb Z}
\newcommand{\F}{\mathbb F}
\newcommand{\br}{\mathbf{r}}
\newcommand{\RP}{\mathbb{RP}}
\newcommand{\CP}{\mathbb{CP}}
\newcommand{\nbit}[1]{\{0, 1\}^{#1}}
\newcommand{\bits}{\{0, 1\}^{n}}
\newcommand{\bbni}{\bigbreak \noindent}
\newcommand{\norm}[1]{\left\vert\left\vert#1\right\vert\right\vert}
\newcommand{\dbar}{\overline{\partial}}
\let\d\relax
\newcommand{\d}{\partial}
\newcommand{\calO}{\mathcal{O}}
\newcommand{\calF}{\mathcal{F}}
\newcommand{\calG}{\mathcal{G}}
\newcommand{\calH}{\mathcal{H}}
\newcommand{\calE}{\mathcal{E}}
\newcommand{\calC}{\mathcal{C}}
\newcommand{\calD}{\mathcal{D}}

\let\1\relax
\newcommand{\1}{\mathbf{1}}
\newcommand{\fr}[2]{\left(\frac{#1}{#2}\right)}
\newcommand{\todo}[1]{\textcolor{red}{\textbf{TODO:} #1}}
\newcommand{\vecz}{\mathbf{z}}
\newcommand{\vecr}{\mathbf{r}}
\DeclareMathOperator{\Cinf}{C^{\infty}}
\DeclareMathOperator{\Id}{Id}
\DeclareMathOperator{\Ell}{Ell}
\DeclareMathOperator{\CL}{\mathcal{CL}}

\DeclareMathOperator{\Alt}{Alt}
\DeclareMathOperator{\Aut}{Aut}
\DeclareMathOperator{\ann}{ann}
\DeclareMathOperator{\codim}{codim}
\DeclareMathOperator{\End}{End}
\DeclareMathOperator{\Hom}{Hom}
\DeclareMathOperator{\id}{id}
\DeclareMathOperator{\M}{M}
\DeclareMathOperator{\Mat}{Mat}
\DeclareMathOperator{\Ob}{Ob}
\DeclareMathOperator{\opchar}{char}
\DeclareMathOperator{\opspan}{span}
\DeclareMathOperator{\rk}{rk}
\DeclareMathOperator{\sgn}{sgn}
\DeclareMathOperator{\Sym}{Sym}
\DeclareMathOperator{\tr}{tr}
\DeclareMathOperator{\img}{img}
\DeclareMathOperator{\coker}{coker}
\DeclareMathOperator{\Spec}{Spec}
\DeclareMathOperator{\CandE}{CandE}
\DeclareMathOperator{\CandO}{CandO}
\DeclareMathOperator{\argmax}{argmax}
\DeclareMathOperator{\first}{first}
\DeclareMathOperator{\last}{last}
\DeclareMathOperator{\cost}{cost}
\DeclareMathOperator{\dist}{dist}
\DeclareMathOperator{\path}{path}
\DeclareMathOperator{\parent}{parent}
\DeclareMathOperator{\argmin}{argmin}
\DeclareMathOperator{\excess}{excess}
\let\Pr\relax
\DeclareMathOperator{\Pr}{\mathbf{Pr}}
\DeclareMathOperator{\Exp}{\mathbb{E}}
\DeclareMathOperator{\Var}{\mathbf{Var}}
\let\limsup\relax
\DeclareMathOperator{\limsup}{limsup}
%Paired Delims
\DeclarePairedDelimiter\ceil{\lceil}{\rceil}
\let\oldceil\ceil
\renewcommand{\ceil}[1]{\oldceil*{#1}}

\DeclarePairedDelimiter{\floor}{\lfloor}{\rfloor}
\let\oldfloor\floor
\renewcommand{\floor}[1]{\oldfloor*{#1}}





\newcommand{\dagstar}{*}

\newcommand{\tbigwedge}{{\textstyle{\bigwedge}}}
\setlength{\parindent}{0pt}
\setlength{\parskip}{5pt}


\usepackage{listings}
\usepackage{courier}
\usepackage{microtype}


\lstset{
  basicstyle=\footnotesize\ttfamily,
  breaklines=true,
  breakatwhitespace=true
  columns=fullflexible,
  keepspaces=true,
  frame=single,
  escapeinside={(*@}{@*)}
}

\begin{document}

\title{Math 71: Abstract Algebra}

\author{Prishita Dharampal}
\date{}
\maketitle


\textbf{Credit Statement:} Talked to Sair Shaikh'26, and Math Stack Exchange.
\\

\begin{problem}{1}
    \textit{Solvable up to sixty!} Recall that $A_5$, which has order $60$, is simple and nonabelian. The goal is to:
    \begin{enumerate}
        \item[($\aleph$)] Prove that all groups of order $< 60$ are solvable.
    \end{enumerate}
    
    This has two steps. 
    \begin{enumerate}
        \item [($\alpha$)] First, use Jordan–Hölder to prove that if $60$ is the first order of a finite nonabelian simple group, then all groups of order $< 60$ are solvable.
        \item [($\beta$)] Second, prove that every nonabelian group of order $< 60$ is not simple. Since the abelian simple groups are precisely those of prime order, for each composite order $n < 60$, we will try to prove that any group of order $n$ is not simple. For example, we already know that no group of order $p^\alpha$, with $\alpha > 1$, is simple and that no group of order $pq$, with $p$ and $q$ primes, is simple. Prove the following additional criteria on the order of a group for the group to not be simple:
    \end{enumerate}

    \textbf{Hints.} 
   Part (a) follows from a direct application of the congruence conditions in the Sylow theorems. For (b), assume the contrary and consider the possible number of Sylow $r$-subgroups; then use this to count the number of elements of order $r$ (any two Sylow $r$-subgroups intersect only at the identity). Combine this with the number of elements of order $p$ and $q$ to find more elements than the order of the group. For (c) and (d), handle small $k$ using the Sylow congruence conditions, and then for large $k$, consider the permutation representation associated to the conjugation action of $G$ on the set of Sylow $2$-subgroups. For (e) and (f), do the same using the Sylow $3$-subgroups. For (g), if neither the Sylow $2$- nor $7$-subgroups are normal, start counting elements in these subgroups to reach a contradiction (while any two Sylow $7$-subgroups only intersect at the identity, how could Sylow $2$-subgroups intersect?).
    
    Finally, use all the criteria you know to handle every composite order $< 60$. Have fun! How much higher can you go using these same tools?
\end{problem}

\newpage

\begin{problem} {($\alpha$)}
    First, use Jordan–Hölder to prove that if 60 is the first order of a finite nonabelian simple group, then all groups of order $<$ 60 are solvable.
\end{problem}

\begin{solution}
    \bbni 
    \bbni 
    If 60 is the first order of a finite non-abelian simple group, then groups of $< 60$ are either abelian, or simple (they're all finite). 
    \begin{enumerate}
        \item \textit{The group is abelian.} By Jordan–Hölder we know that a composition series for group $G$ exists. Because $G$ is abelian, all subgroups are normal $\implies$ all composition factors are normal $\implies$ group is solvable. 
        \item \textit{The group is non-abelian and non-simple.} By Jordan–Hölder we know that a composition series for group $G$ exists. Assume all groups of order less than $n$ are solvable, for some $n < 59$. Then because group $G$ of order $n$ is not simple, we know that at least one non trivial normal subgroup $N$ exists. Obviously, the orders of $N$ and $G/N$ are both less than $n$. Then by our induction hypothesis we know that both $N$ and $G/N$ are solvable, and hence $G$ is solvable. 
    \end{enumerate}
\end{solution}

\newpage

\begin{problem}{($\beta .a$)}

    If $G$ is a finite group of order $p^{k}m$, with $p \nmid m$ and $m < p$ (more generally, no divisor of $m$ other than $1$ is congruent to $1$ modulo $p$), then $G$ has a normal Sylow $p$-subgroup.

\end{problem}

\begin{solution}
    \bbni 
    \bbni 
    By Sylow's Theorem, we know that $G$ ha a Sylow $p$-subgroup $P$ of order $p^k$. We also know that the following to relations must hold: 
    \[n_p \mid m ,\qquad n_p \equiv 1 \text{ (mod $p$)}\]
    where $n_p$ is the number of Sylow p-groups in $G$. But the only divisor of $m$ congruent to 1 modulo $p$ is 1 (given). Hence $n_p = 1$, and the Sylow $p$-group is unique. \\
    And by Corollary 20 if $P$ is the unique Sylow $p$-group in $G$, then it is also normal in $G$. Hence $G$ has a normal Sylow $p-$subgroup and $G$ is not simple. 
\end{solution}

\newpage 

\begin{problem}{($\beta .$b)}

    If $G$ is a finite group of order $pqr$, where $p, q, r$ are primes with $p < q < r$, then $G$ has a normal Sylow subgroup for at least one of $p, q,$ or $r$.

\end{problem}

\begin{solution}
    \bbni 
    \bbni
    By Sylow's Theorems, we know the following, 
    \begin{enumerate}
        \item $G$ has at least one Sylow $r$-subgroup. 
        \item The number of Sylow $r$-subgroups $n_r$ divides $pq$, and $n_r \equiv 1 \text{ (mod $r$)}$. 
    \end{enumerate}
    Then,  $n_r \mid pq\implies  n_r = 1, p , q, pq$. But $n_r \neq p, q$ because both $p$ and $q$ are less than $r$ and hence not congruent to 1 mod $r$. I.e. $n_r = 1, pq$. \\
    If $n_r = 1$, then by part (a), $G$ has a normal Sylow $r$-subgroup. If $n_r = pq$, then there are $pq$ Sylow $r$-subgroups. 
    Because $r$ is prime, all $pq$ subgroups are cyclic, and intersect only at identity. I.e. there are $pq (r -1)$ elements of order $r$ in $G$, and $pq$ elements of other orders. 

    Similarly, we know, 
     \begin{enumerate}
        \item $G$ has at least one Sylow $q$-subgroup. 
        \item The number of Sylow $q$-subgroups $n_r$ divides $pr$, and $n_q \equiv 1 \text{ (mod $q$)}$. 
    \end{enumerate}
    Then,  $n_q \mid pr\implies  n_q = 1, p , r, pr$. But $n_q \neq p$ because $p$ is a prime number less than $q$ and hence not congruent to 1 mod $q$. 
    
    I.e. $n_q = 1, r, pr$. \\
    If $n_q = 1$, then by part (a), $G$ has a normal Sylow $q$-subgroup. 
    If $n_q = r$, then there are $r$ Sylow $q$-subgroups. And because $q$ is prime, all $r$ subgroups are cyclic, and interesect only at identity. I.e. there are $r (q -1)$ elements of order $q$ in $G$. 
    
    If $n_q = pr$, then there are $pr$ Sylow $q$-subgroups. 
    Because $q$ is prime, all $pr$ subgroups are cyclic, and intersect only at identity. I.e. there are $pr (q -1)$ elements of order $q$ in $G$. 

    If we assume that $n_r \neq 1$ and $n_q \neq 1$, then we have 2 cases, 
    \begin{enumerate}
        \item $n_r = pq, n_q =r$
             \begin{align*}
                    \implies pq &\geq rq - r \\
                    pq &\geq r(q-1)
                \end{align*}
            But $r > q$ and $q-1 > p$, so this is a contradiction and at least one of $n_r$ or $n_q$ needs to be 1.  
        \item $n_r = pq, n_q = pr$
             \begin{align*}
                \implies pq &\geq pqr - pr \\
                q &\geq qr -r \\
                q &\geq r (q - 1)
            \end{align*}
            But $r > q$ and $q-1 > 1$, so this is a contradiction and at least one of $n_r$ or $n_q$ needs to be 1.  
    \end{enumerate}

    Thus, there exists at least 1 normal Sylow $p$-subgroup in $G$, and $G$ is not simple. 
\end{solution}

\newpage

\begin{problem}{($\beta .c$)}

    If $G$ is a finite group of order $2^k \cdot 3$, with $k \ge 1$, then $G$ is not simple.

\end{problem}

\begin{solution}
    \bbni 
    \bbni 
    Consider the Sylow $2$-group of order $2^k$, then $n_2 = 1, 3$. If $n_2 = 1$, then the Sylow subgroup is unique and by Corollary 20 also normal. If $n_2 = 3$, then by Sylow's Second Theorem all 3-Sylow subgroups are conjugate to each other. Let the set of these subgroups be $A = \{P_1, P_2, P_3\}$, then we can represent the conjugation action of $G$ on $A$ by $\{g \in G: gP_ig^{-1} \in S\}$. The permutation representation of this action can be defined as $\varphi: G \to S_{\mid A \mid}$. Then we have the following cases, 
    \begin{enumerate}
        \item If the kernel is not trivial, then $\varphi$ is not injective, and $G$ has a normal subgroup. For $\varphi$ to not be trivial, $| G | \, > \, | S_A  | \implies | G |  \, > 6 \implies k> 1$.
        \item If the kernel is trivial, or $k = 1$, then by part (a) $G$ has a normal subgroup. 
    \end{enumerate}
    Hence, $G$ is not simple. 
\end{solution}

\newpage

\begin{problem}{($\beta .d$)}

    If $G$ is a finite group of order $2^k \cdot 5$, with $k \ge 1$, then $G$ is not simple.

\end{problem}

\begin{solution}
    \bbni 
    \bbni 
    Similar to the argument above, consider the Sylow $2$-group of order $2^k$, then $n_2 = 1, 5$. If $n_2 = 1$, then the Sylow subgroup is unique and by Corollary 20 also normal. If $n_2 = 5$, then by Sylow's Second Theorem all 5-Sylow subgroups are conjugate to each other. Let the set of these subgroups be $A = \{P_1, P_2, P_3, P_4, P_5\}$, then we can represent the conjugation action of $G$ on $A$ by $\{g \in G: gP_ig^{-1} \in S\}$. The permutation representation of this action can be defined as $\varphi: G \to S_{\mid A \mid}$. Then we have the following cases, 
    \begin{enumerate}
        \item If the kernel is not trivial, then $\varphi$ is not injective, and $G$ has a normal subgroup. For $\varphi$ to not be trivial, $| G | \, > \, | S_A  | \implies | G |  \, > 120 \implies k> 4$.
        \item If the kernel is trivial, or $k = 1$, then by part (a) $G$ has a normal subgroup. 
        \item If $k = 2$, then by part (a) $G$ has a normal subgroup. 
        \item If $k = 3$, then the number of 5-Sylow groups $n_5 \mid 8 \implies n_5 = 1, 2, 4, 8$ but only $1 \equiv 1$ (mod $5$). 
        \item If $k=4$, then $\mid G \mid = 80$, and $ 80 \nmid 120$, so $\not \exists$ a subgroup of $S_5$ that $G$ is isomorphic to. I.e $\varphi$ is not injective, has a non-trivial kernel, and hence also a normal subgroup. 
    \end{enumerate}
    Hence, $G$ is not simple. 
\end{solution}

\newpage

\begin{problem}{($\beta .e$)}

    If $G$ is a finite group of order $2^2 \cdot 3^k$, with $k \ge 1$, then $G$ is not simple. For $k = 1$, use part (c).
    
\end{problem}

\begin{solution}
    \bbni
    \bbni
    There are 2 cases, if $k=1$, then $G$ is a finite group of order $12$, and can be represented as $|G| = 2^k\cdot 3$, where $k = 1$. Hence $G$ is not simple by part (c). If $k \geq 2$, then we know  by Sylow's First Theorem that there exists at least one Sylow 3-subgroup of order $3^k$ in $G$. The number of Sylow 3-subgroups $n_3$ divides 4, i.e. $n_3 = 1, 2, 4$. But $n_3$ is also congruent to 1 mod 3, so $n_3$ must be 1. By Corollory 20, we know that if the Sylow p-subgroup is unique it is also normal. Hence, $G$ is not simple. 
\end{solution}

\newpage

\begin{problem}{($\beta .f$)}

    If $G$ is a finite group of order $3^k \cdot 5$, with $k \ge 1$, then $G$ is not simple.

\end{problem}

\begin{solution}
    \bbni 
    \bbni 
    We know  by Sylow's First Theorem that there exists at least one Sylow 3-subgroup of order $3^k$ in $G$. The number of Sylow 3-subgroups $n_3$ divides 5, i.e. $n_3 = 1, 5$. But $n_3$ is also congruent to 1 mod 3, so $n_3$ must be 1. By Corollory 20, we know that if the Sylow p-subgroup is unique it is also normal. Hence, $G$ is not simple. 
\end{solution}

\newpage

\begin{problem}{($\beta . g$)}

    No group of order $56$ is simple.    

\end{problem}

\begin{solution}
    \bbni 
    \bbni 
    The prime factorization of 56 is $2 ^3 \cdot 7$. We know  by Sylow's First Theorem that there exists at least one Sylow 7-subgroup of order $7$ in $G$. The number of Sylow 7-subgroups $n_7$ divides 8, i.e. $n_7 = 1, 2, 4, 8$. But $n_7$ is also congruent to 1 mod 7, so $n_7$ can only be either 1 or 8. If $n_7 = 1$ then by Corollory 20, we know that if a Sylow p-group is unique it is also normal, then we are done. But if $n_7 = 8$, then we know that the conjugates intersect only at identity (cyclic groups), i.e. the number of elements of order 7 (all elements in a cyclic group of prime order have order equal to the prime) is $8 ( 7-1) = 42$. So there exist 8 elements of order not equal to 7. 
    
    Again by Sylow's First Theorem we also know that there exists at least one Sylow 2-subgroup of order $8$ in $G$. The order of none of these elements is 7, because $7 \nmid 8$. Then we can say that there only exists 1 Sylow 2-subgroup. And by Corollory 20, we know that if the Sylow p-subgroup is unique it is also normal. 
    Hence, $G$ is not simple. 
\end{solution}

\newpage

\begin{problem}{($\beta . h$)}

    \textit{Optional.} What is the largest number $N > 60$ you can find such that no group of composite order $n$, with $61 \le n \le N$, is simple?
    
\end{problem}

\begin{solution}
    \bbni 
    \bbni
    Using just the facts we have proved so far, we can show that groups with order upto $71$ are not simple. But if we prove that all groups of order 72 are not simple, then we can go upto 83. 

    \textit{Proof for $| G | = 72$}. \\
     $| G | = 72 = 2^3 \cdot 3^2$. But for this group consider the Sylow 3-subgroup, this subgroup has order 9, and the number of Sylow 3-subgroups $n_3 \mid 8 \implies n_3 = 1, 2, 4, 8$, but $n_3$ is also congruent to 1 mod 3 $\implies n_3 = 1, 4$. If $n_3 = 1$, then a normal subgroup exists and $G$ is not simple. If $n_3 = 4$ then we can define a homomorphism from $\varphi : G \to S_4$, but $| S_4 | < | G|$, i.e $\varphi$ is not injective, i.e the kernel is not trivial, i.e. a normal subgroup exists. 
    Hence, the group of order 72 is not simple. 
\end{solution}



\end{document}